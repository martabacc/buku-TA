\chapter{UJI COBA DAN EVALUASI}

Pada bab ini dijelaskan tentang uji coba dan evaluasi dari implementasi yang telah dilakukan pada tugas akhir ini.

\section{Lingkungan Uji Coba}

Linkungan uji coba yang digunakan adalah salah satu sistem yang digunakan situs penilaian daring SPOJ, yaitu kluster \textit{Cube} dengan spesifikasi sebagai berikut:

\begin{enumerate}
	\item Perangkat Keras.
	\begin{itemize}
		\item \textit{Processor} Intel(R) Pentium G860 CPU @ 3GHz.
		\item \textit{Memory} 1536 MB.
	\end{itemize}
	\item Perangkat Lunak.
	\begin{itemize}
		\item \textit{Compiler} clang 4.0.
	\end{itemize}			
\end{enumerate} 

\section{Uji Coba Kebenaran}

Uji coba kebenaran dilakukan dengan mengirimkan kode sumber program ke dalam situs penilaian daring SPOJ dan melakukan hasil uji coba kasus sederhana dengan langkah-langkah sesuai dengan algoritma yang telah dirancang dengan keluaran sistem. Permasalahan yang diselesaikan adalah \problem{}. Hasil uji coba dengan waktu terbaik pada situs SPOJ ditunjukkan pada Gambar \ref{figure:best_submission}.

Selain itu, dilakukan pengujian sebanyak 30 kali pada situs penilaian daring SPOJ untuk melihat variasi waktu dan memori  yang dibutuhkan program. Hasil uji coba sebanyak 30 kali dapat dilihat pada Gambar \ref{figure:submission1}, \ref{figure:submission2} dan \ref{figure:chart}.

Dari hasil uji coba pada Gambar \ref{figure:submission1}, \ref{figure:submission2} dan \ref{figure:chart}  dapat kita tarik beberapa informasi seperti yang tertera pada Tabel \ref{tab:statistik}.

\begin{table}
	\centering
	\begin{tabular}{|l|l|} \hline
		Waktu Maksimal & $ 1,02 $ detik\\ \hline
		Waktu Minimal & $ 0,98 $ detik\\ \hline
		Waktu Rata-Rata & $ 1,004 $ detik\\ \hline
		Memori Maksimal & $ 19 $ MB\\ \hline
		Memori Minimal & $ 19 $ MB\\ \hline
		Memori Rata-Rata & $ 19 $ MB\\ \hline
	\end{tabular}
	\caption{Kecepatan maksimal, minimal dan rata-rata dari hasil uji coba pengumpulan 30 kali pada situs pengujian daring SPOJ}
	\label{tab:statistik}
\end{table}

Berdasarkan Tabel \ref{tab:statistik}, dari percobaan yang dilakukan,  didapatkan waktu eksekusi rata-rata $ 1,004 $ detik dan waktu maksimal $ 1,02 $ detik. Waktu eksekusi tersebut 6 kali lebih cepat dari batas waktu eksekusi yang tertera pada deskripsi permasalahan, yaitu $ 6,459 $ detik.

Selanjutnya akan dilakukan uji coba menggunakan kasus uji yang diberikan pada deskripsi \problem{}. Pada deskripsi permasalahan, terdapat dua kasus. Kasus pertama yaitu kasus di mana nilai dari \textit{string} $ ad1=c $, \textit{string} $ ad2=n $ dan $ X = 1 $.

Sesuai dengan algoritma yang telah dirancang, berdasarkan persamaan \ref{equation:main_answer} yang sudah ditransformasikan ke dalam bentuk \textit{pseudocode} pada Gambar \ref{figure:fungsi_main}, algoritma akan melakukan iterasi variabel $ dist $ dari $ 0 $ hingga $ X $.


Pada awalnya jawaban akhir diinisialisasi dengan nilai $ 0 $. Proses penyelesaian diawali dengan iterasi $ dist=0 $. pada iterasi $ dist = 0 $, nilai jawaban akhir akan ditambahkan dengan $ F_{(c, 1,1)} $, yang merupakan jumlah kombinasi \textit{string} $ orig1 $ tanpa operasi \textit{replace} dengan jarak $ 0 $ terhadap \textit{string} $ ad1 $, yang berdasarkan ilustrasi pada Tabel \ref{tab:f_2_orig1_0_1}, bernilai 1 yang dikalikan dengan $ G_{(n, 1,0)} $, yang merupakan jumlah kombinasi \textit{string} $ orig2 $ dengan operasi \textit{replace} yang memiliki jarak $ 1 $ terhadap \textit{string} $ ad2 $, yang berdasarkan ilustrasi pada Tabel \ref{tab:g_2_orig2_0_1}, bernilai 2. Lalu nilai jawaban akhir akan ditambahkan dengan $ G_{(c, 1,1)} $, yang merupakan jumlah kombinasi \textit{string} $ orig1 $ dengan operasi \textit{replace} dengan jarak $ 0 $ terhadap \textit{string} $ ad1 $, yang berdasarkan ilustrasi pada Tabel \ref{tab:g_2_orig1_0_1}, bernilai 0 yang dikalikan dengan $ F_{(n, 1,0)} $, yang merupakan jumlah kombinasi \textit{string} $ orig2 $ tanpa operasi \textit{replace} yang memiliki jarak $ 1 $ terhadap \textit{string} $ ad2 $, yang berdasarkan ilustrasi pada Tabel \ref{tab:f_2_orig2_0_1}, bernilai 0. Sehingga nilai jawaban dari iterasi $ dist=0 $ adalah 2 dan nilai jawaban akhir sementara adalah $ 2 $.

Berikutnya, pada iterasi $ dist = 1 $, nilai jawaban akhir akan ditambahkan dengan $ F_{(c, 1,0)} $, yang merupakan jumlah kombinasi \textit{string} $ orig1 $ tanpa operasi \textit{replace} dengan jarak $ 1 $ terhadap \textit{string} $ ad1 $, yang berdasarkan ilustrasi pada Tabel \ref{tab:f_2_orig1_1_1}, bernilai 0 yang dikalikan dengan $ G_{(n, 1,1)} $, yang merupakan jumlah kombinasi \textit{string} $ orig2 $ dengan operasi \textit{replace} yang memiliki jarak $ 0 $ terhadap \textit{string} $ ad2 $, yang berdasarkan ilustrasi pada Tabel \ref{tab:g_2_orig2_1_1}, bernilai 0. Lalu nilai jawaban akhir akan ditambahkan dengan $ G_{(c, 1,0)} $, yang merupakan jumlah kombinasi \textit{string} $ orig1 $ dengan operasi \textit{replace} dengan jarak $ 1 $ terhadap \textit{string} $ ad1 $, yang berdasarkan ilustrasi pada Tabel \ref{tab:g_2_orig1_1_1}, bernilai 2 yang dikalikan dengan $ F_{(n, 1,1)} $, yang merupakan jumlah kombinasi \textit{string} $ orig2 $ tanpa operasi \textit{replace} yang memiliki jarak $ 0 $ terhadap \textit{string} $ ad2 $, yang berdasarkan ilustrasi pada Tabel \ref{tab:f_2_orig2_1_1}, bernilai 1. Sehingga nilai jawaban dari iterasi 1 adalah 2 dan nilai jawaban akhir pad kasus di mana \textit{string} $ ad1=c $, \textit{string} $ ad2=n $ dan $ X=1 $ adalah 4.

Kasus berikutnya adalah kasus di mana \textit{string} $ ad1=kbenh $, \textit{string} $ ad2=kbenh $ dan $ X=5 $. Proses yang dilakukan sama seperti pada kasus sebelumnya, yaitu dimulai dengan mengiterasi variable $ dist $ dari $ 0 $ hingga $ X $.

Pada awalnya nilai jawaban akhir diinisialilasi dengan nilai $ 0 $. Pada iterasi $ dist = 0 $, nilai jawaban akhir akan ditambahkan dengan $ F_{(behkn, 31,5)} $, yang merupakan jumlah kombinasi \textit{string} $ orig1 $ tanpa operasi \textit{replace} dengan jarak $ 0 $ terhadap \textit{string} $ ad1 $, yang berdasarkan ilustrasi pada Tabel \ref{tab:f_3_orig1_0_1}, bernilai 1 yang dikalikan dengan $ G_{(behkn, 31,0)} $, yang merupakan jumlah kombinasi \textit{string} $ orig2 $ dengan operasi \textit{replace} yang memiliki jarak $ 5 $ terhadap \textit{string} $ ad2 $, yang berdasarkan ilustrasi pada Tabel \ref{tab:g_3_orig2_0_1} sampai dengan Tabel \ref{tab:g_3_orig2_0_6}, bernilai 8. Lalu nilai jawaban akhir akan ditambahkan dengan $ G_{(behkn, 31,5)} $, yang merupakan jumlah kombinasi \textit{string} $ orig1 $ dengan operasi \textit{replace} dengan jarak $ 0 $ terhadap \textit{string} $ ad1 $, yang berdasarkan ilustrasi pada Tabel \ref{tab:g_3_orig1_0_1} sampai dengan Tabel \ref{tab:g_3_orig1_0_3}, bernilai 0 yang dikalikan dengan $ F_{(behkn, 31,0)} $, yang merupakan jumlah kombinasi \textit{string} $ orig2 $ tanpa operasi \textit{replace} yang memiliki jarak $ 5 $ terhadap \textit{string} $ ad2 $, yang berdasarkan ilustrasi pada Tabel \ref{tab:f_3_orig2_0_1} sampai dengan Tabel \ref{tab:f_3_orig2_0_2}, bernilai 0. Sehingga nilai jawaban dari iterasi 0 adalah 8 dan nilai jawaban akhir hingga pada iterasi 0 adalah 8.

Berikutnya, pada iterasi $ dist = 1 $, nilai jawaban akhir akan ditambahkan dengan $ F_{(behkn, 31,4)} $, yang merupakan jumlah kombinasi \textit{string} $ orig1 $ tanpa operasi \textit{replace} dengan jarak $ 1 $ terhadap \textit{string} $ ad1 $, yang berdasarkan ilustrasi pada Tabel \ref{tab:f_3_orig1_1_1}, bernilai 0 yang dikalikan dengan $ G_{(behkn, 31,1)} $, yang merupakan jumlah kombinasi \textit{string} $ orig2 $ dengan operasi \textit{replace} yang memiliki jarak $ 4 $ terhadap \textit{string} $ ad2 $, yang berdasarkan ilustrasi pada Tabel \ref{tab:g_3_orig2_1_1} sampai dengan Tabel \ref{tab:g_3_orig2_1_5}, bernilai 0. Lalu nilai jawaban akhir akan ditambahkan dengan $ G_{(behkn, 31,4)} $, yang merupakan jumlah kombinasi \textit{string} $ orig1 $ dengan operasi \textit{replace} dengan jarak $ 1 $ terhadap \textit{string} $ ad1 $, yang berdasarkan ilustrasi pada Tabel \ref{tab:g_3_orig1_1_1} sampai dengan Tabel \ref{tab:g_3_orig1_1_3}, bernilai 10 yang dikalikan dengan $ F_{(behkn, 31,1)} $, yang merupakan jumlah kombinasi \textit{string} $ orig2 $ tanpa operasi \textit{replace} yang memiliki jarak $ 4 $ terhadap \textit{string} $ ad2 $, yang berdasarkan ilustrasi pada Tabel \ref{tab:f_3_orig2_1_1} sampai dengan Tabel \ref{tab:f_3_orig2_1_1}, bernilai 0. Sehingga nilai jawaban dari iterasi 1 adalah 0 dan nilai jawaban akhir hingga pada iterasi 1 adalah 8.

Berikutnya, pada iterasi $ dist = 2 $, nilai jawaban akhir akan ditambahkan dengan $ F_{(behkn, 31,3)} $, yang merupakan jumlah kombinasi \textit{string} $ orig1 $ tanpa operasi \textit{replace} dengan jarak $ 2 $ terhadap \textit{string} $ ad1 $, yang berdasarkan ilustrasi pada Tabel \ref{tab:f_3_orig1_2_1}, bernilai 0 yang dikalikan dengan $ G_{(behkn, 31,2)} $, yang merupakan jumlah kombinasi \textit{string} $ orig2 $ dengan operasi \textit{replace} yang memiliki jarak $ 3 $ terhadap \textit{string} $ ad2 $, yang berdasarkan ilustrasi pada Tabel \ref{tab:g_3_orig2_2_1} sampai dengan Tabel \ref{tab:g_3_orig2_2_5}, bernilai 0. Lalu nilai jawaban akhir akan ditambahkan dengan $ G_{(behkn, 31,3)} $, yang merupakan jumlah kombinasi \textit{string} $ orig1 $ dengan operasi \textit{replace} dengan jarak $ 2 $ terhadap \textit{string} $ ad1 $, yang berdasarkan ilustrasi pada Tabel \ref{tab:g_3_orig1_2_1} sampai dengan Tabel \ref{tab:g_3_orig1_2_3}, bernilai 0 yang dikalikan dengan $ F_{(behkn, 31,2)} $, yang merupakan jumlah kombinasi \textit{string} $ orig2 $ tanpa operasi \textit{replace} yang memiliki jarak $ 3 $ terhadap \textit{string} $ ad2 $, yang berdasarkan ilustrasi pada Tabel \ref{tab:f_3_orig2_2_1} sampai dengan Tabel \ref{tab:f_3_orig2_2_1}, bernilai 0. Sehingga nilai jawaban dari iterasi 2 adalah 0 dan nilai jawaban akhir hingga pada iterasi 2 adalah 8.

Berikutnya, pada iterasi $ dist = 3 $, nilai jawaban akhir akan ditambahkan dengan $ F_{(behkn, 31,2)} $, yang merupakan jumlah kombinasi \textit{string} $ orig1 $ tanpa operasi \textit{replace} dengan jarak $ 3 $ terhadap \textit{string} $ ad1 $, yang berdasarkan ilustrasi pada Tabel \ref{tab:f_3_orig1_3_1}, bernilai 0 yang dikalikan dengan $ G_{(behkn, 31,3)} $, yang merupakan jumlah kombinasi \textit{string} $ orig2 $ dengan operasi \textit{replace} yang memiliki jarak $ 2 $ terhadap \textit{string} $ ad2 $, yang berdasarkan ilustrasi pada Tabel \ref{tab:g_3_orig2_3_1} sampai dengan Tabel \ref{tab:g_3_orig2_3_3}, bernilai 0. Lalu nilai jawaban akhir akan ditambahkan dengan $ G_{(behkn, 31,2)} $, yang merupakan jumlah kombinasi \textit{string} $ orig1 $ dengan operasi \textit{replace} dengan jarak $ 3 $ terhadap \textit{string} $ ad1 $, yang berdasarkan ilustrasi pada Tabel \ref{tab:g_3_orig1_3_1} sampai dengan Tabel \ref{tab:g_3_orig1_3_5}, bernilai 0 yang dikalikan dengan $ F_{(behkn, 31,3)} $, yang merupakan jumlah kombinasi \textit{string} $ orig2 $ tanpa operasi \textit{replace} yang memiliki jarak $ 2 $ terhadap \textit{string} $ ad2 $, yang berdasarkan ilustrasi pada Tabel \ref{tab:f_3_orig2_3_1} sampai dengan Tabel \ref{tab:f_3_orig2_3_1}, bernilai 0. Sehingga nilai jawaban dari iterasi 3 adalah 0 dan nilai jawaban akhir hingga pada iterasi 3 adalah 8.

Berikutnya, pada iterasi $ dist = 4 $, nilai jawaban akhir akan ditambahkan dengan $ F_{(behkn, 31,1)} $, yang merupakan jumlah kombinasi \textit{string} $ orig1 $ tanpa operasi \textit{replace} dengan jarak $ 4 $ terhadap \textit{string} $ ad1 $, yang berdasarkan ilustrasi pada Tabel \ref{tab:f_3_orig1_4_1}, bernilai 0 yang dikalikan dengan $ G_{(behkn, 31,4)} $, yang merupakan jumlah kombinasi \textit{string} $ orig2 $ dengan operasi \textit{replace} yang memiliki jarak $ 1 $ terhadap \textit{string} $ ad2 $, yang berdasarkan ilustrasi pada Tabel \ref{tab:g_3_orig2_4_1} sampai dengan Tabel \ref{tab:g_3_orig2_4_3}, bernilai 10. Lalu nilai jawaban akhir akan ditambahkan dengan $ G_{(behkn, 31,1)} $, yang merupakan jumlah kombinasi \textit{string} $ orig1 $ dengan operasi \textit{replace} dengan jarak $ 4 $ terhadap \textit{string} $ ad1 $, yang berdasarkan ilustrasi pada Tabel \ref{tab:g_3_orig1_4_1} sampai dengan Tabel \ref{tab:g_3_orig1_4_5}, bernilai 0 yang dikalikan dengan $ F_{(behkn, 31,4)} $, yang merupakan jumlah kombinasi \textit{string} $ orig2 $ tanpa operasi \textit{replace} yang memiliki jarak $ 1 $ terhadap \textit{string} $ ad2 $, yang berdasarkan ilustrasi pada Tabel \ref{tab:f_3_orig2_4_1} sampai dengan Tabel \ref{tab:f_3_orig2_4_1}, bernilai 0. Sehingga nilai jawaban dari iterasi 4 adalah 0 dan nilai jawaban akhir hingga pada iterasi 4 adalah 8.

Berikutnya, pada iterasi $ dist = 5 $, nilai jawaban akhir akan ditambahkan dengan $ F_{(behkn, 31,0)} $, yang merupakan jumlah kombinasi \textit{string} $ orig1 $ tanpa operasi \textit{replace} dengan jarak $ 5 $ terhadap \textit{string} $ ad1 $, yang berdasarkan ilustrasi pada Tabel \ref{tab:f_3_orig1_5_1}, bernilai 0 yang dikalikan dengan $ G_{(behkn, 31,5)} $, yang merupakan jumlah kombinasi \textit{string} $ orig2 $ dengan operasi \textit{replace} yang memiliki jarak $ 0 $ terhadap \textit{string} $ ad2 $, yang berdasarkan ilustrasi pada Tabel \ref{tab:g_3_orig2_5_1} sampai dengan Tabel \ref{tab:g_3_orig2_5_3}, bernilai 0. Lalu nilai jawaban akhir akan ditambahkan dengan $ G_{(behkn, 31,0)} $, yang merupakan jumlah kombinasi \textit{string} $ orig1 $ dengan operasi \textit{replace} dengan jarak $ 5 $ terhadap \textit{string} $ ad1 $, yang berdasarkan ilustrasi pada Tabel \ref{tab:g_3_orig1_5_1} sampai dengan Tabel \ref{tab:g_3_orig1_5_5}, bernilai 8 yang dikalikan dengan $ F_{(behkn, 31,5)} $, yang merupakan jumlah kombinasi \textit{string} $ orig2 $ tanpa operasi \textit{replace} yang memiliki jarak $ 0 $ terhadap \textit{string} $ ad2 $, yang berdasarkan ilustrasi pada Tabel \ref{tab:f_3_orig2_5_1} sampai dengan Tabel \ref{tab:f_3_orig2_5_1}, bernilai 1. Sehingga nilai jawaban dari iterasi 5 adalah 8 dan nilai jawaban akhir untuk kasus di mana \textit{string} $ ad1 = kbenh $, \textit{string} $ ad2 = kbenh $ dan $ X=5 $ adalah 16.

\section{Analisa Kompleksitas Waktu}

Pada \textit{pseudocode} pada Gambar \ref{figure:fungsi_main}, terdapat fungsi preprocess. Kompleksitas waktu dari fungsi preprocess adalah $ \mathcal{O}(2^{|S|} * |S|) $ di mana $ S $ adalah \textit{string} masukan.

Berikutnya untuk setiap kasus uji terdapat empat fungsi utama. Dengan menggunakan \textit{master theorem}, fungsi readInput memiliki kompleksitas $ \mathcal{O}(1) $. Berdasarkan \textit{pseudocode} fungsi init() pada Gambar \ref{figure:pseudocode_fungsi_init}, fungsi init dapat dipecah menjadi dua bagian utama, yaitu inisialisasi $ memoF $ dan $ memoG $ dan inisialisasi $ charFirstPos $ dan $ charLastPos $. Inisialisasi $ memoF $ dan $ memoG $ masing-masing memiliki kompleksitas waktu $ \mathcal{O}(2 * 2^{|S|} * MAX\_DIST) $ di mana $ MAX\_DIST $ adalah jarak maksimum antar \textit{string} yang mungkin. Sedangkan inisialisasi $ charFirstPos $ dan $ charLastPos $ memiliki kompleksitas waktu $ \mathcal{O}(|S| log |S| + |S|) $ atau dapat disederhanakan menjadi $ \mathcal{O}(|S| log |S|) $. Sehingga fungsi ini memiliki kompleksitas $ \mathcal{O}(2 * 2^{|S|} * MAX\_DIST + |S| log |S|) $ dan dapat disederhanakan menjadi $ \mathcal{O}(2^{|S|} * MAX\_DIST) $.

Fungsi berikutnya adalah fungsi solve. Pada fungsi solve terdapat sebuah iterasi sebanyak $ min(MAX\_DIST, X) $ di mana pada kasus terburuk, $ min(MAX\_DIST, X) $ bisa mencapai $ MAX\_DIST $. Di dalam iterasi tersebut, fungsi solve memanggil fungsi F dan fungsi G masing-masing sebanyak dua kali yang memiliki kompleksitas waktu $ \mathcal{O}(2 * 2^{|S|} * MAX\_DIST) $ atau dapat disederhanakan menjadi $ \mathcal{O}(2^{|S|} * MAX\_DIST) $. Sehingga kompleksitas dari fungsi solve secara keseluruhan adalah $ \mathcal{O}(2^{|S|} * MAX\_DIST^{2}) $.

Terakhir adalah fungsi writeOutput. Kompleksitas waktu dari fungsi writeOutput adalah $ \mathcal{O}(1) $. Sehingga secara keseluruhan, kompleksitas waktu dari algoritma yang telah dirancang pada Tugas Akhir ini adalah $ \mathcal{O}(2^{|S|} * MAX\_DIST^{2} * T) $.

Pada umumnya, eksekusi program pada situs penilaian daring SPOJ adalah $ 1 $ detik untuk setiap $ 100.000.000 $ proses. Pada kasus terburuk, yaitu ketika $ |S|=10 $, $ MAX\_DIST=250 $ dan $ T=10 $, eksekusi program dengan kompleksitas waktu $ \mathcal{O}(2^{|S|} * MAX\_DIST^{2} * T) $ akan melakukan $ 640.000.000 $ proses dimana jika dengan menggunakan standar berupa $ 100.000.000 $ proses per detik, program akan membutuhkan waktu eksekusi sebesar $ 6,4 $ detik. Sehingga Algoritma dengan kompleksitas waktu $ \mathcal{O}(2^{|S|} * MAX\_DIST^{2} * T) $ cukup untuk menyelesaikan \problem{}.