\chapter{KESIMPULAN}

Pada bab ini dijelaskan mengenai kesimpulan dari hasil uji coba yang telah dilakukan.

\section{Kesimpulan}

Dari hasil uji coba yang telah dilakukan terhadap perancangan dan implementasi algoritma untuk menyelesaikan \problem{} dapat diambil kesimpulan sebagai berikut:

\begin{enumerate}
	\item Implementasi algoritma dengan menggunakan pendekatan \dynamicprogramming{} dan teknik \meetinthemiddle{} dapat menyelesaikan permasalahan \problem{} dengan benar.
	\item Kompleksitas waktu sebesar $ \mathcal{O}(2^{|S|} * MAX\_DIST^{2} * T) $ dapat menyelesaikan \problem{}.
	\item Waktu yang dibutuhkan program untuk menyelesaikan \problem{} minimum $ 0.98 $ detik, maksimum $ 1.02 $ detik dan rata-rata $ 1.004 $ detik. Memori yang dibutuhkan adalah sebesar 19 MB.
\end{enumerate}

\section{Saran}

Pada Tugas Akhir kali ini tentunya terdapat kekurangan serta nilai-nilai yang dapat penulis ambil. Berikut adalah saran-saran yang dapat diambil melalui Tugas Akhir ini:

\begin{enumerate}
	\item Paradigma \dynamicprogramming{} adalah pendekatan yang sesuai untuk menyelesaikan permasalahan yang memiliki submasalah yang bersifat tumpang tindih dengan submasalah lainnya.
	\item Teknik \meetinthemiddle{} adalah teknik yang sesuai untuk menyelesaikan permasalahan apabila permasalahan tersebut dapat dibagi menjadi dua atau lebih submasalah yang tidak memiliki ketergantungan satu sama lain.
\end{enumerate}