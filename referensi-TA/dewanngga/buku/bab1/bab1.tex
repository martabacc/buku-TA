\chapter{PENDAHULUAN}
Pada bab ini akan dijelaskan latar belakang, rumusan masalah, batasan masalah, tujuan, metodologi dan sistematika penulisan Tugas Akhir.

\section{Latar Belakang}

Tugas Akhir ini mengacu pada \problem. Diberikan dua buah \textit{string} $orig1$ dan $orig2$. Diberikan tiga tahapan proses enkripsi untuk menghasilkan \textit{string} $ad1$ dan $ad2$ sebagai berikut:

\begin{enumerate}
	\item \textit{String} $orig1$ diacak urutan karakter-karakternya.
	\item \textit{String} $orig2$ diacak urutan karakter-karakternya.
	\item Salah satu karakter dari \textit{string} $orig1$ atau $orig2$ diganti dengan karakter sebelum atau sesudahnya dalam alfabet.
\end{enumerate}	

Jarak dua buah \textit{string} didefinisikan sebagai jumlah dari selisih mutlak dari karakter-karakter pada posisi yang sama. Diberikan sebuah bilangan bulat $X$ yang merupakan jarak dari \textit{string} $orig1$ dan \textit{string} $ad1$ dijumlahkan dengan jarak dari \textit{string} $orig2$ dan \textit{string} $ad2$. Berapakah jumlah kemungkinan kombinasi \textit{string} $orig1$ dan $orig2$ jika diberikan \textit{string} $ad1$, $ad2$ dan nilai $X$.

Untuk menyelesaikan permasalahan di atas, penulis akan menggunakan pendekatan solusi dengan teknik \dynamicprogramming{} dan \meetinthemiddle{}. Selain dapat menjawab pertanyaan dengan benar, waktu juga menjadi salah satu faktor penting untuk memberikan gambaran tentang performa dari algoritma yang dirancang.

Hasil dari Tugas Akhir ini diharapkan dapat memberikan gambaran mengenai performa algoritma dengan teknik \dynamicprogramming{} dan \meetinthemiddle{}.	 

\section{Rumusan Masalah}
Rumusan masalah yang diangkat dalam Tugas Akhir ini adalah sebagai berikut:
\begin{enumerate}
	\item Perancangan algoritma yang sesuai untuk menyelesaikan \problem{} yang didasari oleh teknik \dynamicprogramming{} dan \meetinthemiddle{}.
	\item Implementasi algoritma untuk menyelesaikan \problem.
	\item Analisis performa algoritma yang telah dirancang untuk menyelesaikan \problem.
\end{enumerate}

\section{Batasan Masalah}
Permasalahan yang dibahas pada Tugas Akhir ini memiliki beberapa batasan, yaitu sebagai berikut:

\begin{enumerate}
	\item Implementasi algoritma menggunakan bahasa pemrograman C++.
	\item Uji coba kebenaran dilakukan dengan uji \textit{submission} ke situs penilaian daring SPOJ.
	\item Panjang \textit{string} masukan $ ad1 $ dan $ ad2 $ maksimal bernilai 10.
	\item Karakter pada \textit{string} masukan $ ad1 $ dan $ ad2 $ berada dalam rentang $ `b` \le ad1_{i}, ad2_{i} \le `y` $.
	\item Nilai masukan $ X $ tidak melebihi $ 100000 $.
	\item Batas waktu eksekusi program adalah $ 6,459 $ detik. 
\end{enumerate}

\section{Tujuan}
Tujuan dari Tugas Akhir ini adalah sebagai berikut:

\begin{enumerate}
	\item Menyelesaikan \problem{} dengan algoritma yang telah dirancang dan diimplementasikan menggunakan teknik \dynamicprogramming{} dan \meetinthemiddle{}.
	\item Menguji kebenaran algoritma dengan melakukan uji kebenaran terhadap algoritma yang telah dirancang dan diimplementasikan.
	\item Mengetahui performa algoritma yang dibangun dengan menganalisa hasil uji coba.
\end{enumerate}

\section{Manfaat}
Manfaat dari Tugas Akhir ini adalah sebagai berikut:

\begin{enumerate}
	\item Mengetahui pemanfaatan metode \dynamicprogramming{} dan \meetinthemiddle{} dalam menyelesaikan suatu permasalahan.
	\item Melatih kemampuan analisis karakteristik permasalahan yang dapat diselesaikan dengan metode \dynamicprogramming{} dan \meetinthemiddle{}.
\end{enumerate}

\section{Metodologi}
Metodologi yang digunakan dalam pengerjaan Tugas Akhir ini adalah sebagai berikut:
\begin{enumerate}
	
	\item Penyusunan proposal Tugas Akhir

	Pada tahap ini dilakukan penyusunan proposal Tugas Akhir yang berisi permasalahan dan gagasan solusi yang akan diteliti pada \problem.
	
	\item Studi literatur
	
	Pada tahap ini dilakukan pencarian informasi dan studi literatur mengenai pengetahuan atau metode yang dapat digunakan dalam penyelesaian masalah. Informasi didapatkan dari materi-materi yang berhubungan dengan algoritma yang digunakan untuk penyelesaian permasalahan ini, materi-materi tersebut didapatkan dari buku, jurnal, maupun internet.
	
	\item Desain
	
	Pada tahap ini dilakukan desain rancangan algoritma yang digunakan dalam solusi untuk pemecahan \problem.
	
	\item Implementasi perangkat lunak
	
	Pada tahap ini dilakukan implementasi atau realiasi dari rancangan desain algoritma yang telah dibangun pada tahap desain ke dalam bentuk program.
	
	\item Uji coba dan evaluasi
	
	Pada tahap ini dilakukan uji coba kebenaran implementasi. Pengujian kebenaran dilakukan pada sistem penilaian daring SPOJ sesuai dengan masalah yang dikerjakan untuk diuji apakah luaran dari program telah sesuai.
	
	\item Penyusunan buku Tugas Akhir
	
	Pada tahap ini dilakukan penyusunan buku Tugas Akhir yang berisi dokumentasi hasil pengerjaan Tugas Akhir.
\end{enumerate}

	\section{Sistematika Penulisan}
	Berikut adalah sistematika penulisan buku Tugas Akhir ini:
	\begin{enumerate}
		\item BABI: PENDAHULUAN
		
		Bab ini berisi latar belakang, rumusan masalah, batasan masalah, tujuan, metodologi dan sistematika penulisan Tugas Akhir.
		
		\item BAB II: DASAR TEORI
		
		Bab ini berisi dasar teori mengenai permasalahan dan algoritma penyelesaian yang digunakan dalam Tugas Akhir
		
		\item BAB III: DESAIN
		
		Bab ini berisi desain algoritma dan struktur data yang digunakan dalam penyelesaian permasalahan.
		
		\item BAB IV: IMPLEMENTASI
		
		Bab ini berisi implementasi berdasarkan desain algortima yang telah dilakukan pada tahap desain.
		
		\item BAB V: UJI COBA DAN EVALUASI
		
		Bab ini berisi uji coba dan evaluasi dari hasil implementasi yang telah dilakukan pada tahap implementasi.
		
		\item BAB VI: KESIMPULAN
		
		Bab ini berisi kesimpulan yang didapat dari hasil uji coba yang telah dilakukan.
	\end{enumerate}

\cleardoublepage
