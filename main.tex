\documentclass{ta-its}
\usepackage{hyperref}
\usepackage{cleveref}
\usepackage{multirow}
\usepackage{graphicx}
\usepackage{array}
\usepackage{multirow}
\usepackage{tabularx}
\usepackage{tabulary}
\usepackage{etoolbox}
\usepackage{listings}
\usepackage{longtable}
\usepackage{float}
\usepackage{slantsc}
\usepackage{booktabs}% http://ctan.org/pkg/booktabs


\newcolumntype{L}[1]{>{\raggedright\let\newline\\\arraybackslash\hspace{0pt}}m{#1}}
\newcolumntype{C}[1]{>{\centering\let\newline\\\arraybackslash\hspace{0pt}}m{#1}}
\newcolumntype{R}[1]{>{\raggedleft\let\newline\\\arraybackslash\hspace{0pt}}m{#1}}

\newcommand{\mychapter}[2]{
    \setcounter{chapter}{#1}
    \setcounter{section}{0}
    \chapter*{#2}
    \addcontentsline{toc}{chapter}{#2}
}


\title{Rancang Bangun Aplikasi \textit{web} Lelang \textit{Online} \textit{(E-Auction)} Berbasis Kerangka Kerja Laravel}{E-Auction Web Application Design and Implementation based on Laravel Framework}{KI141502} 

% \author{Nama Lengkap}{NRP}
\author{Ronauli Silva Natalensis Sidabukke}{5113100142}

% \supervisorOne{Nama Pembimbing Satu}{NIP}
% \supervisorTwo{Nama Pembimbing Dua}{NIP}
\supervisorOne{Rully Soelaiman, S.Kom, M.Kom}{197002131994021001}
\supervisorTwo{Rizky Januar Akbar, S.Kom., M.Eng}{198701032014041001}

% \degree{Nama Gelar}{Bidang Studi}{Program Studi}{Jurusan}{Jurusan (English)}{Fakultas}{Fakultas Singkatan}{Fakultas (English)}
\degree{Sarjana Komputer}{Algoritma Pemrograman}{S1}{Teknik Informatika}{Informatics}{Teknologi Informasi}{FTIf}{Information Technology}

% \time{bulan}{tahun}
\time{Juni}{2017}


\begin{document}
    \maketitle
    \pagenumbering{roman}
    \legalityPaper
    \begin{abstrak}
		E-commerce adalah kombinasi antara dunia digital dan transaksi lelang. Di Indonesia, seiring terjadi peningkatan jumlah pengguna internet dan menjamurnya bisnis online atau sering disebut \textit{online shop}. Salah satu jenis transaksi adalah lelang, yaitu metode jual beli yang mengintegrasikan mekanisme lelang dengan Internet.
	    \newline
	    \indent Dalam interaksi antara pelaku lelang online (penjual dan pembeli) pasti terjadi kegagalan/ketidakpuasan dalam transaksi lelang online.Berangkat dari paper "" yang membahas mengenai analisa kesalahan dan strategi lewat survey terhadap pengguna aplikasi lelang online di Taiwan, penulis membangun aplikasi lelang online yang disertai dengan tambahan fitur maupun saran dari paper tersebut.
	    \newline 
	    \indent Tidak hanya berdasarkan paper rujukan, penulis juga menganalisa aplikasi \textit{e-commerce} yang umum digunakan di Indonesia baik \textit{user experience} maupun alur transaksi, dan menambahkan beberapa fitur agar lebih sesuai dengan transaksi jual-beli online yang umum di Indonesia. Dengan aplikasi ini, diharapkan kegagalan dalam transaksi online dapat diperbaiki dan membuka peluang lelang online untuk meramaikan industri \textit{e-commerce} di Indonesia.\\
\noindent \textbf{Kata-Kunci}: \textit{lelang online}, \textit{strategi }
\end{abstrak}
    \begin{abstract}
	E-commerce industry is growing rapidly in Indonesia, along with the increasing number of internet users and number of online shops is also growing. One of e-commerce type is online auction, a buy and sell method that integrates auction mechanism and the Internet.\\
	\indent In the interaction between online auction actors (buyers and sellers), inevitable failure/dissatisfaction of online auction transactions sometimes found. Started by analysing paper about online auction application typologies and strategies through an application's users survey, author want to build online auction application along with additional ideas and suggestions from the paper.  \\
	\indent Author also analyzed and considering user experience, design and transaction flow local e-commerce platforms that are commonly used in Indonesia, in purpose to make the application suits Indonesian's users better. Furthermore, author hopes that this applications can reduce/prevent the expected failures in online transactions and open up online auction opportunity to enliven the e-commerce industry in Indonesia.\\
\noindent \textbf{Keyword}: \textit{online auction}, \textit{typologies and strategies}
\end{abstract}
    \mychapter{0}{KATA PENGANTAR}
%   \begin{figure}[h]
%     \centering
%     \includegraphics[width=0.5\linewidth]{images/bab0/gambarBismillah}
%   \end{figure}
  Puji Syukur kepada Tuhan yang Maha Esa, atas berkatNya penulis dapat menyelesaikan buku berjudul \textbf{\judul}. Dalam pengerjaan Tugas Akhir ini, penulis belajar banyak untuk memperdalam dan meningkatkan apa yang telah dipelajari penulis selama kuliah di Teknik Informatika ITS.
  Tugas Akhir ini terselesaikan tidak lepas dari bantuan dan dukungan banyak pihak. Oleh karena itu, pada kesempatan ini penulis mengucapkan banyak terima kasih kepada:
  \begin{enumerate}
  	\item \textbf{Daddy Jesus} - atas segala berkat, karunia, kesempatan dan rancangan jalanNya-lah penulis masih diberi nafas kehidupan, tenaga dan daya pikir untuk menyelesaikan buku ini. \textit{Thank you, Big Daddy.}
    \item \textbf{Papa dan Mama} yang selalu menguatkan, menasehati, dan luar biasa sabar dalam mengingatkan penulis agar tidak lupa menjaga kesehatan dan tidak lupa ke gereja selama masa studi.
    \item \textbf{Yth. Bapak Rully Soelaiman} yang memberi inspirasi kepada penulis untuk berpikir \textit{scientifically}, bimbingan, nasehat, saran dan memberikan penulis sisi pemikiran dan perspektif baru terhadap setiap masalah.
    \item \textbf{Yth. Bapak Rizky Januar Akbar} sebagai dosen pembimbing yang memberi bimbingan, saran teknis dan administratif, diskusi dan pemecahan masalah dalam pembuatan dan penulisan buku tugas akhir.
    \item \textbf{Keluarga XL Future Leader Scholarship Camp Batch 5} dan KSE ITS yang telah memberikan penulis kesadaran, semangat dan inspirasi untuk terus melanjutkan tugas akhir di saat penulis kehilangan semangat.
    \item \textbf{Keluarga Admin Lab. Pemrograman }(2014 - 2017) , yang telah memberikan penulis banyak pengalaman, pengetahuan dan cerita-cerita untuk dikenang.
    \item \textbf{Keluarga Alumni Budi Mulia Siantar-Surabaya angkatan 2013 } , teman setia disaat suka maupun duka.
    \item  \textbf{Keluarga Pengpro \textit{Furions} dan HMTC Optimasi 2016 }, yang mengajarkan penulis tentang cara organisasi, cara berbicara di depan publik, dan banyak lagi.    
    \item Serta semua pihak yang tidak tertulis - yang telah turut membantu penulis dalam menyelesaikan Tugas Akhir ini.
  \end{enumerate}
  Penulis menyadari bahwa Tugas Akhir ini masih memiliki banyak kekurangan. Oleh karena itu, penulis berharap kritik dan saran dari pembaca sekalian untuk memperbaiki buku ini ke depannya.


  \hfill Surabaya, Juni 2017 \\ \\ 


  \hfill Ronauli Silva N. Sidabukke

\cleardoublepage % Mengisi penanda halaman genap yang kosong


    \tableofcontents % Daftar isi
    \listoftables % Daftar tabel
    \listoffigures % Daftar gambar

  \mainmatter % Halaman utama, dengan judul BAB X
    \chapter{PENDAHULUAN}
  Pada bab ini akan dipaparkan mengenai garis besar Tugas Akhir yang meliputi latar belakang, tujuan, rumusan dan batasan permasalahan, metodologi pembuatan Tugas Akhir, dan sistematika penulisan.
  \section{Latar Belakang}
  	
	\indent Transaksi jual beli saat ini sudah dapat dilakukan lewat berbagai cara, antara lain menggunakan \textit{e-commerce}, atau lewat \textit{social media}, atau bisa dengan melelang di aplikasi lelang \textit{online}. Sedikit berbeda dengan teknik penjualan di lelang online, karena aplikasi ini dapat diakses oleh banyak orang, tentu saja pelelang (\textit{auctioneer}) tidak terbatas pada ruang lelang saja, tapi bisa berasal dari manapun selama mereka mengakses aplikasi tersebut.  Lelang \textit{online} ini tentu saja mendatangkan banyak manfaat, selain biaya yang lebih efisien dan hemat, dan juga tidak menguras waktu karena siapapun, kapanpun, dimanapun dapat mengajukan penawaran ataupun melelang barangnya tanpa harus pergi ke instansi tertentu dan melakukan lelang dengan cara konvensional.
    \\
    \indent Aplikasi serupa telah banyak, namun banyak aspek yang kurang dalam aplikasi tersebut, seperti informasi dari lelang tidak \textit{reliable} (misal: stok barang ternyata sudah habis), alur proses yang tidak jelas sehingga membingungkan pengguna aplikasi, informasi yang kurang jelas, dan produk yang didapatkan ternyata tidak sesuai dengan informasi pada saat produk dilelang (\textit{bad information}) \cite{ying-feng_kuo_online_2016}.
    \\
    \indent Dan dari masalah teknis aplikasi, beberapa sumber menyatakan bahwa ketidakjelasan alur proses yang kurang diperhatikan oleh para developer aplikasi lelang \textit{online} menjadi beberapa alasan yang kuat mengapa lelang online masih kurang diminati \cite{noauthor_sistem_nodate}.
    \\
	\indent Diharapkan, dengan adanya aplikasi ini, beberapa kelemahan yang masih ada pada aplikasi lelang \textit{online} saat ini dapat diperbaiki, dan juga dapat dapat membantu proses \textit{online} yang ada di Indonesia, dan juga mampu memperbaiki citra aplikasi lelang \textit{online} sehingga mampu meningkatkan minat masyarakat terhadap lelang \textit{online}.
    
  \section{Rumusan Masalah}
    Rumusan masalah yang diangkat dalam tugas akhir ini adalah sebagai berikut: 
    \begin{enumerate}
      \item Bagaimana membangun aplikasi lelang online berbasis web?
      \item Bagaimana rancangan arsitektur aplikasi dan fitur yang menganalisa kelemahan aplikasi serupa dan strategi penyelesaian sesuai dengan paper acuan \cite{ying-feng_kuo_online_2016}?
    \end{enumerate}

  \section{Batasan Masalah}
  	\label{batasan-masalah}
    Dari permasalahan yang telah diuraikan di atas, terdapat beberapa batasan masalah pada tugas akhir ini, yaitu:
    \begin{enumerate}
      \item Aplikasi berbasis web dengan bahasa pemrograman PHP.
      \item Aplikasi berbasis kerangka kerja Laravel.
      \item Basis data yang digunakan adalah PostgreSQL.
      \item Aplikasi tidak mencakup proses pembayaran.
    \end{enumerate}

  \section{Tujuan}
  \label{tujuan}
    Tujuan dari pengerjaan Tugas Akhir ini adalah: 
    \begin{enumerate}
      \item Membangun aplikasi lelang online berbasis web yang lebih kredibel sesuai dengan paper yang dijadikan acuan pada tugas akhir ini. 
    \end{enumerate}
    \chapter{LANDASAN TEORI}{}
  \section{Lelang Daring / Lelang \textit{Online}}
   Lelang adalah proses membeli dan menjual barang atau jasa dengan cara menawarkan kepada penawar, menawarkan tawaran harga lebih tinggi, dan kemudian menjual barang kepada penawar harga tertinggi. Dalam teori ekonomi, lelang mengacu pada beberapa mekanisme atau peraturan perdagangan dari pasar modal. \\
	Sementara lelang daring atau lelang melalui internet muncul seiring dengan perkembangan internet. Barang atau jasa yang diperjualbelikan dipasang di situs dan peserta lelang dapat mengikuti acara lelang secara daring. Perusahaan lelang yang berhasil menggunakan sarana internet salah satunya adalah \textit{Ebay} . Di Indonesia, lelang melalui internet (online) sudah dipelopori oleh pemerintah dengan situs lelang online yang dapat diakses melalui website resmi \href{https://www.lelangdjkn.kemenkeu.go.id}{Kemenkeu} \cite{wikipedia_lelang_2016} . 
    Berikut adalah beberapa istilah yang ada dalam lelang online :
    \begin{enumerate}
	\item BID atau \textit{Bidding}, artinya : Menawarkan
    \item BIN (\textit{Buy In Now}) artinya : Beli sesuai harga yang telah ditawarkan penjual
    \item INC (\textit{Increment}) artinya : Minimum kenaikan \textit{bid} setelah \textit{bid} sebelum nya \cite{noauthor_arti_nodate}
    \end{enumerate}
    
    \section{PostgreSQL}
    PostgreSQL adalah sebuah produk \textit{database} relasional yang termasuk dalam kategori \textit{free open source software} (\textit{FOSS}). 
	PostgreSQL terkenal karena fitur-fitur yang advanced dan pendekatan rancangan modelnya menggunakan paradigma \textit{object-oriented}, sehingga sering dikategorikan sebagai \textit{Object Relational Database Management System} (ORDBMS).
    Beberapa fitur PostgreSQL adalah sebagai berikut :
    \begin{enumerate}
    \item \textit{Inheritance}, dimana satu table dapat diturunkan model dan beberapa karakteristik dari table lainnya.
    \item \textit{Multi-Version Concurrency Control} dimana user diberi data snapshot ketika suatu perubahan dilakukan sampai commit.
    \item \textit{Rules} , dimana suatu \textit{query} DML yang dikirimkan ke server akan mengalami penulisan ulang (\textit{rewrite}). Ini terjadi sebelum diproses oleh \textit{query planner}.
    \item dan berbagai fitur lainnya \cite{noauthor_postgresql_nodate}
    \end{enumerate}
    
  \section{Redis}
    Redis adalah \textit{open source}, struktur data yang ditempatkan di memori, digunakan sebagai \textit{database}, \textit{cache} dan \textit{message broker}. Redis mendukung struktur data seperti \textit{string, sets, hash, lists} dan \textit{sorted sets}. Sama seperti cache, setiap key diisi oleh value. Tapi kelebihannya, Redis bisa digunakan untuk melakukan operasi dari value tersebut. Cara terbaik untuk memahami redis adalah membuat model aplikasi tanpa memikirkan bagaimana caranya untuk menyimpan data di dalam \textit{database} \cite{yudana_redis_2015}.

  \section{Node.js}
  Node.js adalah platform perangkat lunak pada sisi-server dan aplikasi jaringan. Ditulis dengan bahasa javascript dan bisa dijalankan pada Windows, Mac OS X dan Linux tanpa perubahan kode program. Node.js memiliki pustaka server HTTP sendiri sehingga memungkinkan untuk menjalankan webserver tanpa menggunakan program webserver seperti Apache atau Lighttpd \cite{noauthor_node.js_2014}.
  
  \section{Socket.io}
	Socket.io adalah \textit{library} Javascript untuk aplikasi web yang bersifat \textit{realtime}. Socket.io menjembatani antara komunikasi dua arah antara \textit{web} \textit{clients} dan \textit{server}. Socket.io terbagi menjadi dua bagian, yaitu \textit{client}-\textit{side} \textit{library} yang berjalan di browser client, dan \textit{server}-\textit{side} \textit{library} yang menggunakan Node.js. Kedua komponen tersebut mempunyai API yang sama. Seperti Node.js, Socket.io juga bersifat \textit{event}-\textit{driven}. Socket.IO menggunakan protokol \textit{websocket} dengan \textit{polling} sebagai opsi \textit{fallback}. Meskipun Socket.IO merupakan ‘pembungkus’ untuk soket web, namun ia memiliki banyak fitur, antara lain broadcast ke banyak soket, dan I/O yang asinkronus \cite{noauthor_socket.io_2016}.
    
    \section{Laravel}
    Laravel adalah \textit{framework} PHP yang dikembangkan pertama kali oleh Taylor Otwell. Walaupun termasuk baru, namun komunitas pengguna laravel sudah berkembang pesat dan mampu menjadi alternatif utama dari sejumlah \textit{framework} besar seperti CodeIgniter dan Yii. Laravel oleh para \textit{developer} disetarakan dengan CodeIgniter dan FuelPHP namun memiliki keunikan tersendiri dari sisi \textit{coding}. Laravel memiliki beberapa keunggulan, diantaranya :
\begin{enumerate}
\item Sintaks yang sederhana dan \textit{programmer}-\textit{fiendly}
\item Tersedia \textit{generator} yang canggih dan memudahkan, Artisan CLI
\item Fitur \textit{Schema} \textit{Builder} untuk berbagai \textit{database}
\item Fitur \textit{Migration} dan \textit{Seeding} untuk berbagai \textit{database}
\item Fitur \textit{Query} \textit{Builder} yang powerful
\item Eloquent ORM (\textit{Object} \textit{Relational} \textit{Mapping})
\item Fitur pembuatan \textit{package} dan \textit{bundle}
\item \textit{Dependency} \textit{Injection} \cite{a}
\end{enumerate}

	\section{Protokol SMTP}
    
	\section{JSON Web Token}
    
   \section{Service Worker}
   
   \section{Repository Pattern}
   
   \section{Concurrency}
   
   \section{Transaction Isolation}
   
   \section{Script Testing}
   
   \section{Laravel Dusk}
   
   


    \chapter{ANALISA DAN PERANCANGAN}  
  \section{Analisa}
  
    \subsection{Analisa Paper Rujukan}
    Dengan perkembangan teknologi, perlahan kebiasaan manusia berubah. Termasuk juga dalam perdagangan, dimana transaksi jual beli barang tidak lagi harus melalui tatap muka. Penjualan online saat ini sudah dapat dilakukan lewat berbagai cara, antara lain menggunakan e-commerce, atau posting di social media, atau bisa juga dengan melelang di aplikasi lelang online. Sedikit berbeda dengan teknik penjualan di lelang online, karena aplikasi ini dapat diakses oleh banyak orang, tentu saja pelelang (auctioneer) tidak terbatas pada ruang lelang saja, tapi bisa berasal dari manapun selama mereka mengakses aplikasi tersebut.  Lelang online ini tentu saja mendatangkan banyak manfaat, selain biaya yang lebih efisien dan hemat, dan juga tidak menguras waktu karena siapapun, kapanpun, dimanapun dapat mengajukan penawaran ataupun melelang barangnya tanpa harus pergi ke instansi tertentu dan melakukan lelang dengan cara konvensional.
    Bercermin terhadap aplikasi \textit{e-commerce} yang telah ada, masalah yang paling sering dialami adalah ketidakpuasan pengguna. Salah satu indikator bahwa suatu perusahaan dikatakan memiliki ketidakpuasan pelanggan adalah karena kegagalan dalam pelayanannya. Seorang pelanggan sangat mungkin memutuskan untuk komplain setelah mengalami ketidakpuasan terhadap layanan suatu perusahaan, dan jika tidak ditangani dengan baik, hal ini bisa berakibat fatal terhadap reputasi dan kepercayaan pengguna terhadap aplikasi tersebut.
    Oleh karena itu, sebuah paper mengangkat topik ini khusus dalam bidang aplikasi lelang online, menganalisa kegagalan dan ketidakpuasan pengguna, beserta solusi-solusi yang ditawarkan oleh pengguna aplikasi untuk memperbaiki kegagalan pelayanan tersebut. 
	  \begin{figure}[H]
        \centering
        \includegraphics[width=\linewidth]{images/bab3/Fatalitas-Kegagalan-Ecommerce.png}
        \caption{Fatalitas kegagalan dalam aplikasi Lelang Online, Kepuasan terhadap Perbaikan Pelayanan dan \textit{Repeat Purchase Intention} setelah Perbaikan Layanan}
        \label{severity-failures}
      \end{figure}
      
      Dalam gambar diatas, dijabarkan beberapa jenis kegagalan yang pernah dialami oleh pengguna aplikasi serta fatalitas/pengaruh buruk kegagalan tersebut terhadap kepercayaan pengguna. 
      
	  \begin{figure}[H]
        \centering
        \includegraphics[width=\linewidth]{images/bab3/Solusi-Perbaikan-Ketidakpuasan.png}
        \caption{Kategori Perbaikan terhadap Kegagalan Pelayanan Lelang Online}
        \label{service-recovery-strategies}
      \end{figure}
      
      Dalam gambar diatas, dijabarkan beberapa solusi yang pernah ditawarkan kepada pengguna aplikasi serta kepuasan dan ketidakpuasan terhada solusi yang diberikan
      
      
`'      Maka berdasarkan hasil analisa tersebut, fitur-fitur yang perlu ditambahkan selain daripada fitur dasar aplikasi lelang online adalah sebagai berikut :
      \begin{enumerate}
      \item Fitur chatting, untuk mengurangi kemungkinan \textit{Bad Information} dimana ekspektasi dan persepsi terhadap barang yang dilelang antara pembeli dan penjual tidak sama dan \textit{Special Needs}, 
      \item Fitur pemberian kupon voucher (\textit{Discount and Correction Plus}) yang bisa berupa \textit{free shipping} atau \textit{discount}.
      \end{enumerate}
  
 
	\subsection{Analisa \textit{User Experience} dari E-Commerce di Indonesia}
    \label{alasan-ux-ecommerce-indonesia}
    Pada saat awal pengerjaan dan pada masa pengerjaan juga, penulis sering sekali menganalisa dan memperhatikan kebiasaan-kebiasaan yang diciptakan saat sedang \textit{browsing} di website \textit{e-commerce} di Indonesia. Salah satu yang paling sering adalah Tokopedia.
    Beberapa hal yang saya perhatikan adalah :
    \begin{enumerate}
    \item Halaman yang muncul bukanlah eagerloading, tapi \textit{lazy loading} - dimana perlahan-lahan komponen muncul satu per satu, hingga semua komponen lengkap ditampilkan.
    \linebreak
    Ini adalah solusi cerdas untuk mengakali \textit{delay loading item} yang sudah pasti jumlahnya sangat banyak (maka butuh \textit{query} yang tentunya memakan waktu cukup lama), namun juga memainkan faktor psikologi / \textit{user behaviour} pengguna dengan membiarkan pengguna melihat tahap demi tahap halaman 'diisi'.
    \linebreak
    Saya pun mencoba menerapkan ini dalam aplikasi Lelang Online ini dengan menggunakan \textit{tools} Vue.js
    \item Layouting yang sederhana namun ringkas - adalah konsep dengan warna-warna yang juga sederhana, tidak terlalu kontras dan tidak terlalu halus untuk dilihat.
    \end{enumerate}
    Dari 2 poin tersebut, sebisa mungkin saya adaptasi ke dalam aplikasi Lelang Online ini.
    
    \subsection{Analisa Keamanan pada koneksi Soket}
    \label{alasan-socket.io}
    Untuk mengakomodasi fitur yang bersifat \textit{realtime}, dibutuhkan koneksi ke soket secara terus menerus. Hal ini tentu dapat menjadi sasaran empuk \textit{security} karena jika tidak diamankan, maka dapat menjadi peluang besar bagi para pihak yang tidak berkepentingan untuk merusak proses bisnis aplikasi.
    Namun, jika dalam setiap koneksi soket harus mengirimkan \textit{credentials}, hal ini tentu menjadi tidak praktis dan malah lebih berbahaya karena membiarkan data-data sensitif seperti \textit{password} dan \textit{username} berlalu-lalang di jaringan internet. Selain itu, \textit{disadvantage}nya adalah ketidakpraktisan untuk selalu meng\textit{query} database setiap kali ada koneksi, tentu saja ini memperlambat kerja \textit{database} dan menambah waktu \textit{delay}.
    Untuk menyiasatinya, penulis menerapkan JWT.io dengan keuntungan sebagai berikut :
    \begin{enumerate}
    \item Tidak perlu \textit{query} ke database karena hanya menggunakan security token yang di\textit{generate} dengan formula tertentu
    \item Tidak perlu ada aplikasi khusus yang menjembatani aplikasi Soket dan aplikasi WebServer - sehingga lebih praktis
    \item Data-data sensitif menjadi lebih terjaga karena tidak perlu dipertukarkan setiap koneksi ke soket.
    \end{enumerate}
    
    \subsection{Analisa \textit{Best Practice} dalam Struktur Perangkat Lunak}
    \label{alasan-best-practice}
    Pada dasarnya, Laravel adalah kerangka kerja MVC. Namun, ada banyak fitur yang ada dalam aplikasi Lelang Online ini yang tidak terakomodasi dalam MVC, misal sebagai berikut :
    \begin{enumerate}
    \item Sistem Verifikasi lewat Email - yang berarti aplikasi harus berinteraksi dengan SMTP server
    \item Sistem \textit{Generate} Token JWT.io , dimana dalam proses \textit{Generate Token} sama sekali tidak ada database dilibatkan.
    \end{enumerate}
    
    Jika fitur-fitur tersebut 'dipaksa' dimuat ke dalam MVC, maka tentu saja strukturnya menjadi ganjil, dan muncul \textit{code smell} berikut :
    \begin{enumerate}
    \item \textit{Large Class}, dimana terdapat satu buah file yang sangat panjang (biasanya merupakan entitas utama, dalam hal ini contohnya barang/\textit{item})
    \item \textit{Inappropriate Intimacy}, dimana terdapat satu kelas yang menyimpan \textit{logic} yang tidak seharusnya ia simpan
    \item \textit{Duplicated Code}
    \end{enumerate}
    
    Untuk menghindari kemungkinan \textit{code smell} tersebut, maka penulis menyiasatinya dengan cara berikut :
    \begin{enumerate}
    \item Penggunaan Repository Pattern untuk memisahkan antara Data Processing Layer dan View Layer.
    Selain lebih rapi, terstruktur, hal ini juga dapat menghindari \textit{Duplicated Code}.
    \item Penambahan komponen baru yaitu Service dan Provider, untuk memisahkan dan merapikan struktur aplikasi.
    Tujuannya, agar jika kedepannya terdapat perbaikan fitur/penambahan fitur, lebih mudah \textit{traceback} terhadap file/kelas yang bertanggungjawab terhadap fitur tersebut.
	\end{enumerate}
    
    \subsection{Analisa Aplikasi Serupa}
    \label{alasan-app-serupa}
    Selama penulisan dan pembuatan aplikasi, penulis selalu mencoba menganalisa aplikasi serupa. Dan pada akhirnya, penulis menemukan aplikasi yang kurang lebih alur bisnis / alur penggunaan aplikasinya serupa yaitu : Carousell.
	Penulis melihat ada beberapa kesamaan antara sifat transaksi aplikasi tugas akhir saya dengan aplikasi tersebut, yaitu :
	\begin{enumerate}
	\item Sama-sama tidak mengakomodasi pembayaran
    \item Sama-sama tidak adanya kepastian harga (bedanya, pada carousell.com yang terjadi adalah bargaining
	\end{enumerate}
	Jadi, untuk alur proses nya, banyak saya adaptasi dari Carousell, dengan maksud agar pengguna lebih familiar dan paham jika mengikuti \textit{base practice} di E-commerce lainnya yang lebih umum digunakan ole pengguna
  
  \pagebreak
  
\section{Perancangan Sistem}

	Pada bagian ini, penulis akan menjelaskan tahap demi tahap untuk menghubungkan analisa dan spesifikasi yang telah dipaparkan pada dua sub-bab selanjutnya, dan direalisasikan ke dalam bentuk desain pada sub-bab ketiga ini.

  
\subsection{Perancangan \textit{Data Sources dan Data Storage}}

	Untuk penyimpanan data, terdapat 2 jenis data yang sifatnya cukup berbeda, yaitu sebagai berikut :
    \\
    
    \begin{enumerate}
    \item \textbf{Data Transaksional disimpan di DBMS SQL - \textit{Relational}}
    \newline
    Data yang sifatnya \textit{transaksional} , seperti data \textit{bidding}, data pengguna, dan lain sebagainya.
    Untuk data ini, lebih baik jika menggunakan database Postgre , untuk menjaga integritas data dan \textit{integrity checking} juga  menjadi lebih mudah.
    \newline
    
    \item 
    \textbf{Data Non-Transaksional disimpan di DBMS NoSQL}
    \newline
    Seperti data \textit{chatting}, data \textit{joined rooms} tidak cocok dimasukkan kedalam database transaksional karena sifat pertambahan datanya yang sangat cepat dan urgensi integritas data tidak terlalu diprioritaskan (dibanding dengan data transaksional pada poin sebelumnya.
    \newline
    Oleh karena itu, baiknya data ini disimpan pada database NoSQL - pada rancang bangun aplikasi ini, DBMS NoSQL yang digunakan adalah MongoDB.
    \newline
    
    
    \item \textbf{Data Citra/Gambar disimpan di \textsc{Amazon Web Services}} \newline
    Gambar-gambar barang yang didaftarkan untuk dilelang, disimpan di \textit{cloud} dengan menggunakan \textit{Amazon Web Services}. Alasan-alasan menggunakan AWS sebagai data storage untuk gambar adalah sebagai berikut :
        \begin{enumerate}[noitemsep,topsep=0pt]
        \item Skalabilitas aplikasi lebih terjaga. 
        \newline Dengan memisahkan penyimpanan antara gambar dan server sehingga lebih mudah me\textit{maintain} perkembangan aplikasi, dan lebih fokus terhadap pengembangan aplikasi.
        \item Menyediakan \textit{built-in} keamanan, fleksibel dan efisiensi \cite{wikipedia_amazon_2016}
        \item Mencoba belajar menggunakan Amazon Web Services
        \end{enumerate}
        
    \item \textbf{Data \textit{Assets} Website}
    \newline
    Untuk data \textit{assets} yang dibutuhkan untuk website, terdapat beberapa kriteria penyimpanan berikut
    
      \begin{itemize}[noitemsep,topsep=0pt]
      \item Jika file tersebut sudah umum digunakan dan terdapat file CDNnya, maka akan akses CDNnya
      \newline
      Hal ini dimaksudkan agar \textit{loading} lebih cepat, sesuai dengan yang tercantum pada sumber \cite{sitepoint_7_2011}
      \item Jika file tersebut merupakan \textit{custom asset} , \textit{asset} yang dikustomisasi khusus untuk aplikasi ini, maka asset tersebut akan disimpan dalam server.
      \end{itemize}
    
    \end{enumerate}
%   doneded

  \subsection{Perancangan Skema \textit{Database}}


%   CDM dan PDM, belom kelar

  \subsection{Perancangan Arsitektur Aplikasi}
	
      \begin{figure}[H]
        \centering
        \includegraphics[width=\textwidth]{images/bab3/diagram/arsitektur-awal.png}
        \caption{Arsitektur Aplikasi Lelang Online 
        		\\
                \textit{External Services} artinya adalah menggunakan \textit{service} dari luar, tidak dibangun sendiri. }
        \label{arsitektur-app-final}
      \end{figure}
    
    \subsubsection{\textbf{\textsc{Nginx} sebagai WebServer dan Proxy Server}}
    {\scshape Nginx} adalah web server multifungsi - dimana selain berfungsi sebagai Webserver, namun juga bisa berfungsi sebagai Load Balancer. Dalam awal pembuatan aplikasi, \textsc{Nginx} hanya digunakan sebagai \textit{web server} untuk melayani permintaan halaman Web dari pengguna.
    \\
    Namun, saat \textit{deployment}, banyak sekali terjadi \textit{issue} yang berkaitan dengan \textit{ssl certificate}, sehingga pada akhirnya Nginx juga digunakan sebagai \textit{proxy server} - dimana Nginx mempunyai fungsi baru yaitu \textit{redirecting request=request} yang masuk ke dalam server, dan meneruskannya ke proses dalam server yang bertugas memproses \textit{request} tersebut.
	   \\
    Beberapa masalah yang ditemukan penulis, jika tidak menggunakan fitur \textsc{Nginx} sebagai proxy server untuk aplikasi Soket yang berbeda port adalah sebagai berikut :
	    \begin{itemize}[noitemsep,topsep=0pt]
	    \item CORS (Cross Origin Reference Source)
	    \newline
	    Dimana pada saat browser mengakses soket dari port lain (meskipun domainnnya sama), browser menganggap bahwa sambungan dari port lain tersebut sebagai \textit{security threat} dan otomatis memutuskan sambungan.
	    \item ERROR :: INSECURE RESPONSE!
	    \newline
	    Hal ini terjadi saat browser membuka sebuah web dengan https - namun mengakses koneksi soket yang tidak terproteksi dengan https. Hal ini juga membuat browser menganggap ini sebagai \textit{security threat}, dan tidak membuka \textit{reply} dari koneksi soket yang tidak terproteksi dengan https tersebut.
	    \end{itemize}
    
    \subsubsection {\textbf{Laravel dan Logika Aplikasi}}
    \textsc{Laravel}, bertugas sebagai Bos Besar, pengelola data dan manajemen data, dan pelayan \textit{request} dalam aplikasi Lelang online ini. Semua request diteruskan, dan diproses oleh \textsc{Laravel}, dan diproses oleh Laravel. 	    
    
    \subsubsection{\textbf{Vue.js sebagai \textit{View Renderer} }}
	Penggunaan Vue.js yang digunakan oleh penulis dimaksudkan untuk membagi beban kerja/\textit{workloads} antara Server dan Pengguna. \\
    Seperti yang saya paparkan pada subbab Analisa (poin \ref{alasan-ux-ecommerce-indonesia} dan \ref{alasan-app-serupa}), ini ditujukan sebagai solusi cerdas untuk mengakali \textit{delay querying} yaitu \textit{sharing workloads} antara server dan client(browser) dan juga \textit{user experience behaviour}, agar lebih sabar menunggu waktu \textit{loading} aplikasi). \\
    Namun, untuk optimasinya, mengingat laju pertambahan data gambar maupun barang pada aplikasi \textit{e-commerce} pastiya sangat cepat dan masif, maka penulis membagi \textit{workloads} antara Laravel sebagai \textit{web server}, dan browser pengguna - dengan menggunakan Vue.js.
    
    
    \subsubsection{\textbf{PostgreSQL sebagai DBMS Transaksional}}
    PostgreSQL bertugas menyimpan data-data yang bersifat transaksional, seperti data \textit{master} : data pengguna, data barang yang terdaftar, data riwayat lelang, data \textit{rating} dan \textit{review}, dan lain-lain.
    
    \subsubsection{\textbf{MongoDB} sebagai DBMS Non-Transaksional - NOSQL}
    MongoDB akan digunakan untuk menyimpan :
	    \begin{itemize}[noitemsep,topsep=0pt]
	    \item Daftar pesan/\textit{chat} yang dikirimkan pengguna
	    \item Daftar \textit{conversation} untuk mempermudah menampilkan \textit{inbox} pengguna
	    \item Daftar gambar/foto yang diunggah bersama dengan barang yang diupload.
	    \end{itemize}
	Ekspektasi dalam menggunakan database NoSQL adalah agar proses \textit{query} lebih cepat, tidak memberatkan database transaksional.

	\subsubsection{\textbf{Node.js} sebagai Asynchronous-Request Server}
    Server yang dibangun dengan menggunakan Node.js akan mengakomodasi \textit{request} yang bersifat \textit{event-driven} dan bersifat asinkronus, seperti transaksi lelang/\textit{bidding} dan \textit{chatting}.
    
    \subsubsection{\textbf{SendGrid} sebagai SMTP Service}
    Untuk mengakomodasi fitur verifikasi otomatis lewat email, dibutuhkan sebuah SMTP service untuk mengirimkan email dari aplikasi ke alamat email pengguna. Dalam hal ini, yang digunakan adalah SendGrid Service.
    
    \subsubsection{\textbf{Amazon S3} sebagai Data Storage Service}
    Untuk menyimpan gambar-gambar dari barang yang di\textit{upload} pada saat mendaftarkan barang.

	\subsubsection{\textbf{Laravel Dusk}}
	Untuk ini, akan dibahas lebih lanjut pada bagian pengujian, karena ini adalah bagian dari perangkat lunak yang digunakan sebagai \textit{testing} dalam arsitektur tersebut.
      
   \pagebreak
   
      
    
  
  
  \input{Chapters/Details/bab3/3cx/3cx-KamusData}
  
  
    w\chapter{IMPLEMENTASI}
  Pada bab ini dibahas mengenai implementasi aplikasi sesuai dengan perancangan sistem yang telah dijelaskan sebelumnya. Bahasa pemrograman yang digunakan antara lain PHP, SQL, Javascript.
  
  \section{Lingkungan Implementasi}
  Lingkungan pembangunan dijelaskan pada subbab ini.
  
  
  \subsection{Lingkungan Pembangun Perangkat Keras}
  
  Aplikasi dideploy secara \textit{online}, dalam sebuah \textit{Virtual Private Server} yang di\textit{host} oleh \textit{Digital Ocean}.
  Spesifikasi VPS yang digunakan adalah sebagai berikut :
  
  \begin{enumerate}
  	\item Hardware
		  \begin{enumerate}
		  	\item CPU: Intel(R) Xeon(R) CPU E5-2630L v2 @ 2.40GHz
		  	\item Operating System : 
		  	\item RAM : 512MB
		  	\item Storage Space : 20GB
		  \end{enumerate}
		  
   \item Operating System
	   \begin{enumerate}
	   	\item Architecture : 64bit
	   	\item Kernel Version : Linux 4.4.0-75-generic x86 64
	   	\item OS Version : Ubuntu 16.04.2 LTS Xenial
	   	\end{enumerate}
	   	
	\item Networking Stats
		\begin{enumerate}
			\item Tersambung ke Internet : Ya
			\item IP Publik : Ya
			\item Alamat IP Publik (IPv4) : 188.166.179.2
			\item \textit{Average Download Speed} : 1371 Mbit/s
			\item \textit{Average Upload Speed} : 860.12 Mbit/s
			\item DNS : Google
		\end{enumerate}
		
	\item Domain Stats
		\begin{enumerate}
			\item HTTPS Support : Yes
			\item SSL Certificate issued by : Avast
			\item Domain : https://Lelangapa.com
			\item Testing-purpose subdomain : https://testing.lelangapa.com
			\item Domain issued by : Namecheap
		\end{enumerate}
	
  \end{enumerate}
  
  \subsection{Lingkungan Pembangun Perangkat Lunak}
  Spesifikasi perangkat lunak yang digunakan untuk membuat tugas akhir ini adalah sebagai berikut:
	  \begin{enumerate}
	  \item Google Chrome sebagai media akses aplikasi
	  \item PgAdmin, sebagai Database Management \& Editor
	  \item PHPStorm sebagai IDE utama
	  \item Nano untuk \textit{shell text editor}
	  \item Postman, untuk \textit{debugging} API \textit{calls} dan system tests
	  \item Power Designer untuk alat bantu desain yang berhubungan dengan grafis seperti diagram, \textit{flowchart}, dll.
	  \end{enumerate}
  
\section{Implementasi Antarmuka}
	
    \subsection{Antarmuka Registrasi}
    
    Penjelasan otorisasi terhadap antarmuka A, link yang tersedia dalam antarmuka A, dan penjelasan \textit{exception} jika terjadi masalah baik otorisasi ataupun autentikasi saat mengakses antarmuka ini.
  
      \begin{figure}[H]
        \centering
        \includegraphics[width=\linewidth]{images/bab4/smpso_code.png}
        \caption{ Pseudocode Controller untuk Menampilkan Antarmuka A }
        \label{pdm}
      \end{figure}
      
    \subsection{Antarmuka Halaman B}
    Penjelasan otorisasi terhadap antarmuka B, link yang tersedia dalam antarmuka B, dan penjelasan \textit{exception} jika terjadi masalah baik otorisasi ataupun autentikasi saat mengakses antarmuka ini.
  
      \begin{figure}[H]
        \centering
        \includegraphics[width=\linewidth]{images/bab4/smpso_code.png}
        \caption{ Pseudocode Controller untuk Menampilkan Antarmuka B }
        \label{pdm}
      \end{figure}
    
    
\section{Pemasangan Proyek}
	Pembangunan dilakukan secara online, dan tersebar (tidak hanya menggunakan satu \textit{service provider} saja.
    Berikut dijelaskan langkah-langkah pembangunan proyek:
    
    \subsection{Konfigurasi Domain}
    Domain yang dipilih berasal dari Namecheap.com , dengan langkah-langkah konfigurasi sebagai berikut :
    \begin{enumerate}
    \item Langkah 1
    \item Langkah 2
    \end{enumerate}
    \subsection{Konfigurasi VPS}
    Domain yang dipilih berasal dari DigitalOcean dan Google Cloud Computing , dengan langkah-langkah konfigurasi sebagai berikut :
    \begin{enumerate}
    \item DigitalOcean
      \begin{enumerate}
      \item Langkah 1
      \item Langkah 2
      \end{enumerate}
    \item Google Cloud Computing
      \begin{enumerate}
      \item Langkah 1
      \item Langkah 2
      \end{enumerate}
    \end{enumerate}
    
    \subsection{Konfigurasi PostgreSQL}
    PostgreSQL diinstal dalam VPS, dengan langkah-langkah konfigurasi sebagai berikut :
    \begin{enumerate}
    \item Langkah 1
    \item Langkah 2
    \end{enumerate}
    
    \subsection{Konfigurasi Node.js}
    Node.js diinstall dalam VPS, dengan langkah-langkah konfigurasi sebagai berikut :
    \begin{enumerate}
    \item Langkah 1
    \item Langkah 2
    \end{enumerate}
    
    \subsection{Konfigurasi MongoDB}
    MongoDB diinstall dalam VPS, dengan langkah-langkah konfigurasi sebagai berikut :
    \begin{enumerate}
    \item Langkah 1
    \item Langkah 2
    \end{enumerate}
    
    \subsection{Konfigurasi SMTP Service}
    SMTP \textit{service} yang digunakan berasal dari sendgrid.net, dengan langkah-langkah konfigurasi sebagai berikut :
    \begin{enumerate}
    \item Langkah 1
    \item Langkah 2
    \end{enumerate}

    	\chapter{PENGUJIAN DAN EVALUASI}
	
  \section{Pengujian}

	Pada subbab ini, penulis akan memaparkan pengujian terhadap aplikasi. Pengujian yang dilakukan adalah pengujian fungsionalitas, dimana penulis menggunakan tool Laravel Dusk sebagai \textit{testing code} untuk menguji fungsionalitas aplikasi.\\
	\tabularnewline Dikarenakan keterbatasan waktu, dan atas saran dari pembimbing, penulis tidak menuliskan \textit{testing script}	untuk keseluruhan fungsionalitas yang sudah pasti teruji, seperti Login (sudah menggunakan \textit{facade} Laravel), transaksi CRUD dll.\\
	Pada pemaparan ini, penulis mengidentifikasikan fungsionalitas utama dalam aplikasi lelang ini adalah sebagai berikut :
	\begin{enumerate}
		\item Pengujian Fungsionalitas Lelang
			  \begin{enumerate}
			  	\item Pengujian Penawaran Lelang
			  \end{enumerate}
	   \item Pengujian Fungsionalitas Voucher
			  \begin{enumerate}
			  	\item Pengujian Penggunaan Voucher
			  \end{enumerate}
	\end{enumerate}
	
	Pada bagian ini juga, penulis menuliskan \textit{summary} pengujian fungsionalitas ini pada subbab \ref{summary-pengujian}.
	
	\subsection{Pengujian Fungsionalitas Lelang}
		Pada pengujian ini, terdapat beberapa skenario pengujian yang dipaparkan dalam tabel berikut :
		
	
	\subsection{Pengujian Fungsionalitas Voucher}
		Pada pengujian ini, terdapat beberapa skenario pengujian yang dipaparkan dalam tabel berikut :
	
	
	\subsection{\textit{Summary} Pengujian}
		\label{summary-pengujian}
		
		Pada summary berikut,
		

  \section{Evaluasi}
	Pada subbab ini, penulis akan memaparkan hasil analisa terhadap aplikasi, perspektif non-IT terhadap pengerjaan maupun lingkup pekerjaan dari aplikasi Lelang Online ini.
	
	\subsection{Evaluasi Aspek Kebutuhan}
	
	\subsection{Evaluasi Aspek Teknis}
	\subsection{Evaluasi Aspek \textit{Bussiness Engineering}}
	
		\subsubsection{Evaluasi Regulasi Terkait}
		
		\subsubsection{\textit{Market Analysis}}
		
		\subsubsection{\textit{Competitors Competitiveness}}
		
		\subsubsection{Evaluasi \textit{Startup Burn Rate}}
		
	\subsection{Evaluasi Aspek Analisa dan Desain}
	\subsection{Evaluasi Aspek Implementasi}
	\subsection{Evaluasi Aspek Pengujian}
	
	\subsection{\textit{Summary} Evaluasi}
	
	\subsection{\textit{Further Enchancements}}  
	\subsection{\textit{Further Readings}}

	


    %\section{Struktur Dokumen \LaTeX{}}
Dokumen \LaTeX{} terdiri dari struktur yang dibuat berdasarkan struktur dokumen sehari-hari. Sebagai penulis dokumen, Anda wajib menggunakan struktur ini sehingga \LaTeX{} dapat melakukan hal lain yang membantu Anda dalam mengorganisir dokumen seperti misalnya pembuatan Daftar Isi. Berikut adalah struktur dokumen yang ada di \LaTeX{} diurutkan berdasarkan hirarkinya.

\begin{ltabulary}{|L|L|} % L = Rata kiri untuk setiap kolom, | = garis batas vertikal.

% Kepala tabel, berulang di setiap halaman
\caption{Struktur hirarki dokumen \LaTeX{}} \label{tabelStrukturDokumen} \\
\hline
\textbf{Nama} & \textbf{Peruntukkan} \\ \hline

\endhead
\endfoot
\endlastfoot

% Isi Tabel
\textbf{\textbackslash{}part\{Judul Bagian\}} & \texttt{book} \\ \hline
\textbf{\textbackslash{}chapter\{Judul Bab\}} & \texttt{book} dan \texttt{report} \\ \hline
\textbf{\textbackslash{}section\{Judul Subbab\}} & semua kecuali \texttt{letter} \\ \hline
\textbf{\textbackslash{}subsection\{Judul Subsubbab\}} & semua kecuali \texttt{letter} \\ \hline
\textbf{\textbackslash{}subsubsection\{Judul Subsubsubbab\}} & semua kecuali \texttt{letter} \\ \hline
\textbf{\textbackslash{}paragraph\{Judul Paragraf\}} & semua\\ \hline

\end{ltabulary}

\subsection{Pengujian Performa}
      Pengujian 
      \subsubsection{Pengujian Kecepatan Fitur A}
      Pengujian fitur ini dilakukan pada lingkungan uji \ref{env_uji1}, dan untuk lebih lengkapnya dapat dilihat pada Tabel \ref{uji1}
      \begin{table}[]
      \centering
      \caption{Pengujian Fitur B}
      \label{uji2}
      \begin{tabular}{llll}
      \multicolumn{1}{c}{\textbf{ID}} & \multicolumn{3}{c}{\textbf{TA-UJI.Proses}}        \\
      Referensi Proses Penggunaan     & \multicolumn{3}{l}{}                              \\
      Nama                            & \multicolumn{3}{l}{}                              \\
      Tujuan Pengujian                & \multicolumn{3}{l}{\multirow{2}{*}{}}             \\
      \textbf{Skenario Pengujian}     & \multicolumn{3}{l}{}                              \\
      Langkah Pengujian               & \multicolumn{3}{l}{}                              \\
      Kecepatan Buka Halaman          & Halaman       & \multicolumn{2}{l}{Google Chrome} \\
                                      & A             & 18 KB           & 0.987s         
      \end{tabular}
      \end{table}
      
    \chapter{PENUTUP}
  Bab ini membahas kesimpulan yang dapat diambil dari tujuan pembuatan sistem dan hubungannya dengan hasil uji coba dan evaluasi yang telah dilakukan. Selain itu, terdapat beberapa saran yang bisa dijadikan acuan untuk melakukan pengembangan dan penelitian lebih lanjut.
  \section{Kesimpulan}
  Dari proses perancangan, implementasi dan pengujian terhadap sistem, dapat diambil beberapa kesimpulan berikut:
  \begin{enumerate}
    \item Kesimpulan 1
    \item Kesimpulan 2
    \item Kesimpulan 3
  \end{enumerate}
  
  \section{Saran}
  Berikut beberapa saran yang diberikan untuk pengembangan lebih lanjut:
  \begin{itemize}
  	\item Menggunakan mekanisme \textit{Queue} sebagai \textit{countermeasure} dari masalah \textit{occurence} (di Laravel sudah ada disediakan \textit{base class}nya sendiri).
    \item Mengikutsertakan pihak yang menguasai/spesialisasi di bidang hukum untuk menetapkan peraturan-peraturan terkait
    \item Saran 2 
    \item Saran 3
  \end{itemize}

    \appendix % Halaman lampiran, dengan judul LAMPIRAN X
  \backmatter % Lampiran tanpa judul LAMPIRAN X, untuk BIODATA PENULIS
\end{document}