\begin{longtable}{|c|l|}
		\caption{Tabel Kasus Penggunaan}
		\label{kasus-penggunaan}
	\\
	
	\hline
		\multicolumn{1}{|c|}{\textbf{\begin{tabular}[c]{@{}c@{}}ID \\ Kasus \\ Penggunaan\end{tabular}}} & \multicolumn{1}{c|}{\textbf{Kasus Penggunaan}} \\ \hline
	\endfirsthead
	
	\multicolumn{2}{r}%
	{\tablename\ \thetable{} -- lanjutan dari halaman sebelumnya} \\
	\hline \multicolumn{1}{|c|}{\textbf{\begin{tabular}[c]{@{}c@{}}ID \\ Kasus \\ Penggunaan\end{tabular}}} & \multicolumn{1}{c|}{\textbf{Kasus Penggunaan}} \\ \hline
	\endhead
	
	\hline \multicolumn{2}{|r|}{{Dilanjutkan ke halaman selanjutnya}} \\ \hline
	\endfoot
	
	\hline
	\endlastfoot
	
	KP-01 & Manajemen Authentikasi Pengguna \\ \hline
	KP-02 & Memanajemen Transaksi Lelang \\ \hline
	KP-03 & Manajemen Barang Lelang \\ \hline
	KP-04 & Manajemen Interaksi Antar Pengguna \\ \hline
%	KP-05 & Monitoring Proses Lelang \\ \hline
	KP-05 & Manajemen kupon \\ \hline
\end{longtable}