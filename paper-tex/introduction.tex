
\section{Pendahuluan}
Lelang adalah proses membeli dan menjual barang atau jasa dengan cara menawarkan kepada penawar, menawarkan tawaran harga lebih tinggi, dan kemudian menjual barang kepada penawar harga tertinggi\cite{balailelang_sejarah_nodate}. Transaksi jual beli saat ini sudah dapat dilakukan lewat berbagai cara, antara lain menggunakan \textit{e-commerce}, atau lewat \textit{social media}, atau bisa dengan melelang di aplikasi lelang \textit{online}. Sedikit berbeda dengan teknik penjualan di lelang online, karena aplikasi ini dapat diakses oleh banyak orang, tentu saja pelelang (\textit{auctioneer}) tidak terbatas pada ruang lelang saja, tapi bisa berasal dari manapun selama mereka mengakses aplikasi tersebut.  Lelang \textit{online} ini tentu saja mendatangkan banyak manfaat, selain biaya yang lebih efisien dan hemat, dan juga tidak menguras waktu karena siapapun, kapanpun, dimanapun dapat mengajukan penawaran ataupun melelang barangnya tanpa harus pergi ke instansi tertentu dan melakukan lelang dengan cara konvensional.\\
\indent Aplikasi serupa telah banyak, namun banyak aspek yang kurang dalam aplikasi tersebut, seperti informasi dari lelang tidak \textit{reliable} (misal: stok barang ternyata sudah habis), alur proses yang tidak jelas sehingga membingungkan pengguna aplikasi, informasi yang kurang jelas, dan produk yang didapatkan ternyata tidak sesuai dengan informasi pada saat produk dilelang (\textit{bad information}) \cite{ying-feng_kuo_online_2016}.\\
\indent Alur proses yang kurang diperhatikan oleh para developer aplikasi lelang \textit{online} menjadi beberapa alasan yang kuat mengapa lelang online masih kurang diminati\cite{noauthor_sistem_nodate}. Selain itu, bidang bisnis yang menuntut perubahan secara cepat tentu saja harus diadaptasi sehingga aplikasi bersifat fleksibel dengan \textit{maintainability} yang tinggi.

% \subsection{Subsection Heading Here}
% \blindtext
