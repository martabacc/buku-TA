
\section{Kesimpulan}
Dari hasil pengamatan selama proses perancangan hasil uji coba yang telah dilakukan terhadap sistem yang dikembangkan, diambil kesimpulan sebagai berikut:
\begin{enumerate}
	\item \textit{Software design} adalah tahap yang sangat penting dan harus dibuat sefleksibel mungkin mengingat \textit{e-commerce} adalah bisnis menuntut perubahan yang cepat, sehingga agar tidak tersaingi kompetitor, harus dirancang sedemikian rupa agar perubahan tersebut dapat diimplementasi tanpa \textit{cost} (misal \textit{refactoring}) yang besar.
	\item Aspek dan \textit{advices} bisnis sangat penting dalam membuat aplikasi \textit{e-commerce}, dimana \textit{well-tailored to customer application}-lah yang dapat memenangkan pasar. Oleh karena itu, sangat penting untuk meninjau aspek \textit{usability} dan \textit{user experience} agar memberi kesan positif pada pengguna, dan pengguna tetap mau bertransaksi kembali dalam aplikasi tersebut.
\end{enumerate}

Pengembangan yang dapat dilakukan berikutnya yaitu dengan:
\begin{enumerate}
	\item \textit{Image Optimization} agar \textit{loading time} dapat dikurangi, dan meningkatkan \textit{user experience} dalam aspek \textit{application performance};
	\item \textit{Advanced data management \& searching}, dimana \textit{data growth} dalam \textit{e-commerce} sangatlah masif sehingga dibutuhkan teknik khusus lebih dari sekedar \textit{query}; dan
	\item \textit{Market Engagement} - menjaga \textit{market} yang telah dibangun dengan menggunakan pendekatan statistik atau \textit{machine learning}, dapat dilakukan \textit{customer scoring} dan \textit{early fraud detection} lewat analisa riwayat transaksi pengguna. Dengan \textit{score} tersebut, kita dapat merekomendasikan barang-barang yang sesuai dengan pengguna dan \textit{early-fraud detection} untuk menciptakan lingkungan lelang yang lebih aman dan terpercaya pada pengguna.
\end{enumerate}