\begin{longtable}{|c|l|X|}
 	\caption{Identifikasi aktor dalam sistem lelang online}
 	\label{identifikasi-aktor} \\ \hline
 	
 	\textbf{No} & \textbf{Aktor} & \textbf{Keterangan} \\ \hline
 	\endfirsthead
 	
 	\multicolumn{3}{c}%
 	{\tablename\ \thetable{} -- lanjutan dari halaman sebelumnya} \\ \hline
 	\textbf{No} & \textbf{Aktor} & \textbf{Keterangan} \\ \hline
 	\endhead
 	
 	
 	\hline \multicolumn{3}{|r|}{{Dilanjutkan ke halaman selanjutnya}} \\ \hline
 	
 	\endfoot
 	
 	\hline
 	
 	\endlastfoot
	 	1	&	Pengguna	&	Pengguna yang dimaksud adalah pengguna yang menggunakan fungsionalitas lelang dalam sistem. Pengguna dapat menjadi seorang \textit{bidder} ataupun seorang \textit{auctioneer}, dimana pengguna dapat menjual barang untuk dilelang, dan dapat pula melelang barang \\ \hline
	 	2	&	\textit{Administrator}	&	Bertugas melihat dan mengawasi jalannya lelang, melihat laporan keluhan dari pengguna, serta melakukan \textit{block} pengguna jika terdeteksi adanya tindakan yang tidak sesuai dengan kaidah lelang.	\\ \hline 	
	 	
	 \end{longtable}