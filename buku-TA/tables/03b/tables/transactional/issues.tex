 \begin{longtable}{|r|X|X|X|}
 	\caption{Kamus Data Tabel \textit{issues}}
 	\label{db-issues} \\ \hline
 	\textbf{No} & \textbf{Nama Atribut} & \textbf{Tipe Data} & \textbf{Keterangan} \\ \hline
 	\endfirsthead
 	
 	\multicolumn{4}{c}%
 	{\tablename\ \thetable{} -- lanjutan dari halaman sebelumnya} \\ \hline
 	\textbf{No} & \textbf{Nama Atribut} & \textbf{Tipe Data} & \textbf{Keterangan} \\ \hline
 	\endhead
 	
 	\hline
 	\endlastfoot
 	
\multicolumn{1}{|c|}{	1	}&	id	&	int	&	PK, Autoincrement	\\ \hline
\multicolumn{1}{|c|}{	2	}&	id\_issue\_type	&	int	&	FK ID tipe issue yang dilaporkan	\\ \hline
\multicolumn{1}{|c|}{	3	}&	id\_user	&	int	&	FK ID user yang melaporkan	\\ \hline
\multicolumn{1}{|c|}{	4	}&	id\_issued\_object	&	int	&	FK ID item yang dilaporkan	\\ \hline
\multicolumn{1}{|c|}{	5	}&	description	&	text	&	Deskripsi Laporan	\\ \hline


 \end{longtable}