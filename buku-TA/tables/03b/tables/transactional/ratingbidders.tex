 \begin{longtable}{|r|X|X|X|}
 	\caption{Kamus Data Tabel \textit{ratingbidders}}
 	\label{db-ratingbidders} \\ \hline
 	\textbf{No} & \textbf{Nama Atribut} & \textbf{Tipe Data} & \textbf{Keterangan} \\ \hline
 	\endfirsthead
 	
 	\multicolumn{4}{c}%
 	{\tablename\ \thetable{} -- lanjutan dari halaman sebelumnya} \\ \hline
 	\textbf{No} & \textbf{Nama Atribut} & \textbf{Tipe Data} & \textbf{Keterangan} \\ \hline
 	\endhead
 	
 	\hline
 	\endlastfoot
 	
\multicolumn{1}{|c|}{	1	}&	id	&	int	&	PK, Autoincrement	\\ \hline
\multicolumn{1}{|c|}{	2	}&	id\_user\_bidder	&	varchar(50)	&	FK ID User Pelelang	\\ \hline
\multicolumn{1}{|c|}{	3	}&	id\_user\_rater	&	int	&	FK ID User pemberi review	\\ \hline
\multicolumn{1}{|c|}{	6	}&	rate	&	bigint	&	Jumlah rating yang diberikan	\\ \hline
\multicolumn{1}{|c|}{	7	}&	id\_item	&	int	&	FK ID barang yang dirating	\\ \hline
\multicolumn{1}{|c|}{	8	}&	rate\_message	&	text	&	Deskripsi Rate	\\ \hline



 \end{longtable}