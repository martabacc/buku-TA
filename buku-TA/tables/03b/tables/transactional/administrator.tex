 \begin{longtable}{|c|X|X|X|}
 	
 	\caption{Spesifikasi Tabel Administrator}
 	\label{db-administrator} \\ \hline
 	
 	\textbf{No} & \textbf{Nama Atribut} & \textbf{Tipe Data} & \textbf{Keterangan} \\ \hline
 	\endfirsthead
 	
 	\multicolumn{4}{c}%
 	{\tablename\ \thetable{} -- lanjutan dari halaman sebelumnya} \\ \hline
 	\textbf{No} & \textbf{Nama Atribut} & \textbf{Tipe Data} & \textbf{Keterangan} \\ \hline
 	\endhead
 	
 	
 	\hline \multicolumn{4}{|r|}{{Dilanjutkan ke halaman selanjutnya}} \\ \hline
 	
 	\endfoot
 	
 	\hline
 	
 	\endlastfoot
 	
 	
 	
 	\multicolumn{1}{|c|}{1}	&	id	&	int	&	PK , Auto increment	\\  \hline
 	\multicolumn{1}{|c|}{2}	&	name	&	varchar(255)	&	Nama	\\  \hline
 	\multicolumn{1}{|c|}{3}	&	username	&	varchar(255)	&	Username yang digunakan	\\  \hline
 	\multicolumn{1}{|c|}{4}	&	password	&	varchar(255)	&	Password	\\  \hline
 	\multicolumn{1}{|c|}{5}	&	created\_at	&	timestamp	&	Timestamp default laravel	\\  \hline
 	\multicolumn{1}{|c|}{6}	&	updated\_at	&	timestamp	&	Timestamp default laravel	\\  \hline
 	\multicolumn{1}{|c|}{7}	&	email	&	varchar(255)	&	Email	\\  \hline
 	
 \end{longtable}