\begin{longtable}{|l|X|X|}
	\caption{Kebutuhan Non-Fungsional Aplikasi Lelang Online}
	\label{tabel-non-fung}
	\\
	
	\hline \multicolumn{1}{|r}{\textbf{No. }} & \multicolumn{1}{|l}{\textbf{Parameter}} & \multicolumn{1}{|l|}{\textbf{Ketersediaan}} \\ \hline  
	\endfirsthead
	
	\multicolumn{3}{c}%
	{\tablename\ \thetable{} -- lanjutan dari halaman sebelumnya} \\ 
	\hline \multicolumn{1}{|r}{\textbf{No. }} & \multicolumn{1}{|l}{\textbf{Parameter}} & \multicolumn{1}{|l|}{\textbf{Ketersediaan}} \\ \hline 
	\endhead
	
	\hline \multicolumn{3}{|r|}{{.. \textit{dilanjutkan ke halaman selanjutnya}}} \\ \hline
	\endfoot
	
	\hline
	\endlastfoot
	
	1  & Ketersediaan & Aplikasi harus dapat berjalan pada browser, tanpa dibatasi waktu dan tempat selama browser tersebut tersambung ke jaringan internet. \\ \hline
	2 & Bahasa & Bahasa yang digunakan pada antarmuka merupakan bahasa Indonesia \\ \hline
	3 & Otorisasi & Setiap pengguna hanya berhak mengakses dan mengubah data yang merupakan milik pengguna tersebut.\\ \hline
	4 & Kecepatan & Rata-rata waktu akses halaman tidak boleh lebih dari 3 detik. \\ \hline
	6 & \textit{Positive User Experience} & Aplikasi memberikan kesan positif terhadap pengalaman penggnuaan aplikasi  \\ \hline
	7 & \textit{Security} & Koneksi terhadap web harus terlindung \textit{https}.\\ \hline
	8 & \textit{Maintainability} & Aplikasi haruslah bersifat \textit{maintainable} terhadap developer, dan tidak terlalu sensitif terhadap perubahan (jika terdapat perubahan fitur di masa depan, tidak harus \textit{merefactor} atau mengubah struktur program secara keseluruhan).\\ \hline
\end{longtable}