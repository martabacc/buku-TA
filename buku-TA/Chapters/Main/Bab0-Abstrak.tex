\begin{abstrak}
		E-commerce adalah kombinasi antara dunia digital dan transaksi lelang. Di Indonesia, seiring terjadi peningkatan jumlah pengguna internet dan menjamurnya bisnis online atau sering disebut \textit{online shop}. Salah satu jenis transaksi adalah lelang, yaitu metode jual beli yang mengintegrasikan mekanisme lelang dengan Internet.
	    \newline
	    \indent Dalam interaksi antara pelaku lelang online (penjual dan pembeli) pasti terjadi kegagalan/ketidakpuasan dalam transaksi lelang online.Berangkat dari paper "Online auction service failures in   {Taiwan}: {Typologies} and recovery strategies" yang membahas mengenai analisa kesalahan dan strategi lewat survey terhadap pengguna aplikasi lelang online di Taiwan, penulis membangun aplikasi lelang online yang disertai dengan tambahan fitur maupun saran dari paper tersebut.
	    \newline 
	    \indent Selain itu, penulis juga menganalisa aplikasi \textit{e-commerce} yang umum digunakan di Indonesia baik \textit{user experience} maupun alur transaksi, dan menambahkan beberapa fitur agar lebih sesuai dengan transaksi jual-beli online yang umum di Indonesia. Dengan tugas akhir ini, diharapkan kegagalan dalam transaksi online dapat diperbaiki dan membuka peluang lelang online untuk meramaikan industri \textit{e-commerce} di Indonesia.\\
\noindent \textbf{Kata-Kunci}: \textit{lelang online}, \textit{strategi }
\end{abstrak}