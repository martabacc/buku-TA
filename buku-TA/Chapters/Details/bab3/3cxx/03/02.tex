% Memperbarui informasi barang yang dilelang

\begin{table}[H]
	\centering
	\caption{Spesifikasi Kasus Penggunaan : Mendaftarkan Barang Lelang}
	\label{uc03.02}
	\begin{tabular}{|r|p{8cm}|}
		\hline
		\textbf{Kode}                                                    & UC-03.02                                                     \\ \hline
		\textbf{Nama}                                                    & \textbf{Memperbarui informasi barang yang dilelang} \\ \hline
		\textbf{Aktor}                                                   & Pengguna 
			\\ \hline
		\textbf{Deskripsi}                                               & Pengguna memperbarui informasi barang yang sebelumnya sudah terdaftar di dalam sistem 
			 \\ \hline
		\textbf{Tipe}                                                    & Fungsional 
			\\ \hline
		\textbf{\textit{Precondition}}
			& Informasi barang belum diperbarui. \\ \hline
		\textbf{\textit{Postcondition}} 
			& Informasi barang sudah diperbarui. \\ \hline
		\multicolumn{2}{|c|}
			{\textbf{Alur Kejadian Normal}}                                                                            \\ \hline
		\multicolumn{1}{|l|}{}                                           & 
			\begin{enumerate}
				\item Pengguna dalam keadaan terautentikasi, mengklik "Item Anda" -> "Manage Items" pada \textit{navbar} bagian atas halaman.
				\item \label{uc0302-show1page}Sistem menampilkan halaman yang berisi daftar barang yang didaftarkan pengguna.
				\item Pengguna mengklik barang yang ingin diperbarui informasinya
				\item \label{uc0302-show2page}Sistem menampilkan halaman \textit{form} "Perbarui barang".
				\item Pengguna mengisi informasi pembaruan barang di dalam form tersebut.
				\item Setelah selesai, pengguna mengklik tombol "Simpan Pembaruan".
				\item \label{al-0302-a} Sistem memvalidasi data (termasuk file gambar) yang dimasukkan pengguna, lalu sistem me\textit{redirect} ke halaman "Kelola Barang" dalam keadaan barang baru sudah ditambahkan.
			\end{enumerate}
		\\ \hline
		
		\multicolumn{2}{|c|}{\textbf{Alur Kejadian Alternatif}}                                                         \\ \hline
		\multicolumn{1}{|l|}{}                                           & 	
			-
			\\ \hline
%		\multicolumn{1}{|l|}{}                                           & 
%			 \begin{itemize}
%			 	\item[\ref{al-0302-a}a.] Sistem tidak dapat memvalidasi data yang dimasukkan pengguna.
%			 	\item[\ref{al-0302-a}b.] Sistem me\textit{redirect} ke halaman "Perbarui Barang" (langkah \ref{uc0302-show2page}) dengan \textit{error message}.
%			 \end{itemize}
%		 \\ \hline
%		 \multicolumn{1}{|l|}{}                                           & \textbf{Gambar yang dimasukkan pengguna tidak dapat divalidasi}
%		 \\ \hline
%		 \multicolumn{1}{|l|}{}                                           & 
%		 \begin{itemize}
%		 	\item[\ref{al-0302-a}a.] Sistem menampilkan \textit{error modal} berisi peringatan bahwa "Kesalahan saat \textit{upload} gambar, silahkan coba lagi"
%		 	\item[\ref{al-0302-a}b.] Sistem me\textit{redirect} ke halaman form "Perbarui Barang" (langkah \ref{uc0301-show2page}) dengan \textit{error message}.
%		 \end{itemize}
%		 \\ \hline
	\end{tabular}
\end{table}