	% Memasukkan kupon
	
	\begin{table}[H]
		\centering
		\caption{Spesifikasi Kasus Penggunaan: Menambahkan Kupon}
		\label{uc06.01}
		\begin{tabular}{|r|p{8cm}|}
			\hline
			\textbf{Kode}
			& UC-05.01
			\\ \hline
			\textbf{Nama}
			& \textbf{Menambahkan Kupon} 
			\\ \hline
			\textbf{Aktor}    
			& \textit{Administrator} 
			\\ \hline
			\textbf{Deskripsi}
			& \textit{Administrator} ingin menggunakan kupon/voucher yang ia miliki untuk pada sebuah transaksi
			\\ \hline
			\textbf{Tipe}
			& Fungsional 
			\\ \hline
			\textbf{\textit{Precondition}}
			& Kupon baru belum berhasil tersimpan di sistem
			\\ \hline
			\textbf{\textit{Postcondition}} 
			& Kupon baru berhasil tersimpan di sistem
			\\ \hline
			\multicolumn{2}{|c|}
			{\textbf{Alur Kejadian Normal}}
			\\ \hline
			\multicolumn{1}{|l|}{} & 
			\begin{enumerate}
				\item \textit{Administrator} membuka halaman 'Tambah Kupon'
				\item Sistem menampilkan halaman \textit{form} Tambah Kupon
				\item \textit{Administrator} memasukkan informasi kupon baru yang akan ditambahkan
				\item Sistem mengecek permintaan penggunaan kupon,jika permintaan dapat diverifikasi dan valid, sistem melakukan \textit{redirect} ke halaman manajemen kupon.
				% \item \label{uc0301-show1page}Sistem menampilkan halaman yang berisi form pendaftaran barang
				% \item \label{al-0301-a} Sistem memvalidasi data yang dimasukkan pengguna
			\end{enumerate}
			\\ \hline
			\multicolumn{2}{|c|}{\textbf{Alur Kejadian Alternatif}} \\ \hline
			\multicolumn{1}{|l|}{}                   
			& -
			\\ \hline
		\end{tabular}
	\end{table}