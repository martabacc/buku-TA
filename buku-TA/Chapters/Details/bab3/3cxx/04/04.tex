	% Mengirim pesan
	
	
	\begin{table}[H]
		\centering
		\caption{Spesifikasi Kasus Penggunaan : Mengirimkan Pesan}
		\label{uc04.04}
		\begin{tabular}{|r|p{8cm}|}
			\hline
			\textbf{Kode}
			& UC-04.04
			\\ \hline
			\textbf{Nama}
			& \textbf{ Mengirim Pesan } 
			\\ \hline
			\textbf{Aktor}    
			& Pengguna 
			\\ \hline
			\textbf{Deskripsi}
			& Pengguna akan mengirimkan pesan kepada pengguna lainnya
			\\ \hline
			\textbf{Tipe}
			& Fungsional 
			\\ \hline
			\textbf{\textit{Precondition}}
			& Pesan yang dikirimkan pengguna belum tersimpan pada sistem
			\\ \hline
			\textbf{\textit{Postcondition}} 
			& Pesan yang dikirimkan pengguna berhasil tersimpan pada sistem
			\\ \hline
			\multicolumn{2}{|c|}
			{\textbf{Alur Kejadian Normal}}
			\\ \hline
			\multicolumn{1}{|l|}{} & 
			\begin{enumerate}
				\item Pengguna mengklik \textit{URL} pengguna tujuan yang ingin dikirimi pesan
				\item Sistem menampilkan halaman profil pengguna tujuan
				\item Pengguna mengklik tombol "Kirim Pesan"
				\item Sistem menampilkan halaman percakapan dengan pengguna tersebut.
				\item \label{al-0404-ex} Pengguna memasukkan pesan yang ingin dikirimkan. lalu klik tombol 'Kirim'
				\item \label{al-0404-a}Sistem mengirimkan pesan, lalu sistem kembali menampilkan halaman pengguna dengan informasi pesan yang sudah terkirim muncul di riwayat percakapan pengguna dengan pengguna tujuan				
				% \item \label{uc0301-show1page}Sistem menampilkan halaman yang berisi form pendaftaran barang
				% \item \label{al-0301-a} Sistem memvalidasi data yang dimasukkan pengguna
			\end{enumerate}
			\\ \hline
			\multicolumn{2}{|c|}{\textbf{Alur Kejadian Alternatif}} \\ \hline
			\multicolumn{1}{|l|}{}                   
			& - \\ \hline
		\end{tabular}
	\end{table}