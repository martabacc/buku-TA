% Melihat daftar barang yang dilelang

\begin{table}[H]
	\centering
	\caption{Spesifikasi Kasus Penggunaan : Melihat barang yang dilelang}
	\label{uc02.01}
	\begin{tabular}{|r|p{8cm}|}
		\hline
		\textbf{Kode}                                                    & UC-02.01                                                     \\ \hline
		\textbf{Nama}                                                    
			& \textbf{Melihat daftar barang yang dilelang}                                         
			\\ \hline
		\textbf{Aktor}                                                   & Pengguna                                                    \\ \hline
		\textbf{Deskripsi}
			& Pengguna melihat daftar barang yang sedang dilelang
			\\ \hline
		\textbf{Tipe}                                                    
			& Fungsional
			\\ \hline
		\textbf{\textit{Pre Condition}}
			& Sistem belum menampilkan daftar barang yang sedang dilelang
			\\ \hline
		\textbf{\textit{Post Condition}}
			& Sistem menampilkan daftar barang yang sedang dilelang
			\\ \hline
		\multicolumn{2}{|c|}{\textbf{Alur Kejadian Normal}}                                                                            \\ \hline
		\multicolumn{1}{|l|}{}                                           & 
		\begin{enumerate}
			\item Pengguna mengklik \textit{icon} aplikasi di kiri atas halaman
			\item Sistem menampilkan halaman depan yang berisi daftar barang yang sedang dilelang
				  \newline
				  \textit{Ket : Pada halaman depan, ditampilkan barang sesuai dengan kategori berdasarkan waktu dan popularitas, seperti Hot Item (barang yang paling ramai transaksi bidnya), Newest Item, dll.}
		\end{enumerate}
		\\ \hline
		\multicolumn{2}{|c|}{\textbf{Alur Kejadian Alternatif}}                                                         \\ \hline
		\multicolumn{1}{|l|}{}                                           
			& -
		\\ \hline
	\end{tabular}
\end{table}