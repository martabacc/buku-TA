
\subsection{Perancangan \textit{Database}}
Pada subbab ini akan dijelaskan bagaimana rancangan basis data yang digunakan pada aplikasi lelang online ini. Sistem basis data yang digunakan pada aplikasi ini menggunakan dua jenis database, yaitu transaksional (menggunakan PostgreSQL) dan nontransaksional (menggunakan MongoDB). \textit{Conceptual Data Model} (CDM) dan \textit{Physical Data Model }(PDM) dari basis data sistem ini dijelaskan pada Gambar \ref{cdm} dan Gambar \ref{pdm}.
	\subsubsection{\textit{Conceptual Database Model}}
	\begin{figure}[H]
		\centering
		\includegraphics[width=\textwidth]{images/bab3/db/cdm.png}
		\caption{\textit{Conceptual Database Model} (PDM) Aplikasi}
		\label{cdm}
	\end{figure}
	
	
	\subsubsection{\textit{Physical Database Model}}
	\begin{figure}[H]
		\centering
		\includegraphics[width=\textwidth]{images/bab3/db/pdm.png}
		\caption{\textit{Physical Database Model} (PDM) Aplikasi}
		\label{pdm}
	\end{figure}
	
%  Konsep Item Caching
%  Pyshical Database Design
%   	% \begin{longtable}{|r|l|l|l|}
% \caption{Spesifikasi Tabel Z}
% \label{users-db}
% \\
% 	\hline
% 	\multicolumn{4}{|c|}{\textbf{
% 		Tabel Users
% 		}} \\ 
% 	\hline
% 	\textbf{Deskripsi} 
% 		& \multicolumn{3}{l|}{
% 			Tabel ini menyimpan informasi dan data \textit{credentials} pengguna.
% 		} \\ 
% 	\hline
% 	\textbf{Penyimpanan} 
% 		& \multicolumn{3}{l|}{Transaksional / PostgreSQL} \\ 
% 	\hline
% 	\textbf{\textbf{\textit{Data Growth}}} 
% 		& \multicolumn{3}{l|}{Linear} \\ 
% 		\hline
% \multicolumn{4}{|c|}{\textbf{Penjelasan Kolom Tabel User}} \\ \hline
% \multicolumn{1}{|l|}{No} & Nama Atribut & Tipe Data & Keterangan \\ \hline
% 	\endfirsthead

	
% 	\multicolumn{4}{c}%
% 	{\tablename\ \thetable{} -- lanjutan dari halaman sebelumnya} \\
% 	\hline \multicolumn{1}{|l|}{No} & Nama Atribut & Tipe Data & Keterangan \\ \hline} \\ \hline
% 	\endhead
	
% 	\hline \multicolumn{4}{|r|}{{Dilanjutkan ke halaman selanjutnya}} \\ \hline
% 	\endfoot
	
% 	\hline
% 	\endlastfoot
	



% 		\end{longtable}
    