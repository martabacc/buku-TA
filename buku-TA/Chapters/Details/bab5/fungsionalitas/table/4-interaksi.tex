\begin{longtable}{|X|X|}
		\caption{Pengujian Fungsionalitas Fitur Manajemen Interaksi Antarpengguna}
		\label{uji-fungsional-4-interaksi}
	\\
	
	\hline
		\textbf{ID} & \textbf{UJ-1-KP04} \\ \hline
	\endfirsthead
	
	\multicolumn{2}{r}%
	{\tablename\ \thetable{} -- lanjutan dari halaman sebelumnya} \\
	\hline 
		\textbf{ID} & \textbf{UJ-1-KP04} \\ \hline
	\endhead
	
	\hline \multicolumn{2}{|r|}{{Dilanjutkan ke halaman selanjutnya}} \\ \hline
	\endfoot
	
	\hline
	\endlastfoot
	
	\textbf{Referensi Kasus Penggunaan}
		& KP04 \\ \hline
	\textbf{Nama}
		& Pengujian fitur manajemen interaksi antarpengguna \\ \hline
	\textbf{Skenario 1}
		& Menguji fitur melihat review pengguna \\ \hline
	Kondisi Awal
		& Sistem menampilkan halaman profil pengguna\\ \hline
	Data Uji
		& Data uji menggunakan data penulis \\ \hline
	Langkah pengujian
		& Membuka halaman profil pengguna yang ingin dilihat \textit{review}nya. \\ \hline
	Hasil yang Diharapkan
		& Sistem berhasil menampilkan profil pengguna yang ingin dilihat \textit{review}nya. \\ \hline	
	Hasil Pengujian
		& 100\% berhasil \\ \hline	
	Kondisi Akhir
		& \textit{Screenshot} pengujian ini dapat dilihat pada Gambar \ref{ss-kp04-01} \\ \hline	

	\textbf{Skenario 2}
		& Menguji fitur mencari barang \\ \hline
	Kondisi Awal
		& Sistem menampilkan halaman dengan elemen \textit{input} search barang \\ \hline
	Data Uji
		&  \\ \hline
	Langkah pengujian
		& \begin{enumerate}
		\item Elemen \textit{input search} diisi dengan kata kunci ``Jersey''
		\item Setelah selesai mengisi, mengklik tombol ``Search''
	\end{enumerate} \\ \hline
	Hasil yang Diharapkan
		& Barang yang mengandung kata ``Jersey'' muncul dalam hasil pencarian \\ \hline
	Hasil Pengujian
		& 100\% berhasil \\ \hline	
		
		
	\textbf{Skenario 3}
		& Menguji fitur berkirim pesan \\ \hline
	Kondisi Awal
		& Pengguna sedang membuka halaman perpesanan kepada pengguna yang ingin dituju\\ \hline
	Data Uji
		& Data ujinya adalah pesan ``Apakah barang X masih ada?'' \\ \hline
	Langkah pengujian
		& \begin{enumerate}
		\item Memasukkan data uji ke dalam elemen \textit{input} 
		\item Mengklik tombol ``Kirim Pesan''
	\end{enumerate} \\ \hline
	Hasil yang Diharapkan
		& Pesan berhasil terkirim dan di\textit{update} di halaman yang sedang dibuka secara \textit{realtime} \\ \hline
	Hasil Pengujian
		& 100\% berhasil \\ \hline	
		
	\textbf{Skenario 4}
		& Menguji fitur melihat daftar pesan \\ \hline
	Kondisi Awal
		& Sistem menampilkan halaman daftar pesan\\ \hline
	Data Uji
		& -\\ \hline
	Langkah pengujian
		& - \\ \hline
	Hasil yang Diharapkan
		& Daftar pesan berhasil ditampilkan \\ \hline
	Hasil Pengujian
		& 100\% berhasil \\ \hline	
		
\end{longtable}