	
	Metode pengujian dirancang sesuai dengan kebutuhan fungsional dan nonfungsional yang telah didefinisikan. Metode-metode pengujian yang digunakan pada tugas akhir ini adalah sebagai berikut:
	
	\begin{enumerate}
	\item \textbf{Pengujian Fungsionalitas} \\
	Pengujian fungsionalitas sistem dilakukan secara mandiri dengan menyiapkan sejumlah skenario. Deskripsi proses pengujian secara lengkap akan dijelaskan pada Subbab \ref{uji-fungsional}.
	
	\item \textbf{Pengujian Kecepatan}\\
	Pengujian kecepatan dilakukan pada setiap kasus penggunaan, yaitu pencatatan waktu yang untuk menampilkan sebuah halaman dan atau melakukan sebuah \textit{request} ke aplikasi. Deskripsi proses pengujian performa akan secara lengkap dijelaskan pada Subbab \ref{uji-performa}.
	
	\item \textbf{Pengujian \textit{User Experience}} \\
	Pengujian \textit{user experience} dilakukan secara statistik, untuk menguji apakah bahwa benar aplikasi yang dibangun memberikan \textit{positive user experience} kepada penggunanya. \textit{Key performance indicator} yang digunakan didasarkan pada paper ``\textit{Development of an Instrument Measuring User Satisfaction of the Human-Computer Interface}''. Deskripsi proses pengujian performa akan secara lengkap dijelaskan pada Subbab \ref{uji-userexperience}.
	
	\item \textbf{Pengujian \textit{Maintainability}} \\
	Pengujian \textit{maintainability} dimaksudkan untuk menguji apakah benar aplikasi yang dibangun bersifat \textit{maintainable} kepada \textit{developer}. Pengujian ini dilakukan secara statistik, dengan mengacu kepada paper ``A Software Maintainability Evaluation Methodology''. Deskripsi proses pengujian secara lengkap akan dijelaskan pada Subbab \ref{uji-maintainability}.
	\end{enumerate}
