\subsection{Strategi \textit{Deployment} \textsc{Vue.js}}
	% todo inline numbering
	
\begin{enumerate}
\item \textbf{\textit{Package Dependencies}} \\
	Pada versi terbaru Laravel (5.4*), Laravel secara \textit{default} menyertakan \textit{package} Laravel Mix - yaitu fitur untuk \textit{compiling assets} dengan Webpack, dengan hasil akhir \textit{compiled assets} (terutama \textit{script} Javascript) yang eksekusinya jauh lebih cepat, karena menggunakan V8 -- sebuah \textit{engine} Javascript yang telah dioptimasi yang bersifat \textit{just-in-time} (JIT) yang memproduksi \textit{machine code} dari sebuah \textit{script} Javascript lalu dieksekusi.\\
  
  \textbf{\textit{Main Problem}} \\
  Masalah muncul saat versi Laravel yang digunakan untuk membangun aplikasi adalah versi (5.3) -- dan jika Laravelnya di\textit{upgrade}, tidak ada jaminan bahwa \textit{deprecated dependencies} (keadaan dimana sebuah \textit{package} tidak di\textit{support} oleh versi terbaru) -- yang berarti harus \textit{refactoring code} yang pasti memakan waktu lama.\\			  

	\textbf{\textit{Insights}} \\
	Penulis menganalisa perbedaan mendasar package.json antara Laravel 5.3 dan 5.4 adalah sebagai berikut:
	\begin{enumerate}[label={\alph*}.]
  	  	\item Basis : Perubahan basis yang awalnya Gulp menjadi Webpack
  	  	\item \textit{Dependencies} : Webpack ternyata menggunakan beberapa plugin tambahan yang tidak diakomodasi dalam package.json di versi 5.3
  	  	\item \textit{Run Script} : Terdapat beberapa perubahan signifikan terhadap \textit{run script alias} di versi 5.4 - dibandingkan pada versi 5.3.
  	  	\item \textit{Additional Files} : Terdapat beberapa file konfigurasi tambahan agar proses kompilasi aset dapat berjalan dengan baik.
	\end{enumerate}
	\ \\
  
  \textbf{\textit{Solution}} \\
  Penulis lalu mengoreksi dan \textit{update package.json} dengan pendekatan \textit{trial and error}, dan bisa terselesaikan dengan script berikut :
	\begin{lstlisting}[language=json]
{ "private": true,
  "scripts": 	{
	"_comment" : "Lists of running npm commands defined here"
  },
  "devDependencies": {
	"axios": "^0.15.3",
	"bootstrap-sass": "^3.3.7",
	"cross-env": "^3.2.3",
	"jquery": "^3.1.1",
	"laravel-mix": "0.*",
	"lodash": "^4.17.4",
	"vue": "^2.1.10"
  },
  "dependencies": {
	"vue-resource": "^1.3.1"
  } }
\end{lstlisting}	  	

			  	
	\ \\		  
\item 
	\textbf{\textit{Dependencies Optimization}} \textbf{\textit{Problem}} \\
	Setelah menulis beberapa \textit{script} Vue, penulis menganalisa bahwa setiap \textit{script} Vue ternyata mempunyai \textit{dependencies} yang sama, yaitu axios, Promise, toastr dan vue. Setiap file Vue meng\textit{include} sebuah \textit{script} yang berisi:
	\begin{lstlisting}[style=htmlcssjs]
	window.axios = require('axios');
	window.toastr = require('toastr');
	window. = require('toastr');
	require('vue-resource');
\end{lstlisting}

	Hal ini mengakibatkan semua file vue yang di\textit{compile} ukurannya cukup besar (~400KB), padahal sebenarnya di dalam setiap file tersebut sebenarnya ada yang sama. Hal ini tentu tidak efektif, karna sebenarnya hal-hal yang sama tersebut bisa dipisahkan, dan dijadikan \textit{cache} sehingga \textit{loading} halaman bisa jauh lebih cepat.\\
		
	\textbf{\textit{Insight \& Solution}} \\
	Setelah penulis berdiskusi di forum Slack, beberapa pengguna Vue menyarankan untuk \textit{compile} keseluruhan \textit{dependencies} yang digunakan kedalam satu file terpisah, dan hanya menulis logika Vue untuk setiap file Vue.\\
	Isi file webpack.mix.js (file yang di\textit{compile} oleh Webpack)	menjadi seperti berikut.
	
\begin{lstlisting}[style=htmlcssjs]
/* dependencies all compiled into one single file */
mix.js('scripts/dependencies.js', 'public/js');

/* dependencies all compiled into one single file */
mix.js('scripts/favorites.js', 'public/js');
mix.js('scripts/other_vue_script.js', 'public/js');
\end{lstlisting}
		
	Dan di HTML, untuk \textit{including} \textit{script} dituliskan seperti berikut:
	\begin{lstlisting}[style=htmlcssjs]
<script src="dependencies.js" ></script>
<script src="custom_page_script.js" ></script>
\end{lstlisting}

\end{enumerate}