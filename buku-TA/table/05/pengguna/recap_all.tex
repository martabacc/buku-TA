\begin{longtable}{|X|c|c|c|}
\caption{Rekapitulasi Hasil Pengujian Pengguna Aplikasi Lelang \textit{Online}} 
\label{user-test-recap} 
\\

\hline\textbf{Parameter} 
& \textbf{\begin{tabular}[c]{@{}c@{}}Rata-rata\\ Nilai\\ Aplikasi Lain\end{tabular}}
& \textbf{\begin{tabular}[c]{@{}c@{}c@{}c@{}}Rata-rata\\ Nilai\\ Aplikasi\\ Lelangapa\end{tabular}} 
& \textbf{\begin{tabular}[c]{@{}c@{}}Persentase \\ Perbedaan \end{tabular}} \\ \hline 
\endfirsthead

\multicolumn{4}{c}%
{\tablename\ \thetable{} -- dilanjutkan dari halaman sebelumnya} \\
\hline\textbf{Parameter} 
& \textbf{\begin{tabular}[c]{@{}c@{}}Rata-rata\\ Nilai\\ Aplikasi Lain\end{tabular}}
& \textbf{\begin{tabular}[c]{@{}c@{}c@{}}Rata-rata\\ Nilai\\ Aplikasi\\	 Lelangapa\end{tabular}} 
& \textbf{\begin{tabular}[c]{@{}c@{}}Persentase \\ Perbedaan \end{tabular}} \\ \hline 
\endhead

\hline \multicolumn{4}{|r|}{{dilanjutkan ke halaman setelahnya}} \\ \hline
\endfoot

\hline
\endlastfoot

Desain \& Impresi Web	&	3,3	&	4,1	&	20\%	(meningkat)\\ \hline
Kejelasan \& konsistensi sistem	&	3,5	&	4,2	&	17\% (meningkat)	\\ \hline
Kemudahan penggunaan	&	3,1	&	3,9	&	21\% (meningkat)	\\ \hline
Kejelasan status proses	&	3,7	&	3,9	&	5\% (meningkat)	\\ \hline
Error message yang jelas	&	3,3	&	4	&	18\% (meningkat)	\\ \hline
Performa	&	3,7	&	3,8	&	3\% (meningkat)	\\ \hline
Rating keseluruhan	&	3,7	&	4,3	&	14\% (meningkat)	\\ \hline
Rekomendasi aplikasi pada teman?	&	3,4	&	4,3	& 21\% (meningkat)	\\ \hline

\end{longtable}