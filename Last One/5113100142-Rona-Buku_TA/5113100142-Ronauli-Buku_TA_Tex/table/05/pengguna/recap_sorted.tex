\begin{longtable}{|X|c|}
	\caption{Rekapitulasi Perbedaan Hasil Pengujian Pengguna Aplikasi Lelang \textit{Online}} 
	\label{user-test-recap-sorted} 
	\\
	
	\hline\textbf{Parameter} & \textbf{\begin{tabular}[c]{@{}c@{}}Persentase \\ Perbedaan \end{tabular}} \\ \hline 
	\endfirsthead
	
	\multicolumn{2}{c}%
	{\tablename\ \thetable{} -- dilanjutkan dari halaman sebelumnya} \\
	\hline\textbf{Parameter} & \textbf{\begin{tabular}[c]{@{}c@{}}Persentase \\ Perbedaan \end{tabular}} \\ \hline 
	\endhead
	
	\hline \multicolumn{2}{|r|}{{dilanjutkan ke halaman setelahnya}} \\ \hline
	\endfoot
	
	\hline
	\endlastfoot
	
	Performa	&	3\% 	\\ \hline
	Kejelasan status proses	&	5\% 	\\ \hline
	\textit{Rating} keseluruhan	&	14\% 	\\ \hline
	Kejelasan \& konsistensi sistem	&	17\% 	\\ \hline
	\textit{Error message} yang jelas &	18\% 	\\ \hline
	Desain \& Impresi \textit{Web}	&	20\%	\\ \hline
	Kemudahan penggunaan	&	21\% 	\\ \hline
	Akan merekomendasikan aplikasi ini pada teman?	&	21\% 	\\ \hline
	
\end{longtable}