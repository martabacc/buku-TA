\begin{abstrak}
		E-commerce adalah kombinasi antara dunia digital dan transaksi lelang. Di Indonesia, seiring terjadi peningkatan jumlah pengguna internet dan menjamurnya bisnis \textit{online} atau sering disebut \textit{online shop}. Salah satu jenis transaksi adalah lelang, yaitu metode jual beli yang mengintegrasikan mekanisme lelang dengan internet.
	    \newline
	    \indent Dalam interaksi antara pelaku lelang \textit{online} (penjual dan pembeli) pasti terjadi kegagalan/ketidakpuasan dalam transaksi lelang \textit{online}. Berangkat dari penelitian "Online auction service failures in   {Taiwan}: {Typologies} and recovery strategies" yang membahas mengenai analisa kesalahan dan strategi lewat survey terhadap pengguna aplikasi lelang \textit{online} di Taiwan, aplikasi lelang \textit{online} ini dibangun disertai dengan tambahan fitur maupun saran dari penelitian tersebut.
	    \newline 
	    \indent Analisa aplikasi \textit{e-commerce} yang umum digunakan di Indonesia juga dilakukan, baik dari aspek \textit{user experience} maupun alur transaksi agar lebih sesuai dengan transaksi jual-beli \textit{online} yang umum di Indonesia. Dengan tugas akhir ini, diharapkan kegagalan dalam transaksi \textit{online} dapat diperbaiki dan membuka peluang lelang \textit{online} untuk meramaikan industri \textit{e-commerce} di Indonesia.\\
\noindent \textbf{Kata-Kunci}: \textit{lelang online}, \textit{strategi }
\end{abstrak}