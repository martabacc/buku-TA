% Melihat barang yang pernah didaftarkan

\begin{table}[H]
	\centering
	\caption{Spesifikasi Kasus Penggunaan: Melihat Barang yang Pernah Didaftarkan}
	\label{uc03.03}
	\begin{tabular}{|r|p{8cm}|}
		\hline
		\textbf{Kode}                                                    & UC-03.03                                                     \\ \hline
		\textbf{Nama}                                                    & \textbf{Melihat Daftar Barang yang Pernah Dilelang} \\ \hline
		\textbf{Aktor}                                                   & Pengguna 
		\\ \hline
		\textbf{Deskripsi}                                               & Pengguna hendak melihat daftar semua barang yang pernah didaftarkan untuk dilelang di dalam sistem.
		\\ \hline
		\textbf{Tipe}                                                    & Fungsional 
		\\ \hline
		\textbf{\textit{Precondition}}
		& Informasi daftar barang belum ditampilkan. \\ \hline
		\textbf{\textit{Postcondition}} 
		& Informasi daftar barang sudah ditampilkan. \\ \hline
		\multicolumn{2}{|c|}
		{\textbf{Alur Kejadian Normal}}                                                                            \\ \hline
		\multicolumn{1}{|l|}{}                                           & 
		\begin{enumerate}
			\item Pengguna dalam keadaan terautentikasi, mengklik "Item Anda" -> "Manage Items" pada \textit{navbar} bagian atas halaman.
			\item \label{uc0302-show1page}Sistem menampilkan halaman yang berisi daftar barang yang didaftarkan pengguna.
		\end{enumerate}
		\\ \hline
		\multicolumn{2}{|c|}{\textbf{Alur Kejadian Alternatif}}                                                         \\ \hline
		\multicolumn{1}{|l|}{}                                           & -
		\\ \hline
	\end{tabular}
\end{table}