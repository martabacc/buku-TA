
\begin{table}[H]
	\centering
	\caption{Spesifikasi Kasus Penggunaan: Mendaftarkan Barang Lelang}
	\label{uc03.01}
\begin{tabular}{|r|p{8cm}|}
		\hline
		\textbf{Kode}                                                    
			& UC-03.01                                                     
			\\ \hline
		\textbf{Nama}                                                    
			& \textbf{Mendaftarkan Barang Lelang} 
			\\ \hline
		\textbf{Aktor}                                                   
			& Pengguna 
			\\ \hline
		\textbf{Deskripsi}                                               
			& Pengguna mendaftarkan barang untuk dilelang di dalam sistem 
			 \\ \hline
		\textbf{Tipe}                                                    
			& Fungsional 
			\\ \hline
		\textbf{\textit{Precondition}}
			& Barang yang akan dilelang belum terdaftar dalam sistem 
			\\ \hline
		\textbf{\textit{Postcondition}} 
			& Barang yang akan dilelang sudah terdaftar dalam sistem 
			\\ \hline
		\multicolumn{2}{|c|}
			{\textbf{Alur Kejadian Normal}}                                                                            
			\\ \hline
		\multicolumn{1}{|l|}{}                                           & 
			\begin{enumerate}
				\item Pengguna dalam keadaan terautentikasi, mengklik "Item Anda" -> "Add Items" pada \textit{navbar} bagian atas halaman.
				\item \label{uc0301-show1page}Sistem menampilkan halaman yang berisi form pendaftaran barang
				\item Pengguna mengisi form tersebut sesuai data barang
				\item Setelah selesai mengisi, pengugna mengklik tombol "Daftar Barang"
				\item \label{al-0301-a} Sistem memvalidasi data yang dimasukkan pengguna
				\item Jika data valid, sistem me\textit{redirect} ke halaman "Kelola Barang" dalam keadaan barang baru sudah ditambahkan.
			\end{enumerate}
		\\ \hline
		\multicolumn{2}{|c|}{\textbf{Alur Kejadian Alternatif}}                                                         \\ \hline
		\multicolumn{1}{|l|}{}                                           & \textbf{Data barang yang dimasukkan pengguna tidak valid}
			\\ \hline
		\multicolumn{1}{|l|}{}                                           & 
			 \begin{itemize}
			 	\item[\ref{al-0301-a}a.] Sistem tidak dapat memvalidasi data yang dimasukkan pengguna.
			 	\item[\ref{al-0301-a}b.] Sistem me\textit{redirect} ke halaman form "Tambah Barang" (langkah \ref{uc0301-show1page}) dengan \textit{error message}.
			 \end{itemize}
		 \\ \hline
	\end{tabular}
\end{table}