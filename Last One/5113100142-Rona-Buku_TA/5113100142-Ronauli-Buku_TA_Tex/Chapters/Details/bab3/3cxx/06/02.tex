	% Memasukkan kupon
	
	\begin{table}[H]
		\centering
		\caption{Spesifikasi Kasus Penggunaan: Mendaftarkan Barang Lelang}
		\label{uc04.06}
		\begin{tabular}{|r|p{8cm}|}
			\hline
			\textbf{Kode}
			& UC-05.02
			\\ \hline
			\textbf{Nama}
			& \textbf{Memasukkan Kupon pada Transaksi} 
			\\ \hline
			\textbf{Aktor}    
			& Pengguna 
			\\ \hline
			\textbf{Deskripsi}
			& Pengguna ingin menggunakan kupon/voucher yang ia miliki untuk pada sebuah transaksi
			\\ \hline
			\textbf{Tipe}
			& Fungsional 
			\\ \hline
			\textbf{\textit{Precondition}}
			& Pengguna belum berhasil melakukan \textit{submit} kode kupon ke dalam transaksi barang
			\\ \hline
			\textbf{\textit{Postcondition}} 
			& Pengguna berhasil melakukan \textit{submit} kode kupon ke dalam transaksi barang
			\\ \hline
			\multicolumn{2}{|c|}
			{\textbf{Alur Kejadian Normal}}
			\\ \hline
			\multicolumn{1}{|l|}{} & 
			\begin{enumerate}
				\item Pengguna membuka halaman 'Riwayat Transaksi Lelang'
				\item Sistem menampilkan halaman Riwayat Transaksi Lelang pengguna
				\item Pengguna mengklik tombol 'Masukkan Kupon' pada transaksi yang diinginkan
				\item Sistem mengecek permintaan penggunaan kupon
				\item Jika permintaan dapat diverifikasi dan valid, sistem menampilkan \textit{modal} berisi \textit{input field} kupon
				\item Pengguna memasukkan kupon yang ingin dimasukkan, lalu mengklik tombol 'Submit'
				\item Sistem memvalidasi kupon dan status barang
				\item Jika valid, sistem menerapkan penggunaan kupon ke dalam transaksi barang
				\item Sistem lalu menampilkan \textit{modal} yang berisi informasi sukses penggunaan kupon pada transaksi
				% \item \label{uc0301-show1page}Sistem menampilkan halaman yang berisi form pendaftaran barang
				% \item \label{al-0301-a} Sistem memvalidasi data yang dimasukkan pengguna
			\end{enumerate}
			\\ \hline
			\multicolumn{2}{|c|}{\textbf{Alur Kejadian Alternatif}} \\ \hline
			\multicolumn{1}{|l|}{}                   
			& -
			\\ \hline
		\end{tabular}
	\end{table}