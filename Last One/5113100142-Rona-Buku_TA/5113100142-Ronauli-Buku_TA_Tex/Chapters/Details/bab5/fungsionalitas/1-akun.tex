\begin{longtable}{|X|X|}
		\caption{Pengujian Fungsionalitas Fitur Manajemen Akun}
		\label{uji-fungsional-1-akun}
	\\
	
	\hline
		\multicolumn{1}{|c|}{\textbf{ID}} & \multicolumn{1}{c|}{\textbf{UJ-1-KP01}} \\ \hline
	\endfirsthead
	
	\multicolumn{2}{r}%
	{\tablename\ \thetable{} -- lanjutan dari halaman sebelumnya} \\
	\hline \multicolumn{1}{|c|}{\textbf{\begin{tabular}[c]{@{}c@{}}ID \\ Kasus \\ Penggunaan\end{tabular}}} & \multicolumn{1}{c|}{\textbf{Kasus Penggunaan}} \\ \hline
	\endhead
	
	\hline \multicolumn{2}{|r|}{{Dilanjutkan ke halaman selanjutnya}} \\ \hline
	\endfoot
	
	\hline
	\endlastfoot
	
	\textbf{Referensi Kasus Penggunaan }
		& KP01 \\ \hline
	\textbf{Nama}
		& Pengujian fitur manajemen autentikasi \\ \hline
	\textbf{Skenario 1}
		& Menguji fitur registrasi \\ \hline
	Kondisi Awal
		& Sistem menampilkan halaman registrasi beserta form registrasi\\ \hline
	Data Uji
		& Data uji menggunakan data penulis \\ \hline
	Langkah pengujian
		& \begin{enumerate}
			\item Form registrasi diisi sesuai data uji
			\item Setelah selesai mengisi, mengklik tombol "Registrasi"
		\end{enumerate} \\ \hline
	Hasil yang Diharapkan
		& Sistem berhasil menyimpan data tersebut, \textit{email} konfirmasi terkirim ke \textit{mailbox} \textit{email} yang didaftarkan \\ \hline	
	Hasil Pengujian
		& 100\% berhasil \\ \hline	
	Kondisi Akhir
		& \textit{Screenshot} pengujian ini dapat dilihat pada Gambar \ref{ss-kp01-01} \\ \hline	

	\textbf{Skenario 2}
		& Menguji fitur login \\ \hline
	Kondisi Awal
		& Sistem menampilkan form login \\ \hline
	Data Uji
		& Data uji menggunakan data penulis yang sudah diregistrasi sebelumnya \\ \hline
	Langkah pengujian
		& \begin{enumerate}
		\item Form tersebut diisi sesuai data uji
		\item Setelah selesai mengisi, mengklik tombol "Login"
	\end{enumerate} \\ \hline
	Hasil yang Diharapkan
		& Login berhasil
	Hasil Pengujian
		& 100\% berhasil \\ \hline	
	Kondisi Akhir
		& \textit{Screenshot} pengujian ini dapat dilihat pada Gambar \ref{ss-kp01-02}  \\ \hline	
		
		
	\textbf{Skenario 3}
		& Menguji fitur konfirmasi \textit{email} \\ \hline
	Kondisi Awal
		& Pengguna sedang membuka \textit{email} dan mengklik link yang dikirimkan oleh sistem (noreply@noreply.lelangapa.com) \\ \hline
	Data Uji
		& Data uji menggunakan data penulis yang sudah diregistrasi sebelumnya \\ \hline
	Langkah pengujian
		& \begin{enumerate}
		\item Mengklik URL dari URL konfirmasi \textit{email} yang dikirimkan oleh sistem
	\end{enumerate} \\ \hline
	Hasil yang Diharapkan
		& \textit{email} berhasil dikonfirmasi
	Hasil Pengujian
		& 100\% berhasil \\ \hline	
	Kondisi Akhir
		& \textit{Screenshot} pengujian ini dapat dilihat pada Gambar \ref{ss-kp01-03}  \\ \hline	
\end{longtable}