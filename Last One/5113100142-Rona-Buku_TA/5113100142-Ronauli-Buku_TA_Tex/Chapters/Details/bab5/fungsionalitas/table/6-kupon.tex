\begin{longtable}{|X|X|}
		\caption{Pengujian Fungsionalitas Fitur Manajemen Kupon}
		\label{uji-fungsional-5-kupon}
	\\
	
	\hline
		\textbf{ID} & \textbf{UJ-1-KP05} \\ \hline
	\endfirsthead
	
	\multicolumn{2}{r}%
	{\tablename\ \thetable{} -- lanjutan dari halaman sebelumnya} \\
	\hline 
		\textbf{ID} & \textbf{UJ-1-KP05} \\ \hline
	\endhead
	
	\hline \multicolumn{2}{|r|}{{Dilanjutkan ke halaman selanjutnya}} \\ \hline
	\endfoot
	
	\hline
	\endlastfoot
	
	\textbf{Referensi Kasus Penggunaan}
		& KP05 \\ \hline
	\textbf{Nama}
		& Pengujian fitur manajemen kupon \\ \hline

		
	\textbf{Skenario 1}
		& Menguji fitur melihat daftar kupon \\ \hline
	Kondisi Awal
		& Sistem menampilkan halaman yang menampilkan halaman daftar kupon\\ \hline
	Data Uji
		& - \\ \hline
	Langkah pengujian
		& Membuka halaman profil pengguna yang ingin dilihat \textit{review}nya. \\ \hline
	Hasil yang Diharapkan
		& Sistem berhasil menampilkan profil pengguna yang ingin dilihat \textit{review}nya. \\ \hline	
	Hasil Pengujian
		& 100\% berhasil \\ \hline	

	\textbf{Skenario 2}
		& Menguji fitur menambah kupon \\ \hline
	Kondisi Awal
		& Sistem menampilkan halaman tambah kupon beserta formnya\\ \hline
	Data Uji
		&  Data-data kupon yang ingin ditambahkan\\ \hline
	Langkah pengujian
		& \begin{enumerate}
		\item Mengisi \textit{form} tambah kupon dengan informasi data uji
		\item Setelah selesai mengisi, mengklik tombol ``Tambah''
	\end{enumerate} \\ \hline
	Hasil yang Diharapkan 
		& Kupon berhasil ditambahkan, dan muncul di halaman daftar tambah kupon \\ \hline
	Hasil Pengujian
		& 100\% berhasil \\ \hline		
		
		
	\textbf{Skenario 3}
		& Menguji fitur memperbarui kupon \\ \hline
	Kondisi Awal
		& Sistem menampilkan halaman perbarui kupon\\ \hline
	Data Uji
		& Data ujinya adalah mengubah tanggal \textit{expiredate} kupon menjadi 1 minggu setelahnya \\ \hline
	Langkah pengujian
		& \begin{enumerate}
		\item Memasukkan data uji ke dalam \textit{form}
		\item Mengklik tombol ``Perbarui''
	\end{enumerate} \\ \hline
	Hasil yang Diharapkan
		& Pesan berhasil terkirim dan di\textit{update} di halaman yang sedang dibuka secara \textit{realtime} \\ \hline
	Hasil Pengujian
		& 100\% berhasil \\ \hline		
		
		
	\textbf{Skenario 4}
		& Menguji fitur menggunakan kupon \\ \hline
	Kondisi Awal
		& Pengguna membuka halaman riwayat penawaran\\ \hline
	Data Uji
		& Kupon ``SKRIPSIBAHAGIA'' yang mendiskon harga sebesar 20\% dan ``AKULULUS'' yang merupakan kupon \textit{freeshipping}  \\ \hline
	Langkah pengujian
		& \begin{enumerate}
		\item Memasukkan data uji ke dalam elemen \textit{input} nama kupon
		\item Mengklik tombol ``Gunakan kupon''
	\end{enumerate} \\ \hline
	Hasil yang Diharapkan
		& Kupon berhasil digunakan, dan muncul \textit{modal} yang menandakan kesuksesan penggunaan kupon \\ \hline
	Hasil Pengujian
		& 100\% berhasil \\ \hline	

\end{longtable}