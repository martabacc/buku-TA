\section{Evaluasi}
	Pada subbab ini, penulis akan memaparkan hasil analisa terhadap aplikasi, perspektif non-IT terhadap pengerjaan maupun lingkup pekerjaan dari aplikasi Lelang \textit{online} ini.
		
	\subsection{Pendekatan Hukum Perlindungan Konsumen}
	
	Sebagaimana lelang \textit{online} adalah salah satu jenis dari jenis transaksi jual-beli barang, tentu saja dalam pelaksanaannya diatur oleh undang-undang dan diawasi pemerintah, terutama untuk lelang harta-harta berharga seperti surat tanah, mobil, ijin usaha, dan lain-lain. Namun, penulis sadar banyak sekali kekurangan pengkajian hukum dan peraturan-peraturan penting seperti hak-hak dan kewajiban masing-masing pihak, \textit{rule} agar lelang dapat berjalan dan diawasi dengan baik, dan peraturan lainnya\\
	Untuk mempermudah pembahasan dan agar pemaparannya lebih kredibel (karena penulis tidak \textit{capable} dan kredibel untuk memaparkan hal ini), maka penulis mengutip sebuah kasus penipuan \textit{online} - dalam sebuah pertanyaan di platform konsultasi hukum online. \textit{Platform} \textit{online} ini - hukumonline.com. Pada platform ini, sering terdapat kajian kasus-kasus hukum dan menggunakan pendekatan hukum untuk penyelesaiannya. Dalam kasus ini, setelah berdiskusi dengan teman dan saudara yang mengambil spesialisasi hukum, penulis dengan hati-hati memaparkan sesuai dengan poin-poin yang persis ada dalam forum \textit{online} tersebut, dan tidak mengubah satupun kata agar kebenaran informasi yang disampaikan tidak berubah.\\
	\indent Pada keterangannya, pengkajian kasus ini menggunakan pendekatan utama pada \textbf{Undang-Undang Nomor 8 Tahun 1999 tentang Perlindungan Konsumen ("UU PK") } dan \textbf{Peraturan Pemerintah Nomor 82 Tahun 2012 tentang Penyelenggaraan Sistem dan Transaksi Elektronik ("PP PSTE") }. PP PSTE sendiri merupakan turunan dari \textbf{Undang-Undang Nomor 11 Tahun 2008 tentang Informasi dan Transaksi Elekronik ("UU ITE")}.
	\subsubsection{Perlindungan Hukum Bagi Konsumen Belanja \textit{Online}}
	Pemaparan subbab ini berupa pertanyaan yang dikaji dalam hukumonline.com, disajikan dalam bentuk pertanyaan dan jawaban.\\
\textbf{Pertanyaan}
\begin{displayquote}
	Saya pernah belanja barang secara online, tapi barang yang saya beli tidak sama dengan yang saya lihat di foto pada iklan yang dipajang. Pertanyaan saya, apakah itu termasuk pelanggaran hak konsumen? Apakah saya dapat menuntut penjual untuk mengembalikan uang atau mengganti barang yang saya beli tersebut? Terima kasih.
\end{displayquote}


\textbf{Jawaban oleh hukumonline.com: \\ Pendekatan Hukum Perlindungan Konsumen dalam Transaksi Jual Beli/Belanja secara \textit{Online}} \\ \ \\
\indent Dengan pendekatan UU Perlindungan Konsumen, kasus yang Anda sampaikan tersebut dapat kami simpulkan sebagai salah satu pelanggaran terhadap hak konsumen. \\

\indent Pasal 4 UU PK menyebutkan bahwa hak konsumen/dalam kasus ini adalah pengguna aplikasi lelang online adalah sebagai berikut:
\begin{enumerate}[label=\alph*.]
	\item hak atas kenyamanan, keamanan, dan keselamatan dalam mengkonsumsi barang dan/atau jasa;
	\item hak untuk memilih barang dan/atau jasa serta mendapatkan barang dan/atau jasa tersebut sesuai dengan nilai tukar dan kondisi serta jaminan yang dijanjikan;
	\item \textbf{hak atas informasi yang benar, jelas, dan jujur mengenai kondisi dan jaminan barang dan/atau jasa};
	\item hak untuk didengar pendapat dan keluhannya atas barang dan/atau jasa yang digunakan;
	\item \textbf{hak untuk mendapatkan advokasi, perlindungan, dan upaya penyelesaian sengketa perlindungan konsumen secara patut};
	\item hak untuk mendapat pembinaan dan pendidikan konsumen;
	\item hak unduk diperlakukan atau dilayani secara benar dan jujur serta tidak diskriminatif;
	\item \textbf{hak untuk mendapatkan kompensasi, ganti rugi dan/atau penggantian, apabila barang dan/atau jasa yang diterima tidak sesuai dengan perjanjian atau tidak sebagaimana mestinya};
	\item hak-hak yang diatur dalam ketentuan peraturan perundangundangan lainnya.
\end{enumerate}

\indent Di sisi lain, kewajiban bagi pelaku usaha (dalam hal ini adalah penjual online), sesuai Pasal 7 UU PK adalah:
\begin{enumerate}[label=\alph*.]
	\item beritikad baik dalam melakukan kegiatan usahanya;
	\item \textbf{memberikan informasi yang benar, jelas dan jujur mengenai kondisi dan jaminan barang dan/atau jasa serta memberi penjelasan penggunaan, perbaikan dan pemeliharaan};
	\item memperlakukan atau melayani konsumen secara benar dan jujur serta tidak diskriminatif;
	\item \textbf{menjamin mutu barang dan/atau jasa yang diproduksi dan/atau diperdagangkan berdasarkan ketentuan standar mutu barang dan/atau jasa yang berlaku};
	\item \textbf{memberi kesempatan kepada konsumen untuk menguji, dan/atau mencoba barang  dan/atau jasa tertentu serta memberi jaminan dan/atau garansi atas barang yang dibuat dan/atau yang diperdagangkan};
	\item \textbf{memberi kompensasi, ganti rugi dan/atau penggantian atas kerugian akibat penggunaan, pemakaian dan pemanfaatan barang dan/atau jasa yang diperdagangkan};
	\item \textbf{memberi kompensasi, ganti rugi dan/atau penggantian apabila barang dan/atau jasa yang diterima atau dimanfaatkan tidak sesuai dengan perjanjian}.
\end{enumerate}\ \\

\indent Selaku konsumen sesuai Pasal 4 huruf h UU PK tersebut, berhak mendapatkan kompensasi, ganti rugi dan/atau penggantian apabila barang dan/atau jasa yang diterima tidak sesuai dengan perjanjian atau tidak sebagaimana mestinya. Sedangkan, pelaku usaha itu sendiri sesuai Pasal 7 huruf (g) UU PK berkewajiban memberi kompensasi, ganti rugi dan/atau penggantian apabila barang dan/atau jasa yang diterima atau dimanfaatkan tidak sesuai dengan perjanjian.\\

\indent Apabila pelaku usaha (dalam hal ini, penulis karena penulis sebagai perantara antara pelelang dan pembeli) tidak melaksanakan kewajibannya, pelaku usaha dapat dipidana berdasarkan Pasal 62 UUPK, yang berbunyi:
\begin{displayquote}
	"Pelaku usaha yang melanggar ketentuan sebagaimana dimaksud dalam Pasal 8, Pasal 9, Pasal 10, Pasal 13 ayat (2), Pasal 15, Pasal 17 ayat (1) huruf a, huruf b, huruf c, huruf e, ayat (2) dan Pasal 18 dipidana dengan \textbf{pidana penjara paling lama 5 (lima) tahun} atau pidana denda paling banyak \textbf{Rp 2.000.000.000,00 (dua milyar rupiah)}".
\end{displayquote}

\subsubsection{Kontrak Elektronik dan Perlindungan Konsumen berdasarkan UU ITE dan PP PSTE}
Transaksi jual beli Anda, meskipun dilakukan secara online, berdasarkan UU ITE dan PP PSTE tetap diakui sebagai transaksi elektronik yang dapat dipertanggungjawabkan. Persetujuan konsumen untuk membeli barang secara \textit{online} dengan cara melakukan klik persetujuan atas transaksi merupakan bentuk tindakan penerimaan yang menyatakan persetujuan dalam kesepakatan pada transaksi elektronik. Tindakan penerimaan tersebut biasanya didahului pernyataan persetujuan atas syarat dan ketentuan jual beli secara online yang dapat kami katakan juga sebagai salah satu bentuk \textbf{Kontrak Elektronik}. Kontrak Elektronik menurut \textbf{Pasal 47 ayat (2) PP PSTE} dianggap sah apabila:
\begin{enumerate}[label=\alph*.]
	\item terdapat kesepakatan para pihak;
	dilakukan oleh subjek hukum yang cakap atau yang berwenang mewakili sesuai dengan ketentuan peraturan \item perundang-undangan;
	\item terdapat hal tertentu; dan
	\item objek transaksi tidak boleh bertentangan dengan peraturan perundang-undangan, kesusilaan, dan ketertiban umum.
\end{enumerate}
\ \\
\indent Kontrak Elektronik itu sendiri menurut \textbf{Pasal 48 ayat (3) PP PSTE} setidaknya harus memuat hal-hal sebagai berikut:
\begin{enumerate}[label=\alph*.]
	\item data identitas para pihak;
	\item objek dan spesifikasi;
	\item persyaratan Transaksi Elektronik;
	\item harga dan biaya;
	\item prosedur dalam hal terdapat pembatalan oleh para pihak;
	\item ketentuan yang memberikan hak kepada pihak yang dirugikan untuk dapat mengembalikan barang dan/atau meminta penggantian produk jika terdapat cacat tersembunyi; dan
	\item pilihan hukum penyelesaian Transaksi Elektronik.
\end{enumerate}
\ \\
\indent Terkait dengan perlindungan konsumen, \textbf{Pasal 49 ayat (1) PP PSTE menegaskan bahwa Pelaku Usaha yang menawarkan produk melalui Sistem Elektronik wajib menyediakan informasi yang lengkap dan benar berkaitan dengan syarat kontrak, produsen, dan produk yang ditawarkan.}

\subsubsection{\textit{Summary}}
	\indent Untuk mencapai hasil pengujian yang maksimal, seharusnya aplikasi lelang online ini melakukan tes pasar/tes pengguna, dimana aplikasi ini benar-benar \textit{dilempar} ke pasar, lewat iklan/\textit{endorsements}/\textit{digital advertising} lainnya. Penulis sudah mempersiapkan aplikasi untuk dapat melakukan tes pasar, yaitu \textit{hosting} aplikasi, pembelian domain sedemikian rupa agar aplikasi ini dapat diakses masyarakat luas dan benar-benar siap untuk tes pasar.\\
	\indent Ppenulis sadar bahwa dalam proses pengerjaan maupun saat \textit{requirement gathering}, sangat sedikit pertimbangan hukum (seperti peraturan lelang yang adil, proses pengawasan lelang yang sesuai dengan peraturan pemerintah, dll) dimasukkan dalam proses pembuatan aplikasi, sementara proses jual-beli itu sendiri proses yang vital dan hak kewajiban masing-masing pelaku (baik pembeli, penjual dan ) diatur dalam undang-undang. \\ 
	\indent Poin-poin yang dicetak tebal pada pemaparan pasal 4 dan pasal 7 UU PK adalah poin-poin yang tidak bisa penulis penuhi dalam posisi pelaku usaha dalam pengerjaan tugas akhir ini. Terkait dengan sanksi dan pidana yang menunggu jika penulis tetap menjalankan tes pasar tanpa mengikuti undang-undang yang ada, inkapabilitas penulis dalam sebagai penyelenggara usaha (lelang online), maka penulis memutuskan untuk tidak melakukan tes pasar.
	\\

	Dasar hukum yang digunakan (tetap dikutip dari hukumonline.org):
	\begin{enumerate}
		\item Kitab Undang-Undang Hukum Pidana
		\item Undang-Undang Nomor 8 Tahun 1999 tentang Perlindungan Konsumen
		\item Undang-Undang Nomor 11 Tahun 2008 tentang Informasi dan Transaksi Elekronik
		\item Peraturan Pemerintah Nomor 82 Tahun 2012 tentang Penyelenggaraan Sistem dan Transaksi Elektronik
	\end{enumerate}
			
	% \subsection{Evaluasi Pengujian Fungsionalita s}
			
	\subsubsection{Pengujian Fungsionalitas Manajemen Autentikasi}	
		\LTXtable{\textwidth}{Chapters/Details/bab5/fungsionalitas/table/1-akun}
	
	%insert gambar here ada 3 gambar
	%deskripsi singkat
%	\documentclass{ta-its}
\usepackage{hyperref}
\usepackage{cleveref}
\usepackage{multirow}
\usepackage{graphicx}
\usepackage{array}
\usepackage{multirow}
\usepackage{tabularx}
\usepackage{tabulary}
\usepackage{etoolbox}
\usepackage{listings}
\usepackage{longtable}
\usepackage{float}
\usepackage{slantsc}
\usepackage{booktabs}% http://ctan.org/pkg/booktabs


\newcolumntype{L}[1]{>{\raggedright\let\newline\\\arraybackslash\hspace{0pt}}m{#1}}
\newcolumntype{C}[1]{>{\centering\let\newline\\\arraybackslash\hspace{0pt}}m{#1}}
\newcolumntype{R}[1]{>{\raggedleft\let\newline\\\arraybackslash\hspace{0pt}}m{#1}}

\newcommand{\mychapter}[2]{
    \setcounter{chapter}{#1}
    \setcounter{section}{0}
    \chapter*{#2}
    \addcontentsline{toc}{chapter}{#2}
}


\title{Rancang Bangun Aplikasi \textit{web} Lelang \textit{Online} \textit{(E-Auction)} Berbasis Kerangka Kerja Laravel}{E-Auction Web Application Design and Implementation based on Laravel Framework}{KI141502} 

% \author{Nama Lengkap}{NRP}
\author{Ronauli Silva Natalensis Sidabukke}{5113100142}

% \supervisorOne{Nama Pembimbing Satu}{NIP}
% \supervisorTwo{Nama Pembimbing Dua}{NIP}
\supervisorOne{Rully Soelaiman, S.Kom, M.Kom}{197002131994021001}
\supervisorTwo{Rizky Januar Akbar, S.Kom., M.Eng}{198701032014041001}

% \degree{Nama Gelar}{Bidang Studi}{Program Studi}{Jurusan}{Jurusan (English)}{Fakultas}{Fakultas Singkatan}{Fakultas (English)}
\degree{Sarjana Komputer}{Algoritma Pemrograman}{S1}{Teknik Informatika}{Informatics}{Teknologi Informasi}{FTIf}{Information Technology}

% \time{bulan}{tahun}
\time{Juni}{2017}


\begin{document}
    \maketitle
    \pagenumbering{roman}
    \legalityPaper
    \begin{abstrak}
		E-commerce adalah kombinasi antara dunia digital dan transaksi lelang. Di Indonesia, seiring terjadi peningkatan jumlah pengguna internet dan menjamurnya bisnis online atau sering disebut \textit{online shop}. Salah satu jenis transaksi adalah lelang, yaitu metode jual beli yang mengintegrasikan mekanisme lelang dengan Internet.
	    \newline
	    \indent Dalam interaksi antara pelaku lelang online (penjual dan pembeli) pasti terjadi kegagalan/ketidakpuasan dalam transaksi lelang online.Berangkat dari paper "" yang membahas mengenai analisa kesalahan dan strategi lewat survey terhadap pengguna aplikasi lelang online di Taiwan, penulis membangun aplikasi lelang online yang disertai dengan tambahan fitur maupun saran dari paper tersebut.
	    \newline 
	    \indent Tidak hanya berdasarkan paper rujukan, penulis juga menganalisa aplikasi \textit{e-commerce} yang umum digunakan di Indonesia baik \textit{user experience} maupun alur transaksi, dan menambahkan beberapa fitur agar lebih sesuai dengan transaksi jual-beli online yang umum di Indonesia. Dengan aplikasi ini, diharapkan kegagalan dalam transaksi online dapat diperbaiki dan membuka peluang lelang online untuk meramaikan industri \textit{e-commerce} di Indonesia.\\
\noindent \textbf{Kata-Kunci}: \textit{lelang online}, \textit{strategi }
\end{abstrak}
    \begin{abstract}
	E-commerce industry is growing rapidly in Indonesia, along with the increasing number of internet users and number of online shops is also growing. One of e-commerce type is online auction, a buy and sell method that integrates auction mechanism and the Internet.\\
	\indent In the interaction between online auction actors (buyers and sellers), inevitable failure/dissatisfaction of online auction transactions sometimes found. Started by analysing paper about online auction application typologies and strategies through an application's users survey, author want to build online auction application along with additional ideas and suggestions from the paper.  \\
	\indent Author also analyzed and considering user experience, design and transaction flow local e-commerce platforms that are commonly used in Indonesia, in purpose to make the application suits Indonesian's users better. Furthermore, author hopes that this applications can reduce/prevent the expected failures in online transactions and open up online auction opportunity to enliven the e-commerce industry in Indonesia.\\
\noindent \textbf{Keyword}: \textit{online auction}, \textit{typologies and strategies}
\end{abstract}
    \mychapter{0}{KATA PENGANTAR}
%   \begin{figure}[h]
%     \centering
%     \includegraphics[width=0.5\linewidth]{images/bab0/gambarBismillah}
%   \end{figure}
  Puji Syukur kepada Tuhan yang Maha Esa, atas berkatNya penulis dapat menyelesaikan buku berjudul \textbf{\judul}. Dalam pengerjaan Tugas Akhir ini, penulis belajar banyak untuk memperdalam dan meningkatkan apa yang telah dipelajari penulis selama kuliah di Teknik Informatika ITS.
  Tugas Akhir ini terselesaikan tidak lepas dari bantuan dan dukungan banyak pihak. Oleh karena itu, pada kesempatan ini penulis mengucapkan banyak terima kasih kepada:
  \begin{enumerate}
  	\item \textbf{Daddy Jesus} - atas segala berkat, karunia, kesempatan dan rancangan jalanNya-lah penulis masih diberi nafas kehidupan, tenaga dan daya pikir untuk menyelesaikan buku ini. \textit{Thank you, Big Daddy.}
    \item \textbf{Papa dan Mama} yang selalu menguatkan, menasehati, dan luar biasa sabar dalam mengingatkan penulis agar tidak lupa menjaga kesehatan dan tidak lupa ke gereja selama masa studi.
    \item \textbf{Yth. Bapak Rully Soelaiman} yang memberi inspirasi kepada penulis untuk berpikir \textit{scientifically}, bimbingan, nasehat, saran dan memberikan penulis sisi pemikiran dan perspektif baru terhadap setiap masalah.
    \item \textbf{Yth. Bapak Rizky Januar Akbar} sebagai dosen pembimbing yang memberi bimbingan, saran teknis dan administratif, diskusi dan pemecahan masalah dalam pembuatan dan penulisan buku tugas akhir.
    \item \textbf{Keluarga XL Future Leader Scholarship Camp Batch 5} dan KSE ITS yang telah memberikan penulis kesadaran, semangat dan inspirasi untuk terus melanjutkan tugas akhir di saat penulis kehilangan semangat.
    \item \textbf{Keluarga Admin Lab. Pemrograman }(2014 - 2017) , yang telah memberikan penulis banyak pengalaman, pengetahuan dan cerita-cerita untuk dikenang.
    \item \textbf{Keluarga Alumni Budi Mulia Siantar-Surabaya angkatan 2013 } , teman setia disaat suka maupun duka.
    \item  \textbf{Keluarga Pengpro \textit{Furions} dan HMTC Optimasi 2016 }, yang mengajarkan penulis tentang cara organisasi, cara berbicara di depan publik, dan banyak lagi.    
    \item Serta semua pihak yang tidak tertulis - yang telah turut membantu penulis dalam menyelesaikan Tugas Akhir ini.
  \end{enumerate}
  Penulis menyadari bahwa Tugas Akhir ini masih memiliki banyak kekurangan. Oleh karena itu, penulis berharap kritik dan saran dari pembaca sekalian untuk memperbaiki buku ini ke depannya.


  \hfill Surabaya, Juni 2017 \\ \\ 


  \hfill Ronauli Silva N. Sidabukke

\cleardoublepage % Mengisi penanda halaman genap yang kosong


    \tableofcontents % Daftar isi
    \listoftables % Daftar tabel
    \listoffigures % Daftar gambar

  \mainmatter % Halaman utama, dengan judul BAB X
    \chapter{PENDAHULUAN}
  Pada bab ini akan dipaparkan mengenai garis besar Tugas Akhir yang meliputi latar belakang, tujuan, rumusan dan batasan permasalahan, metodologi pembuatan Tugas Akhir, dan sistematika penulisan.
  \section{Latar Belakang}
  	
	\indent Transaksi jual beli saat ini sudah dapat dilakukan lewat berbagai cara, antara lain menggunakan \textit{e-commerce}, atau lewat \textit{social media}, atau bisa dengan melelang di aplikasi lelang \textit{online}. Sedikit berbeda dengan teknik penjualan di lelang online, karena aplikasi ini dapat diakses oleh banyak orang, tentu saja pelelang (\textit{auctioneer}) tidak terbatas pada ruang lelang saja, tapi bisa berasal dari manapun selama mereka mengakses aplikasi tersebut.  Lelang \textit{online} ini tentu saja mendatangkan banyak manfaat, selain biaya yang lebih efisien dan hemat, dan juga tidak menguras waktu karena siapapun, kapanpun, dimanapun dapat mengajukan penawaran ataupun melelang barangnya tanpa harus pergi ke instansi tertentu dan melakukan lelang dengan cara konvensional.
    \\
    \indent Aplikasi serupa telah banyak, namun banyak aspek yang kurang dalam aplikasi tersebut, seperti informasi dari lelang tidak \textit{reliable} (misal: stok barang ternyata sudah habis), alur proses yang tidak jelas sehingga membingungkan pengguna aplikasi, informasi yang kurang jelas, dan produk yang didapatkan ternyata tidak sesuai dengan informasi pada saat produk dilelang (\textit{bad information}) \cite{ying-feng_kuo_online_2016}.
    \\
    \indent Dan dari masalah teknis aplikasi, beberapa sumber menyatakan bahwa ketidakjelasan alur proses yang kurang diperhatikan oleh para developer aplikasi lelang \textit{online} menjadi beberapa alasan yang kuat mengapa lelang online masih kurang diminati \cite{noauthor_sistem_nodate}.
    \\
	\indent Diharapkan, dengan adanya aplikasi ini, beberapa kelemahan yang masih ada pada aplikasi lelang \textit{online} saat ini dapat diperbaiki, dan juga dapat dapat membantu proses \textit{online} yang ada di Indonesia, dan juga mampu memperbaiki citra aplikasi lelang \textit{online} sehingga mampu meningkatkan minat masyarakat terhadap lelang \textit{online}.
    
  \section{Rumusan Masalah}
    Rumusan masalah yang diangkat dalam tugas akhir ini adalah sebagai berikut: 
    \begin{enumerate}
      \item Bagaimana membangun aplikasi lelang online berbasis web?
      \item Bagaimana rancangan arsitektur aplikasi dan fitur yang menganalisa kelemahan aplikasi serupa dan strategi penyelesaian sesuai dengan paper acuan \cite{ying-feng_kuo_online_2016}?
    \end{enumerate}

  \section{Batasan Masalah}
  	\label{batasan-masalah}
    Dari permasalahan yang telah diuraikan di atas, terdapat beberapa batasan masalah pada tugas akhir ini, yaitu:
    \begin{enumerate}
      \item Aplikasi berbasis web dengan bahasa pemrograman PHP.
      \item Aplikasi berbasis kerangka kerja Laravel.
      \item Basis data yang digunakan adalah PostgreSQL.
      \item Aplikasi tidak mencakup proses pembayaran.
    \end{enumerate}

  \section{Tujuan}
  \label{tujuan}
    Tujuan dari pengerjaan Tugas Akhir ini adalah: 
    \begin{enumerate}
      \item Membangun aplikasi lelang online berbasis web yang lebih kredibel sesuai dengan paper yang dijadikan acuan pada tugas akhir ini. 
    \end{enumerate}
    \chapter{LANDASAN TEORI}{}
  \section{Lelang Daring / Lelang \textit{Online}}
   Lelang adalah proses membeli dan menjual barang atau jasa dengan cara menawarkan kepada penawar, menawarkan tawaran harga lebih tinggi, dan kemudian menjual barang kepada penawar harga tertinggi. Dalam teori ekonomi, lelang mengacu pada beberapa mekanisme atau peraturan perdagangan dari pasar modal. \\
	Sementara lelang daring atau lelang melalui internet muncul seiring dengan perkembangan internet. Barang atau jasa yang diperjualbelikan dipasang di situs dan peserta lelang dapat mengikuti acara lelang secara daring. Perusahaan lelang yang berhasil menggunakan sarana internet salah satunya adalah \textit{Ebay} . Di Indonesia, lelang melalui internet (online) sudah dipelopori oleh pemerintah dengan situs lelang online yang dapat diakses melalui website resmi \href{https://www.lelangdjkn.kemenkeu.go.id}{Kemenkeu} \cite{wikipedia_lelang_2016} . 
    Berikut adalah beberapa istilah yang ada dalam lelang online :
    \begin{enumerate}
	\item BID atau \textit{Bidding}, artinya : Menawarkan
    \item BIN (\textit{Buy In Now}) artinya : Beli sesuai harga yang telah ditawarkan penjual
    \item INC (\textit{Increment}) artinya : Minimum kenaikan \textit{bid} setelah \textit{bid} sebelum nya \cite{noauthor_arti_nodate}
    \end{enumerate}
    
    \section{PostgreSQL}
    PostgreSQL adalah sebuah produk \textit{database} relasional yang termasuk dalam kategori \textit{free open source software} (\textit{FOSS}). 
	PostgreSQL terkenal karena fitur-fitur yang advanced dan pendekatan rancangan modelnya menggunakan paradigma \textit{object-oriented}, sehingga sering dikategorikan sebagai \textit{Object Relational Database Management System} (ORDBMS).
    Beberapa fitur PostgreSQL adalah sebagai berikut :
    \begin{enumerate}
    \item \textit{Inheritance}, dimana satu table dapat diturunkan model dan beberapa karakteristik dari table lainnya.
    \item \textit{Multi-Version Concurrency Control} dimana user diberi data snapshot ketika suatu perubahan dilakukan sampai commit.
    \item \textit{Rules} , dimana suatu \textit{query} DML yang dikirimkan ke server akan mengalami penulisan ulang (\textit{rewrite}). Ini terjadi sebelum diproses oleh \textit{query planner}.
    \item dan berbagai fitur lainnya \cite{noauthor_postgresql_nodate}
    \end{enumerate}
    
  \section{Redis}
    Redis adalah \textit{open source}, struktur data yang ditempatkan di memori, digunakan sebagai \textit{database}, \textit{cache} dan \textit{message broker}. Redis mendukung struktur data seperti \textit{string, sets, hash, lists} dan \textit{sorted sets}. Sama seperti cache, setiap key diisi oleh value. Tapi kelebihannya, Redis bisa digunakan untuk melakukan operasi dari value tersebut. Cara terbaik untuk memahami redis adalah membuat model aplikasi tanpa memikirkan bagaimana caranya untuk menyimpan data di dalam \textit{database} \cite{yudana_redis_2015}.

  \section{Node.js}
  Node.js adalah platform perangkat lunak pada sisi-server dan aplikasi jaringan. Ditulis dengan bahasa javascript dan bisa dijalankan pada Windows, Mac OS X dan Linux tanpa perubahan kode program. Node.js memiliki pustaka server HTTP sendiri sehingga memungkinkan untuk menjalankan webserver tanpa menggunakan program webserver seperti Apache atau Lighttpd \cite{noauthor_node.js_2014}.
  
  \section{Socket.io}
	Socket.io adalah \textit{library} Javascript untuk aplikasi web yang bersifat \textit{realtime}. Socket.io menjembatani antara komunikasi dua arah antara \textit{web} \textit{clients} dan \textit{server}. Socket.io terbagi menjadi dua bagian, yaitu \textit{client}-\textit{side} \textit{library} yang berjalan di browser client, dan \textit{server}-\textit{side} \textit{library} yang menggunakan Node.js. Kedua komponen tersebut mempunyai API yang sama. Seperti Node.js, Socket.io juga bersifat \textit{event}-\textit{driven}. Socket.IO menggunakan protokol \textit{websocket} dengan \textit{polling} sebagai opsi \textit{fallback}. Meskipun Socket.IO merupakan ‘pembungkus’ untuk soket web, namun ia memiliki banyak fitur, antara lain broadcast ke banyak soket, dan I/O yang asinkronus \cite{noauthor_socket.io_2016}.
    
    \section{Laravel}
    Laravel adalah \textit{framework} PHP yang dikembangkan pertama kali oleh Taylor Otwell. Walaupun termasuk baru, namun komunitas pengguna laravel sudah berkembang pesat dan mampu menjadi alternatif utama dari sejumlah \textit{framework} besar seperti CodeIgniter dan Yii. Laravel oleh para \textit{developer} disetarakan dengan CodeIgniter dan FuelPHP namun memiliki keunikan tersendiri dari sisi \textit{coding}. Laravel memiliki beberapa keunggulan, diantaranya :
\begin{enumerate}
\item Sintaks yang sederhana dan \textit{programmer}-\textit{fiendly}
\item Tersedia \textit{generator} yang canggih dan memudahkan, Artisan CLI
\item Fitur \textit{Schema} \textit{Builder} untuk berbagai \textit{database}
\item Fitur \textit{Migration} dan \textit{Seeding} untuk berbagai \textit{database}
\item Fitur \textit{Query} \textit{Builder} yang powerful
\item Eloquent ORM (\textit{Object} \textit{Relational} \textit{Mapping})
\item Fitur pembuatan \textit{package} dan \textit{bundle}
\item \textit{Dependency} \textit{Injection} \cite{a}
\end{enumerate}

	\section{Protokol SMTP}
    
	\section{JSON Web Token}
    
   \section{Service Worker}
   
   \section{Repository Pattern}
   
   \section{Concurrency}
   
   \section{Transaction Isolation}
   
   \section{Script Testing}
   
   \section{Laravel Dusk}
   
   


    \chapter{ANALISA DAN PERANCANGAN}  
  \input{Chapters/Details/bab3/3a-Analisa}
  
  \pagebreak
  \input{Chapters/Details/bab3/3c-Rancangan}
  
  
    w\chapter{IMPLEMENTASI}
  Pada bab ini dibahas mengenai implementasi aplikasi sesuai dengan perancangan sistem yang telah dijelaskan sebelumnya. Bahasa pemrograman yang digunakan antara lain PHP, SQL, Javascript.
  
  \section{Lingkungan Implementasi}
  Lingkungan pembangunan dijelaskan pada subbab ini.
  
  
  \subsection{Lingkungan Pembangun Perangkat Keras}
  
  Aplikasi dideploy secara \textit{online}, dalam sebuah \textit{Virtual Private Server} yang di\textit{host} oleh \textit{Digital Ocean}.
  Spesifikasi VPS yang digunakan adalah sebagai berikut :
  
  \begin{enumerate}
  	\item Hardware
		  \begin{enumerate}
		  	\item CPU: Intel(R) Xeon(R) CPU E5-2630L v2 @ 2.40GHz
		  	\item Operating System : 
		  	\item RAM : 512MB
		  	\item Storage Space : 20GB
		  \end{enumerate}
		  
   \item Operating System
	   \begin{enumerate}
	   	\item Architecture : 64bit
	   	\item Kernel Version : Linux 4.4.0-75-generic x86 64
	   	\item OS Version : Ubuntu 16.04.2 LTS Xenial
	   	\end{enumerate}
	   	
	\item Networking Stats
		\begin{enumerate}
			\item Tersambung ke Internet : Ya
			\item IP Publik : Ya
			\item Alamat IP Publik (IPv4) : 188.166.179.2
			\item \textit{Average Download Speed} : 1371 Mbit/s
			\item \textit{Average Upload Speed} : 860.12 Mbit/s
			\item DNS : Google
		\end{enumerate}
		
	\item Domain Stats
		\begin{enumerate}
			\item HTTPS Support : Yes
			\item SSL Certificate issued by : Avast
			\item Domain : https://Lelangapa.com
			\item Testing-purpose subdomain : https://testing.lelangapa.com
			\item Domain issued by : Namecheap
		\end{enumerate}
	
  \end{enumerate}
  
  \subsection{Lingkungan Pembangun Perangkat Lunak}
  Spesifikasi perangkat lunak yang digunakan untuk membuat tugas akhir ini adalah sebagai berikut:
	  \begin{enumerate}
	  \item Google Chrome sebagai media akses aplikasi
	  \item PgAdmin, sebagai Database Management \& Editor
	  \item PHPStorm sebagai IDE utama
	  \item Nano untuk \textit{shell text editor}
	  \item Postman, untuk \textit{debugging} API \textit{calls} dan system tests
	  \item Power Designer untuk alat bantu desain yang berhubungan dengan grafis seperti diagram, \textit{flowchart}, dll.
	  \end{enumerate}
  
\section{Implementasi Antarmuka}
	
    \subsection{Antarmuka Registrasi}
    
    Penjelasan otorisasi terhadap antarmuka A, link yang tersedia dalam antarmuka A, dan penjelasan \textit{exception} jika terjadi masalah baik otorisasi ataupun autentikasi saat mengakses antarmuka ini.
  
      \begin{figure}[H]
        \centering
        \includegraphics[width=\linewidth]{images/bab4/smpso_code.png}
        \caption{ Pseudocode Controller untuk Menampilkan Antarmuka A }
        \label{pdm}
      \end{figure}
      
    \subsection{Antarmuka Halaman B}
    Penjelasan otorisasi terhadap antarmuka B, link yang tersedia dalam antarmuka B, dan penjelasan \textit{exception} jika terjadi masalah baik otorisasi ataupun autentikasi saat mengakses antarmuka ini.
  
      \begin{figure}[H]
        \centering
        \includegraphics[width=\linewidth]{images/bab4/smpso_code.png}
        \caption{ Pseudocode Controller untuk Menampilkan Antarmuka B }
        \label{pdm}
      \end{figure}
    
    
\section{Pemasangan Proyek}
	Pembangunan dilakukan secara online, dan tersebar (tidak hanya menggunakan satu \textit{service provider} saja.
    Berikut dijelaskan langkah-langkah pembangunan proyek:
    
    \subsection{Konfigurasi Domain}
    Domain yang dipilih berasal dari Namecheap.com , dengan langkah-langkah konfigurasi sebagai berikut :
    \begin{enumerate}
    \item Langkah 1
    \item Langkah 2
    \end{enumerate}
    \subsection{Konfigurasi VPS}
    Domain yang dipilih berasal dari DigitalOcean dan Google Cloud Computing , dengan langkah-langkah konfigurasi sebagai berikut :
    \begin{enumerate}
    \item DigitalOcean
      \begin{enumerate}
      \item Langkah 1
      \item Langkah 2
      \end{enumerate}
    \item Google Cloud Computing
      \begin{enumerate}
      \item Langkah 1
      \item Langkah 2
      \end{enumerate}
    \end{enumerate}
    
    \subsection{Konfigurasi PostgreSQL}
    PostgreSQL diinstal dalam VPS, dengan langkah-langkah konfigurasi sebagai berikut :
    \begin{enumerate}
    \item Langkah 1
    \item Langkah 2
    \end{enumerate}
    
    \subsection{Konfigurasi Node.js}
    Node.js diinstall dalam VPS, dengan langkah-langkah konfigurasi sebagai berikut :
    \begin{enumerate}
    \item Langkah 1
    \item Langkah 2
    \end{enumerate}
    
    \subsection{Konfigurasi MongoDB}
    MongoDB diinstall dalam VPS, dengan langkah-langkah konfigurasi sebagai berikut :
    \begin{enumerate}
    \item Langkah 1
    \item Langkah 2
    \end{enumerate}
    
    \subsection{Konfigurasi SMTP Service}
    SMTP \textit{service} yang digunakan berasal dari sendgrid.net, dengan langkah-langkah konfigurasi sebagai berikut :
    \begin{enumerate}
    \item Langkah 1
    \item Langkah 2
    \end{enumerate}

    	\chapter{PENGUJIAN DAN EVALUASI}
	
  \input{Chapters/Details/bab5/5a-Pengujian}

  \input{Chapters/Details/bab5/5b-Evaluasi}


    %\section{Struktur Dokumen \LaTeX{}}
Dokumen \LaTeX{} terdiri dari struktur yang dibuat berdasarkan struktur dokumen sehari-hari. Sebagai penulis dokumen, Anda wajib menggunakan struktur ini sehingga \LaTeX{} dapat melakukan hal lain yang membantu Anda dalam mengorganisir dokumen seperti misalnya pembuatan Daftar Isi. Berikut adalah struktur dokumen yang ada di \LaTeX{} diurutkan berdasarkan hirarkinya.

\begin{ltabulary}{|L|L|} % L = Rata kiri untuk setiap kolom, | = garis batas vertikal.

% Kepala tabel, berulang di setiap halaman
\caption{Struktur hirarki dokumen \LaTeX{}} \label{tabelStrukturDokumen} \\
\hline
\textbf{Nama} & \textbf{Peruntukkan} \\ \hline

\endhead
\endfoot
\endlastfoot

% Isi Tabel
\textbf{\textbackslash{}part\{Judul Bagian\}} & \texttt{book} \\ \hline
\textbf{\textbackslash{}chapter\{Judul Bab\}} & \texttt{book} dan \texttt{report} \\ \hline
\textbf{\textbackslash{}section\{Judul Subbab\}} & semua kecuali \texttt{letter} \\ \hline
\textbf{\textbackslash{}subsection\{Judul Subsubbab\}} & semua kecuali \texttt{letter} \\ \hline
\textbf{\textbackslash{}subsubsection\{Judul Subsubsubbab\}} & semua kecuali \texttt{letter} \\ \hline
\textbf{\textbackslash{}paragraph\{Judul Paragraf\}} & semua\\ \hline

\end{ltabulary}

\subsection{Pengujian Performa}
      Pengujian 
      \subsubsection{Pengujian Kecepatan Fitur A}
      Pengujian fitur ini dilakukan pada lingkungan uji \ref{env_uji1}, dan untuk lebih lengkapnya dapat dilihat pada Tabel \ref{uji1}
      \begin{table}[]
      \centering
      \caption{Pengujian Fitur B}
      \label{uji2}
      \begin{tabular}{llll}
      \multicolumn{1}{c}{\textbf{ID}} & \multicolumn{3}{c}{\textbf{TA-UJI.Proses}}        \\
      Referensi Proses Penggunaan     & \multicolumn{3}{l}{}                              \\
      Nama                            & \multicolumn{3}{l}{}                              \\
      Tujuan Pengujian                & \multicolumn{3}{l}{\multirow{2}{*}{}}             \\
      \textbf{Skenario Pengujian}     & \multicolumn{3}{l}{}                              \\
      Langkah Pengujian               & \multicolumn{3}{l}{}                              \\
      Kecepatan Buka Halaman          & Halaman       & \multicolumn{2}{l}{Google Chrome} \\
                                      & A             & 18 KB           & 0.987s         
      \end{tabular}
      \end{table}
      
    \chapter{PENUTUP}
  Bab ini membahas kesimpulan yang dapat diambil dari tujuan pembuatan sistem dan hubungannya dengan hasil uji coba dan evaluasi yang telah dilakukan. Selain itu, terdapat beberapa saran yang bisa dijadikan acuan untuk melakukan pengembangan dan penelitian lebih lanjut.
  \section{Kesimpulan}
  Dari proses perancangan, implementasi dan pengujian terhadap sistem, dapat diambil beberapa kesimpulan berikut:
  \begin{enumerate}
    \item Kesimpulan 1
    \item Kesimpulan 2
    \item Kesimpulan 3
  \end{enumerate}
  
  \section{Saran}
  Berikut beberapa saran yang diberikan untuk pengembangan lebih lanjut:
  \begin{itemize}
  	\item Menggunakan mekanisme \textit{Queue} sebagai \textit{countermeasure} dari masalah \textit{occurence} (di Laravel sudah ada disediakan \textit{base class}nya sendiri).
    \item Mengikutsertakan pihak yang menguasai/spesialisasi di bidang hukum untuk menetapkan peraturan-peraturan terkait
    \item Saran 2 
    \item Saran 3
  \end{itemize}

    \appendix % Halaman lampiran, dengan judul LAMPIRAN X
  \backmatter % Lampiran tanpa judul LAMPIRAN X, untuk BIODATA PENULIS
\end{document}
		
	\subsubsection{Pengujian Fungsionalitas Manajemen Penawaran}
	\begin{longtable}{|X|X|}
		\caption{Pengujian Fungsionalitas Fitur Manajemen }
		\label{uji-fungsional-1-akun}
	\\
	
	\hline
		\multicolumn{1}{|c|}{\textbf{ID}} & \multicolumn{1}{c|}{\textbf{UJ-1-KP02}} \\ \hline
	\endfirsthead
	
	\multicolumn{2}{r}%
	{\tablename\ \thetable{} -- lanjutan dari halaman sebelumnya} \\
	\hline 
		\multicolumn{1}{|c|}{\textbf{ID}} & \multicolumn{1}{c|}{\textbf{UJ-1-KP02}} \\ \hline
	\endhead
	
	\hline \multicolumn{2}{|r|}{{Dilanjutkan ke halaman selanjutnya}} \\ \hline
	\endfoot
	
	\hline
	\endlastfoot
	
	\textbf{Referensi Kasus Penggunaan}
		& KP02 \\ \hline
	\textbf{Nama}
		& Pengujian fitur manajemen penawaran \\ \hline
	\textbf{Skenario 1}
		& Menguji fitur melihat barang yang sedang aktif dilelang \\ \hline
	Kondisi Awal
		& Sistem \\ \hline
	Data Uji
		& Data uji menggunakan data penulis \\ \hline
	Langkah pengujian
		& Membuka halaman website aplikasi lelang online via \textit{browser} di alamat https://lelangapa.com \\ \hline
	Hasil yang Diharapkan
		& Sistem berhasil menampilkan data barang yang sedang aktif dilelang \\ \hline	
	Hasil Pengujian
		& 100\% berhasil \\ \hline	
	Kondisi Akhir
		& \textit{Screenshot} pengujian ini dapat dilihat pada gambar \ref{ss-kp02-01} \\ \hline	

	\textbf{Skenario 2}
		& Menguji fitur mencari barang \\ \hline
	Kondisi Awal
		& Sistem menampilkan halaman dengan elemen \textit{input} search barang \\ \hline
	Data Uji
		&  \\ \hline
	Langkah pengujian
		& \begin{enumerate}
		\item Elemen \textit{input search} diisi dengan kata kunci ``Jersey''
		\item Setelah selesai mengisi, mengklik tombol ``Search''
	\end{enumerate} \\ \hline
	Hasil yang Diharapkan
		& Barang yang mengandung kata ``Jersey'' muncul dalam hasil pencarian
	Hasil Pengujian
		& 100\% berhasil \\ \hline	
	Kondisi Akhir
		& \textit{Screenshot} pengujian ini dapat dilihat pada gambar \ref{ss-kp02-02}  \\ \hline	
		
		
	\textbf{Skenario 3}
		& Menawar barang \\ \hline
	Kondisi Awal
		& Pengguna sedang membuka halaman lelang barang \\ \hline
	Data Uji
		& Data ujinya adalah memasukkan penawaran harga yang lebih tinggi pada penawaran harga saat itu \\ \hline
	Langkah pengujian
		& \begin{enumerate}
		\item Memasukkan harga penawaran
		\item Mengklik tombol ``Tawar''
	\end{enumerate} \\ \hline
	Hasil yang Diharapkan
		& Penawaran yang baru berhasil masuk ke dalam sistem, dan harga terbaru diupdate di halaman secara \textit{realtime}
	Hasil Pengujian
		& 100\% berhasil \\ \hline	
	Kondisi Akhir
		& \textit{Screenshot} pengujian ini dapat dilihat pada gambar \ref{ss-kp02-03}  \\ \hline	
		
\end{longtable}
		
	\subsubsection{Pengujian Fungsionalitas Manajemen Barang Lelang}
		\LTXtable{\textwidth}{Chapters/Details/bab5/fungsionalitas/table/3-barang}
	
	%insert gambar here ada 3 gambar
	%deskripsi singkat
			
	\subsubsection{Pengujian Fungsionalitas Manajemen Interaksi Antarpengguna}
			\LTXtable{\textwidth}{Chapters/Details/bab5/fungsionalitas/table/4-interaksi}
		
		%insert gambar here ada 3 gambar
		%deskripsi singkat


	
	\subsubsection{Pengujian Fungsionalitas Manajemen Laporan}
			\LTXtable{\textwidth}{Chapters/Details/bab5/fungsionalitas/table/5-laporan}
		
		%insert gambar here ada 3 gambar
		%deskripsi singkat


	
	\subsubsection{Pengujian Fungsionalitas Manajemen Kupon}
			\LTXtable{\textwidth}{Chapters/Details/bab5/fungsionalitas/table/6-kupon}
		
		%insert gambar here ada 3 gambar
		%deskripsi singkat


	
			

	
	\subsection{Evaluasi Pengujian \textit{User Experience}}
	Rekapitulasi pengujian pengguna/\textit{user test} adalah sebagai berikut:

\indent Rekapitulasi tersebut diurutkan berdasarkan besarnya/signifikansi persentase perbedaan antara kedua platform, dapat dilihat pada tabel \ref{user-test-recap-sorted}.
\LTXtable{\textwidth}{table/05/pengguna/recap_sorted}

Visualisasi perbandingan dapat dilihat lewat diagram garis pada gambar \ref{diagram-pengguna-chart}.

\begin{figure}[H]
	\centering
	\includegraphics[width=\textwidth]{images/bab5/ujipengguna/chart.png}
	\caption{Diagram perbandingan pengujian \textit{user experience} pengguna}
	\label{diagram-pengguna-chart}
\end{figure}

Dalam hal ini, dapat disimpulkan bahwa impresi \textit{user experience} sudah baik, dan skor \textit{user experience}nya sedikit diatas sistem serupa lainnya. Perbedaannya sudah cukup signifikan adalah \textit{recommendation}, kemudahan penggunaan dan desain web yang baik. Namun, yang menjadi perhatian adalah perbedaan \textit{performa} yang masih sangat kecil. Hal ini terkait dengan pengujian \textit{speed test} di subbab selanjutnya.
	\subsection{Evaluasi Pengujian Kecepatan}
	Rekapitulasi pengujian kecepatan adalah sebagai berikut:

Hasil rata-rata dari kecepatan \textit{loading} dan \textit{request} terhadap sistem adalah sebagai berikut, sesuai dengan segmentasi yang telah dipaparkan pada subbab Pengujian Kecepatan:
\begin{enumerate}
	\item \textit{DOM Loading}: 104,2 ms
	\item \textit{Scripting}: 914,6 ms
	\item \textit{Rendering}: 313,7 ms
\end{enumerate}

Visualisasi perbandingan dan komparasi antara ketiga segmentasi tersebut dapat dilihat lewat diagram batang pada Gambar \ref{diagram-pengguna-chart}.

\begin{figure}[H]
	\centering
	\includegraphics[width=\textwidth]{images/bab5/speed/bar-chart.png}
	\caption{Diagram batang hasil pengujian kecepatan sistem }
	\label{chart-speed-test}
\end{figure}

\begin{figure}[H]
	\centering
	\includegraphics[width=\textwidth]{images/bab5/speed/circle-chart.png}
	\caption{Diagram lingkaran hasil pengujian kecepatan}
	\label{circle-chart-speed-test}
\end{figure}

Dari angka tersebut, dapat dilihat bahwa proses \textit{scripting} sangat memakan waktu, yang berarti konten dan \textit{assets} yang di\textit{load} dalam page cukup besar. Tampak dari waktu \textit{loading page} halaman menampilkan daftar barang mencapai waktu \textit{loading} terlama, karena pada halaman tersebut terdapat sangat banyak gambar berukuran besar dan adanya beberapa \textit{script} dibutuhkan yang kurang efisien (butuh \textit{owl-carousel}, \textit{bootstrap} untuk me\textit{load} setiap halaman). \\
	\subsection{Evaluasi Pengujian \textit{Maintainability}}
	Sesuai dengan daftar pengujian yang telah dilakukan pada subbab Pengujian \textit{Maintainability}, maka dapat dievaluasi sesuai dengan tabel \ref{maintainability-evaluation}.

\LTXtable{\textwidth}{table/05/maintainability/recap_sorted}

Visualisasi perbandingan dapat dilihat lewat diagram garis pada gambar \ref{diagram-pengguna-chart}.

\begin{figure}[H]
	\centering
	\includegraphics[width=\textwidth]{images/bab5/maintainability/maintainability-evaluation.png}
	\caption{Diagram evaluasi pengujian \textit{maintainability}}
	\label{maintainability-diagram}
\end{figure}

Dengan total skor 80\% yang tepat sama dengan \textit{evaluator goal} yang disebutkan pada subbab Deskripsi Pengujian \textit{Maintainability}, maka dapat dikatakan bahwa \textit{maintainability} pada sistem ini sudah baik.
	\subsection{\textit{Summary} Evaluasi}
	Summary dari evaluasi pengujian dapat dilihat pada tabel \ref{summary-pengujian}

\LTXtable{\textwidth}{table/05/evaluation_all_summarized}
	

	