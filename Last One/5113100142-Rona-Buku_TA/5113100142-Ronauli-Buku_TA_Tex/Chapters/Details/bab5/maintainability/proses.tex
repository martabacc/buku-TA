Pengujian dilakukan dengan menggunakan kriteria yang mengacu kepada paper \textit{A Software Maintainability Evaluation Methodology}\cite{peercy_software_nodate}. Paper ini menyebutkan bahwa ada 2 kriteria dasar penilaian \textit{maintainability} dengan \textit{weight} berbeda, yaitu dokumentasi (\textit{weight: 40\%}) dan \textit{source code}(\textit{weight: 60\%}). Menurut paper tersebut, terdapat 5 faktor utama \textit{maintainability}, yaitu sebagai berikut:
\begin{enumerate}
	\item \textit{modularity};
	\item \textit{descriptiveness};
	\item \textit{consistency};
	\item \textit{simplicity}; dan
	\item \textit{trackability}
\end{enumerate}

\ \indent Pengujian ini dilakukan dengan cara menyebar \textit{form online} dengan menggunakan media Google Form. Responden yang disasar adalah responden dengan latar belakang \textit{software engineer} dengan tujuan agar responden dapat membandingkan pengalaman responden tersebut dengan kualitas \textit{maintainability} tugas akhir ini.