\subsection{Memperbarui Kupon}
Halaman ini hanya dapat diakses oleh \textit{administrator} yang sebelumnya sudah \textit{login}. Detail kasus penggunaan dapat dilihat Tabel \ref{uc06.03}.\\
\indent Tidak ada \textit{view logic} ataupun logika \textit{UI} khusus dalam halaman ini. Kode sumber implementasi \textit{back-end} dapat dilihat pada Kode Sumber \ref{cdbe.06-03}.

\begin{lstlisting}[label=cdbe.06-03,style=php,caption=Implementasi \textit{Back-end} Kasus Penggunaan Memperbarui kupon]

/** File : app/Http/Controllers/CouponController
 * Menampilkan halaman perbarui kupon
 * Method : GET
 */
public function edit($id)
{
    try{
        $data['user'] = $this->userRepository->checkbox();
        $data['coupon'] = $this->couponRepository->find($id);
        return view('pages.coupon.edit', $data);
    }
    catch (\Exception $e){
        abort(403);
    }
}
/**
 * Update data kupon yang dimasukkan
 * fungsi ini dijalankan ketika form di halaman edit  kupon disubmit
 * terkait dengan CouponRepository
 */

public function update(Request $request, $id)
{

    $data = $request->all();
    if($this->couponRepository->update($id, $data)){
        return redirect('coupon')->with('success','Sukses memperbarui kupon!');
    }
    else return redirect('coupon')->with('error','Mohon maaf, gagal memperbarui kupon. Silahkan coba lagi');
}


/**
 * File CouponRepository
 */
public function update($id, $data)
{
    $coupon = $this->find($id);
    return $coupon->update($data);

}
\end{lstlisting}
	  
%  \begin{figure}[H]
%    \centering
%    \includegraphics[width=\textwidth]{images/bab4/ui/06-03.png}
%    \caption{Halaman Antarmuka Kasus Penggunaan Memperbarui kupon}
%    \label{ui.01-01}
%  \end{figure}
      
