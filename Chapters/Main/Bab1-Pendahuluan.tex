\chapter{PENDAHULUAN}
  Pada bab ini akan dipaparkan mengenai garis besar Tugas Akhir yang meliputi latar belakang, tujuan, rumusan dan batasan permasalahan, metodologi pembuatan Tugas Akhir, dan sistematika penulisan.
  \section{Latar Belakang}
  	
	\indent Transaksi jual beli saat ini sudah dapat dilakukan lewat berbagai cara, antara lain menggunakan \textit{e-commerce}, atau lewat \textit{social media}, atau bisa dengan melelang di aplikasi lelang \textit{online}. Sedikit berbeda dengan teknik penjualan di lelang online, karena aplikasi ini dapat diakses oleh banyak orang, tentu saja pelelang (\textit{auctioneer}) tidak terbatas pada ruang lelang saja, tapi bisa berasal dari manapun selama mereka mengakses aplikasi tersebut.  Lelang \textit{online} ini tentu saja mendatangkan banyak manfaat, selain biaya yang lebih efisien dan hemat, dan juga tidak menguras waktu karena siapapun, kapanpun, dimanapun dapat mengajukan penawaran ataupun melelang barangnya tanpa harus pergi ke instansi tertentu dan melakukan lelang dengan cara konvensional.
    \\
    \indent Aplikasi serupa telah banyak, namun banyak aspek yang kurang dalam aplikasi tersebut, seperti informasi dari lelang tidak \textit{reliable} (misal: stok barang ternyata sudah habis), alur proses yang tidak jelas sehingga membingungkan pengguna aplikasi, informasi yang kurang jelas, dan produk yang didapatkan ternyata tidak sesuai dengan informasi pada saat produk dilelang (\textit{bad information}) \cite{ying-feng_kuo_online_2016}.
    \\
    \indent Dan dari masalah teknis aplikasi, beberapa sumber menyatakan bahwa ketidakjelasan alur proses yang kurang diperhatikan oleh para developer aplikasi lelang \textit{online} menjadi beberapa alasan yang kuat mengapa lelang online masih kurang diminati \cite{noauthor_sistem_nodate}.
    \\
	\indent Diharapkan, dengan adanya aplikasi ini, beberapa kelemahan yang masih ada pada aplikasi lelang \textit{online} saat ini dapat diperbaiki, dan juga dapat dapat membantu proses \textit{online} yang ada di Indonesia, dan juga mampu memperbaiki citra aplikasi lelang \textit{online} sehingga mampu meningkatkan minat masyarakat terhadap lelang \textit{online}.
    
  \section{Rumusan Masalah}
    Rumusan masalah yang diangkat dalam tugas akhir ini adalah sebagai berikut: 
    \begin{enumerate}
      \item Bagaimana membangun aplikasi lelang online berbasis web?
      \item Bagaimana rancangan arsitektur aplikasi dan fitur yang menganalisa kelemahan aplikasi serupa dan strategi penyelesaian sesuai dengan paper acuan \cite{ying-feng_kuo_online_2016}?
    \end{enumerate}

  \section{Batasan Masalah}
  	\label{batasan-masalah}
    Dari permasalahan yang telah diuraikan di atas, terdapat beberapa batasan masalah pada tugas akhir ini, yaitu:
    \begin{enumerate}
      \item Aplikasi berbasis web dengan bahasa pemrograman PHP.
      \item Aplikasi berbasis kerangka kerja Laravel.
      \item Basis data yang digunakan adalah PostgreSQL.
      \item Aplikasi tidak mencakup proses pembayaran.
    \end{enumerate}

  \section{Tujuan}
  \label{tujuan}
    Tujuan dari pengerjaan Tugas Akhir ini adalah: 
    \begin{enumerate}
      \item Membangun aplikasi lelang online berbasis web yang lebih kredibel sesuai dengan paper yang dijadikan acuan pada tugas akhir ini. 
    \end{enumerate}