\chapter{PENUTUP}
  Bab ini membahas kesimpulan yang dapat diambil dari tujuan pembuatan sistem dan hubungannya dengan hasil uji coba dan evaluasi yang telah dilakukan. Selain itu, terdapat beberapa saran yang bisa dijadikan acuan untuk melakukan pengembangan dan penelitian lebih lanjut.
  \section{Kesimpulan}
  Dari proses perancangan, implementasi dan pengujian terhadap sistem, dapat diambil beberapa kesimpulan berikut:
  \begin{enumerate}
    \item Kualitas perancangan dan desain sistem dan fleksibilitas sistem sangat penting dalam rancang bangun aplikasi jual-beli online, karena sifat perubahan yang sangat cepat.
    \item \textit{User Experience} adalah faktor yang sangat penting dalam kesuksesan platform jual-beli online
    \item Selain \textit{user experience}, \textit{maintainability} juga sangat penting karena yang menjaga dan memperbarui perangkat lunak jika ada perubahan adalah \textit{developer} sendiri. Jika sebuah sistem \textit{maintainability}nya buruk, maka sistem tersebut juga tidak fleksibel terhadap perubahan karena \textit{developer} juga pasti pusing untuk menambahkan fitur yang 
  \end{enumerate}
  
  \section{Saran}
  Berikut beberapa saran yang diberikan untuk pengembangan lebih lanjut:
  \begin{enumerate}
    \item Mengikutsertakan pihak yang \textit{capable}/kredibel dan ahli di bidang hukum dan \textit{bussiness process} untuk menetapkan alur, memperbaiki alur dan membuat alur monitoring untuk proses lelang yang lebih aman, kredibel.
    \item Mempelajari platform lelang \textit{online} di luar negeri yang sudah sukses, yakni mempelajari ide-ide, alur aktivitas dan penggunaan kaidah \textit{user experience} dan \textit{usability} dalam website tersebut dan dampaknya terhadap 
  \end{enumerate}
  
