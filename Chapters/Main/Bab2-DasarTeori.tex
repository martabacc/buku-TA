\chapter{LANDASAN TEORI}{}
  \section{Lelang Daring / Lelang \textit{Online}}
   Lelang adalah proses membeli dan menjual barang atau jasa dengan cara menawarkan kepada penawar, menawarkan tawaran harga lebih tinggi, dan kemudian menjual barang kepada penawar harga tertinggi. Dalam teori ekonomi, lelang mengacu pada beberapa mekanisme atau peraturan perdagangan dari pasar modal. \\
	Sementara lelang daring atau lelang melalui internet muncul seiring dengan perkembangan internet. Barang atau jasa yang diperjualbelikan dipasang di situs dan peserta lelang dapat mengikuti acara lelang secara daring. Perusahaan lelang yang berhasil menggunakan sarana internet salah satunya adalah \textit{Ebay} . Di Indonesia, lelang melalui internet (online) sudah dipelopori oleh pemerintah dengan situs lelang online yang dapat diakses melalui website resmi \href{https://www.lelangdjkn.kemenkeu.go.id}{Kemenkeu} \cite{wikipedia_lelang_2016} . 
    Berikut adalah beberapa istilah yang ada dalam lelang online :
    \begin{enumerate}
	\item BID atau \textit{Bidding}, artinya : Menawarkan
    \item BIN (\textit{Buy In Now}) artinya : Beli sesuai harga yang telah ditawarkan penjual
    \item INC (\textit{Increment}) artinya : Minimum kenaikan \textit{bid} setelah \textit{bid} sebelum nya \cite{noauthor_arti_nodate}
    \end{enumerate}
    
    \section{PostgreSQL}
    PostgreSQL adalah sebuah produk \textit{database} relasional yang termasuk dalam kategori \textit{free open source software} (\textit{FOSS}). 
	PostgreSQL terkenal karena fitur-fitur yang advanced dan pendekatan rancangan modelnya menggunakan paradigma \textit{object-oriented}, sehingga sering dikategorikan sebagai \textit{Object Relational Database Management System} (ORDBMS).
    Beberapa fitur PostgreSQL adalah sebagai berikut :
    \begin{enumerate}
    \item \textit{Inheritance}, dimana satu table dapat diturunkan model dan beberapa karakteristik dari table lainnya.
    \item \textit{Multi-Version Concurrency Control} dimana user diberi data snapshot ketika suatu perubahan dilakukan sampai commit.
    \item \textit{Rules} , dimana suatu \textit{query} DML yang dikirimkan ke server akan mengalami penulisan ulang (\textit{rewrite}). Ini terjadi sebelum diproses oleh \textit{query planner}.
    \item dan berbagai fitur lainnya \cite{noauthor_postgresql_nodate}
    \end{enumerate}
    
  \section{Redis}
    Redis adalah \textit{open source}, struktur data yang ditempatkan di memori, digunakan sebagai \textit{database}, \textit{cache} dan \textit{message broker}. Redis mendukung struktur data seperti \textit{string, sets, hash, lists} dan \textit{sorted sets}. Sama seperti cache, setiap key diisi oleh value. Tapi kelebihannya, Redis bisa digunakan untuk melakukan operasi dari value tersebut. Cara terbaik untuk memahami redis adalah membuat model aplikasi tanpa memikirkan bagaimana caranya untuk menyimpan data di dalam \textit{database} \cite{yudana_redis_2015}.

  \section{Node.js}
  Node.js adalah platform perangkat lunak pada sisi-server dan aplikasi jaringan. Ditulis dengan bahasa javascript dan bisa dijalankan pada Windows, Mac OS X dan Linux tanpa perubahan kode program. Node.js memiliki pustaka server HTTP sendiri sehingga memungkinkan untuk menjalankan webserver tanpa menggunakan program webserver seperti Apache atau Lighttpd \cite{noauthor_node.js_2014}.
  
  \section{Socket.io}
	Socket.io adalah \textit{library} Javascript untuk aplikasi web yang bersifat \textit{realtime}. Socket.io menjembatani antara komunikasi dua arah antara \textit{web} \textit{clients} dan \textit{server}. Socket.io terbagi menjadi dua bagian, yaitu \textit{client}-\textit{side} \textit{library} yang berjalan di browser client, dan \textit{server}-\textit{side} \textit{library} yang menggunakan Node.js. Kedua komponen tersebut mempunyai API yang sama. Seperti Node.js, Socket.io juga bersifat \textit{event}-\textit{driven}. Socket.IO menggunakan protokol \textit{websocket} dengan \textit{polling} sebagai opsi \textit{fallback}. Meskipun Socket.IO merupakan ‘pembungkus’ untuk soket web, namun ia memiliki banyak fitur, antara lain broadcast ke banyak soket, dan I/O yang asinkronus \cite{noauthor_socket.io_2016}.
    
    \section{Laravel}
    Laravel adalah \textit{framework} PHP yang dikembangkan pertama kali oleh Taylor Otwell. Walaupun termasuk baru, namun komunitas pengguna laravel sudah berkembang pesat dan mampu menjadi alternatif utama dari sejumlah \textit{framework} besar seperti CodeIgniter dan Yii. Laravel oleh para \textit{developer} disetarakan dengan CodeIgniter dan FuelPHP namun memiliki keunikan tersendiri dari sisi \textit{coding}. Laravel memiliki beberapa keunggulan, diantaranya :
\begin{enumerate}
\item Sintaks yang sederhana dan \textit{programmer}-\textit{fiendly}
\item Tersedia \textit{generator} yang canggih dan memudahkan, Artisan CLI
\item Fitur \textit{Schema} \textit{Builder} untuk berbagai \textit{database}
\item Fitur \textit{Migration} dan \textit{Seeding} untuk berbagai \textit{database}
\item Fitur \textit{Query} \textit{Builder} yang powerful
\item Eloquent ORM (\textit{Object} \textit{Relational} \textit{Mapping})
\item Fitur pembuatan \textit{package} dan \textit{bundle}
\item \textit{Dependency} \textit{Injection} \cite{a}
\end{enumerate}

	\section{Protokol SMTP}
    
	\section{JSON Web Token}
    
   \section{Service Worker}
   
   \section{Repository Pattern}
   
   \section{Concurrency}
   
   \section{Transaction Isolation}
   
   \section{Script Testing}
   
   \section{Laravel Dusk}
   
   

