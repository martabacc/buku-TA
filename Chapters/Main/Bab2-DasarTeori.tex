\chapter{LANDASAN TEORI}

  \section{Lelang}
  Lelang adalah proses membeli dan menjual barang atau jasa dengan cara menawarkan kepada penawar, menawarkan tawaran harga lebih tinggi, dan kemudian menjual barang kepada penawar harga tertinggi. Dalam teori ekonomi, lelang mengacu pada beberapa mekanisme atau peraturan perdagangan dari pasar modal \cite{balailelang_sejarah_nodate}.
  \\ \indent
  Lelang menurut sejarahnya berasal dari bahasa Latin \textit{auctio} yang berarti peningkatan harga secara bertahap. Para ahli menemukan bahwa dalam literatur Yunani, lelang telah dikenal 450 tahun sebelum Masehi. Jenis lelang yang populer saat itu antara lain adalah karya seni, tembakau, kuda, budak dan sebagainya\cite{pratama_lelang_2012}.
  
  
  \section{\textit{Vendu Reglement}}
  Lelang dilegalisasi \& resmi masuk Indonesia dalam perundang-undangan sejak 1908, yaitu dengan berlakunya \textit{Vendu Reglement}, Stbl. 1908 No. 189 dan \textit{Vendu Instructie}, Stbl 1908 No. 190. \textit{Vendu Reglement} ini berisikan peraturan-peraturan dasar lelang yang berlaku hingga saat ini, dan menjadi dasar hukum penyelenggaraan lelang di Indonesia\cite{balailelang_sejarah_nodate}.
  
 
  \section{Lelang Daring / Lelang \textit{Online}}
   Lelang adalah proses membeli dan menjual barang atau jasa dengan cara menawarkan kepada penawar, menawarkan tawaran harga lebih tinggi, dan kemudian menjual barang kepada penawar harga tertinggi. Dalam teori ekonomi, lelang mengacu pada beberapa mekanisme atau peraturan perdagangan dari pasar modal. \\
	Sementara lelang daring atau lelang melalui internet muncul seiring dengan perkembangan internet. Barang atau jasa yang diperjualbelikan dipasang di situs dan peserta lelang dapat mengikuti acara lelang secara daring. Perusahaan lelang yang berhasil menggunakan sarana internet salah satunya adalah \textit{Ebay}. Di Indonesia, lelang melalui internet (online) sudah dipelopori oleh pemerintah dengan situs lelang online yang dapat diakses melalui website resmi \href{https://www.lelangdjkn.kemenkeu.go.id}{Kemenkeu} \cite{wikipedia_lelang_2016}. 
    Berikut adalah beberapa istilah yang ada dalam lelang online:
    \begin{enumerate}
	\item BID atau \textit{Bidding}, artinya: Menawarkan
    \item BIN (\textit{Buy In Now}) artinya: Beli sesuai harga yang telah ditawarkan penjual
    \item INC (\textit{Increment}) artinya: Minimum kenaikan \textit{bid} setelah \textit{bid} sebelum nya \cite{noauthor_arti_nodate}
    \end{enumerate}
   
   \section{\textit{Repository Pattern}}
	\textit{Repository Pattern} adalah sebuah pola dalam struktur \textit{software engineering} yang memisahkan \textit{data management layer} ke dalam sebuah layer tersendiri - yang di\textit{handle} oleh sebuah bagian struktur yang disebut repository. 
	\\ \indent
	Jika menggunakan \textit{pattern} ini, semua kode spesifik yang terkait dengan \textit{persistence logic} dan implementasi akses data berhenti sampai di \textit{repository} (\textit{controller} hanya me\textit{redirect} \textit{request} dan validasi \textit{request})\cite{noauthor_repository_2016}.
	\\ \indent
   Dalam sebuah referensi, disebutkan bahwa: "\textit{The repository pattern covers large centralized transaction-oriented databases, the blackboard systems used for some AI applications, and systems with predetermined execution patterns in which different phases add information to a single complex data structures (e.g., compilers). These variants differ chiefly in their control structure.}" \cite{shaw_patterns_1996}.
   
   \section{\textit{Data Growth}}
   Yang dimaksud dalam \textit{data growth} pada \textit{section} ini adalah seberapa cepat perkembangan jumlah data yang disimpan oleh server. Data tersebut bisa berupa \textit{row} dalam \textit{database} ataupun data \textit{gambar, video dll} \cite{insidebigdata_exponential_2017}.
   \\ \indent
   Sejak tahun 2000s, perkembangan data meningkat pesat dan memunculkan bisnis penyedia \textit{data storage} dan \textit{networking equipment}.
   \\ \indent Sebagai ilustrasi, pada buku referensi tercatat bahwa \textit{data equilibrium flow} firma-firma \textit{e-Commerce} pada dekade 1990-2000 meningkat hingga 1.5 \textit{billion gigabytes} setiap tahunnya\cite{d._vanhoose_e-commerce_2011}.
  
   
   \section{\textit{Concurrency}}
   Konkurensi adalah bisa dikatakan sebagai suatu fitur di mana database management system(DBMS) mengijinkan banyak transaksi pada saat bersamaan untuk mengakses data yang sama. Dalam melakukan konkurensi dibutuhkan suatu Concurency Control Mechanism (CCM) agar transaksi yang dilakukan oleh banyak user pada suatu sistem di dalam waktu yang bersamaan tidak saling “mengganggu” dan tidak menghasilkan inconsistency data.
   \\ \indent
   Tiga masalah umum yang muncul dalam konkurensi adalah sebagai berikut: 
	   \begin{enumerate}
	   	\item \textit{Lost Update Problem }\\ \indent Masalah operasi update yang sukses dari seorang pengguna kemudian ditimpali oleh operasi update dari pengguna lain.
	   	\item\textit{ Uncomited dependency problem } (ketergantungan yg tidak sukses/modifikasi sementara) \\ \indent
	   	Masalah terjadi saat suatu transaksi membaca data dari transaksi lain yg belum dicommit.
	   	\item \textit{Inconsistent analysis problem} \cite{noauthor_sistem_2013}
	   \end{enumerate}   
   
   \section{NoSQL}
   NoSQL adalah istilah yang dikenal dalam teknologi komputasi untuk merujuk kepada kelas yang luas dari sistem manajemen basis data yang diidentifikasikan dengan tidak mematuhi aturan pada model sistem manajemen basis data relasional yang banyak digunakan.
   \\ \indent
   NoSQL tidak dibangun terutama dengan table dan umumnya tidak menggunakan SQL untuk memanipulasi data, sehingga sering ditafsirkan sebagai “tidak hanya SQL” \cite{wikipedia_nosql_nodate}.
   
   \section{Protokol SMTP}
   Simple Mail Transfer Protocol (SMTP) adalah suatu protokol yang digunakan untuk mengirimkan pesan e-mail antar server, yang bisa dianalogikan sebagai kantor pos. Ketika kita mengirim sebuah e-mail, komputer kita akan mengarahkan e-mail tersebut ke sebuah SMTP server, untuk diteruskan ke mail-server tujuan \cite{noauthor_smtp_nodate}. 
   
   \section{Whitelist}
   Whitelist sendiri adalah memindahkan daftar alamat atau domain dari pengirim email dengan harapan agar muncul pada kotak pesan email utama Anda. Sederhananya, hal ini dilakukan untuk menghindari agar pesan email yang dikirimkan oleh pengirim pesan sebenarnya tidak terbaca sebagai spam \cite{wikipedia_sendgrid_2017}.
   
   \section{JSON \textit{Web Token}}
   \textit{JSON Web Token }atau lebih dikenal dengan JWT, merupakan sebuah token berbentuk JSON yang padat-informasi (ukurannya), informasi mandiri untuk ditransmisikan antar pihak yang terkait. Token tersebut ini dapat diverifikasi dan dipercaya karena sudah di-sign secara digital. Token JWT bisa di-sign dengan menggunakan secret (algoritma HMAC) atau pasangan public / private key (algoritma RSA) \cite{noauthor_jwt_2016}.
   
   \section{\textit{Service Worker}}
   \textit{Service worker }adalah skrip yang dijalankan browser Anda di latar belakang, terpisah dari laman web, yang membuka pintu ke berbagai fitur yang tidak memerlukan laman web atau interaksi pengguna. Saat ini, service worker sudah menyertakan berbagai fitur seperti pemberitahuan push dan sinkronisasi latar belakang. Di masa mendatang, service worker akan mendukung hal-hal lainnya seperti sinkronisasi berkala atau geofencing. Fitur inti yang didiskusikan dalam tutorial adalah kemampuan mencegat dan menangani permintaan jaringan, termasuk mengelola cache respons lewat program\cite{google_developers_service_2017}.
   
       
   \section{Laravel}
   Laravel adalah \textit{framework} PHP yang dikembangkan pertama kali oleh Taylor Otwell. Walaupun termasuk baru, namun komunitas pengguna laravel sudah berkembang pesat dan mampu menjadi alternatif utama dari sejumlah \textit{framework} besar seperti CodeIgniter dan Yii. Laravel oleh para \textit{developer} disetarakan dengan CodeIgniter dan FuelPHP namun memiliki keunikan tersendiri dari sisi \textit{coding}. Laravel memiliki beberapa keunggulan, diantaranya:
   \begin{enumerate}
      	\item Sintaks yang sederhana dan \textit{programmer}-\textit{fiendly}
      	\item Tersedia \textit{generator} yang canggih dan memudahkan, Artisan CLI
      	\item Fitur \textit{Schema} \textit{Builder} untuk berbagai \textit{database}
      	\item Fitur \textit{Migration} dan \textit{Seeding} untuk berbagai \textit{database}
      	\item Fitur \textit{Query} \textit{Builder} yang powerful
      	\item Eloquent ORM (\textit{Object} \textit{Relational} \textit{Mapping})
      	\item Fitur pembuatan \textit{package} dan \textit{bundle}
      	\item \textit{Dependency} \textit{Injection} \cite{noauthor_apa_2016}
   \end{enumerate}
   
   \section{Vue.js / Vue}
   Vue adalah sebuah \textit{framework} Javascript yang \textit{progressive} dan bersifat \textit{open-source} untuk membangun \textit{user interface}. Integrasi kedalam project yang menggunakan \textit{library } Javascript lain menjadi lebih mudah dengan Vue karena Vue memang didesain untuk \textit{incrementally adoptable}. Vue juga dapat berfungsi sebagai \textit{web application framewowrk} untuk membangun sebuah \textit{single-page applications}.
   \\ \indent
   Dalam 2016 \textit{Javascript Survey}, Vue mendapatkan 89\% untuk kategori \textit{developer satisfaction rating}. Vue mengakumulasikan 98 \textit{stars Github} setiap hari, dan menduduki peringkat ke-10 \textit{Project Github} dengan bintang terbanyak \textit{all the time}\cite{wikipedia_vue.js_2017}.
          
   \section{Node.js}
   Node.js adalah platform perangkat lunak pada sisi-server dan aplikasi jaringan. Ditulis dengan bahasa javascript dan bisa dijalankan pada Windows, Mac OS X dan Linux tanpa perubahan kode program. Node.js memiliki pustaka server HTTP sendiri sehingga memungkinkan untuk menjalankan webserver tanpa menggunakan program webserver seperti Apache atau Lighttpd \cite{noauthor_node.js_2014}.
   
   \section{npm / \textit{Node Package Manager}}
   NPM memiliki dua fungsi utama, yaitu sebagai repositori online yang berisi banyak package atau module untuk aplikasi NodeJS dan yang kedua adalah sebuah utilitas baris perintah (command line) yang digunakan untuk menginstal paket-paket yang dibutuhkan dan juga untuk mengelola versi dan ketergantungan package dari NodeJS. Dengan NPM Anda akan mudah mencari, menginstal, uninstall aplikasi atau module/package Node.js\cite{azurri_node_2016}.
   
   \section{Socket.io}
   Socket.io adalah \textit{library} Javascript untuk aplikasi web yang bersifat \textit{realtime}. Socket.io menjembatani antara komunikasi dua arah antara \textit{web} \textit{clients} dan \textit{server}. Socket.io terbagi menjadi dua bagian, yaitu \textit{client}-\textit{side} \textit{library} yang berjalan di browser client, dan \textit{server}-\textit{side} \textit{library} yang menggunakan Node.js. Kedua komponen tersebut mempunyai API yang sama. Seperti Node.js, Socket.io juga bersifat \textit{event}-\textit{driven}. Socket.IO menggunakan protokol \textit{websocket} dengan \textit{polling} sebagai opsi \textit{fallback}. Meskipun Socket.IO merupakan ‘pembungkus’ untuk soket web, namun ia memiliki banyak fitur, antara lain broadcast ke banyak soket, dan I/O yang asinkronus \cite{noauthor_socket.io_2016}.
         
   
   \section{PostgreSQL}
   PostgreSQL adalah sebuah produk \textit{database} relasional yang termasuk dalam kategori \textit{free open source software} (\textit{FOSS}). 
   PostgreSQL terkenal karena fitur-fitur yang advanced dan pendekatan rancangan modelnya menggunakan paradigma \textit{object-oriented}, sehingga sering dikategorikan sebagai \textit{Object Relational Database Management System} (ORDBMS).
   Beberapa fitur PostgreSQL adalah sebagai berikut:
   \begin{enumerate}
      	\item \textit{Inheritance}, dimana satu table dapat diturunkan model dan beberapa karakteristik dari table lainnya.
      	\item \textit{Multi-Version Concurrency Control} dimana user diberi data snapshot ketika suatu perubahan dilakukan sampai commit.
      	\item \textit{Rules}, dimana suatu \textit{query} DML yang dikirimkan ke server akan mengalami penulisan ulang (\textit{rewrite}). Ini terjadi sebelum diproses oleh \textit{query planner}.
      	\item dan berbagai fitur lainnya \cite{noauthor_postgresql_nodate}
   \end{enumerate}
   
   \section{MongoDB}
   MongoDB (dari "humongous") adalah sistem basis data berorentasi dokumen lintas platform. Diklasifikasikan sebagai basis data "NoSQL", MongoDB menghindari struktur basis data relasional tabel berbasis tradisional yang mendukung JSON seperti dokumen dengan skema dinamis (MongoDB menyebutnya sebagai format BSON), membuat integrasi data dalam beberapa jenis aplikasi lebih mudah dan lebih cepat. Dirilis di bawah kombinasi dari GNU Affero General Public License dan Lisensi Apache, MongoDB adalah perangkat lunak bebas dan sumber terbuka\cite{noauthor_mongodb_2017}.
   
   \section{Redis}
   Redis adalah \textit{open source}, struktur data yang ditempatkan di memori, digunakan sebagai \textit{database}, \textit{cache} dan \textit{message broker}. Redis mendukung struktur data seperti \textit{string, sets, hash, lists} dan \textit{sorted sets}. Sama seperti cache, setiap key diisi oleh value. Tapi kelebihannya, Redis bisa digunakan untuk melakukan operasi dari value tersebut. Cara terbaik untuk memahami redis adalah membuat model aplikasi tanpa memikirkan bagaimana caranya untuk menyimpan data di dalam \textit{database} \cite{yudana_redis_2015}.
   
   \section{SendGrid}
   SendGrid adalah sebuah \textit{customer comumunication platform} untuk email \textit{marketing} dan transaksional yang berbasis di Denver, Colorado.
   \\ \indent
   SendGrid menyediakan layanan pengiriman email yang berbasis \textit{cloud} kepada pihak bisnis. Layanan yang ditawarkan sangat beragam, mulai dari \textit{shipping notifications}, \textit{friend requests} dan lain lagi. \\ \indent
   Selain itu, juga dapat \textit{handling} ISP monitoring, \textit{domain keys, feedback loops}, dan juga memberikan report \textit{opened mails, unsubscribes, bounces dan spam reports}. Pada tahun 2012, SendGrid menggaet Twilio dan menambahkan layanan integrasi SMS, Suara dan \textit{push notification}. \cite{wikipedia_sendgrid_2017}
   
   \section{Amazon Web Service}
	Amazon Web Services adalah sekumpulan layanan-layanan berbasis Cloud Computing yang di sediakan oleh Amazon sejak tahun 2002. Meskipun salah satu perusahaan raksasa internet ini sering kita kenal untuk membeli buku dan lagu, namun sekarang Amazon telah menambah layanannya dalam hal infrastrutktur cloud computing. Amazon Web Services ini menyediakan layanan-layanan nya yang saling terintegrasi dan mudah kustomisasi\cite{wikipedia_amazon_2016}.
	\\ \indent
	Dalam website resminya, disebutkan bahwa AWS dapat membantu aplikasi menjadi lebih cepat, lebih aman, dan menghemat \textit{costs} dengan \textit{scaling} \textit{performance} menggunakan teknologi \textit{cloud computing}\cite{web_services_amazon_nodate}.
   
   
   \section{\textit{Test Script}}
   Test Script dalam dunia \textit{software testing} adalah set instruksi atau sekumpulan baris kode yang akan melakukan \textit{testing} terhadap fungsionalitas sistem dengan target tertentu \cite{noauthor_test_2016}.
   \\ \indent Ada beberapa jenis \textit{script test}:
   \begin{enumerate}
   	\item \textit{Manual testing}, atau lebih sering disebut \textit{test cases}
   	\item \textit{Automated Testing}
   \end{enumerate}
   
   \section{Laravel Dusk}
   Laravel Dusk adalah sebuah fitur baru yang ditujukan untuk \textit{functional testing}, yang baru diluncurkan dan dibenamkan secara \textit{default} pada Laravel versi 5.4. Dalam \textit{site} dokumentasinya, disebutkan bahwa Laravel Dusk menyediakan \textit{browser automation \& testing API} yang ekspresif dan mudah digunakan. Secara otomatis, Dusk tidak memerlukan instalasi JDK atau Selenium pada \textit{host}, namun menggunakan \textit{ChromeDriver standalone}, namun juga tetap bsia menggunakan driver Selenium yang kompatibel \cite{laravel_browser_nodate}.

