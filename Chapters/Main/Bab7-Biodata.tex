\chapter{BIODATA PENULIS}
		\begin{wrapfigure}{l}{0.3\textwidth}
			\includegraphics[width=0.3\textwidth]{images/foto-diri.jpg}
		\end{wrapfigure}
		\textbf{Ronauli Silva NS}, seorang yang lahiran \& besar di Siantar Medan - Sumatera Utara, sangat suka belajar. Diberi amanah untuk menjadi \textit{administrator} Laboratorium Pemrograman di tahun 2015, penulis belajar banyak mengenai administrasi \textit{server}, rancang bangun aplikasi terutama di bidang web. Selain itu, beberapa \textit{project} yang diambil penulis mengenai rancang bangun aplikasi yang baik dan buruk yang mengajarkan penulis cara memperbaiki, menangkal dan \& mengoptimasinya.\\
		Selain itu, penulis juga banyak belajar \textit{softskills} saat menjabat sebagai Sekretaris Departemen HMTC ITS 2015/2016 (Kabinet Optimasi) dan lewat pelatihan-pelatihan yang diberikan donatur beasiswa saat penulis masih diamanahi sebagai beswan Karya Salemba Empat 2014-2016.\\ \\
		Motto penulis yaitu "\textit{Always go for the extra miles}", membawa penulis mengambil topik tugas akhir ini, dimana penulis dapat menerapkan perbaikan, optimasi dan pelajaran yang penulis petik dari \textit{project-project} sebelumnya, dengan bimbingan dosen-dosen pembimbing penulis yang baik hati. Dalam pendalaman topik tugas akhir ini juga, penulis banyak belajar dan menjadi sangat tertarik mendalami \textit{bussiness engineering}, \textit{user experiences and usability}, dan \textit{data scientist \& engineering}. \\ \\
		Dengan segala kerendahan hati, ilmu penulis masihlah setitik dibandingkan susu sebelanga. Penulis sangat mengharapkan diskusi, ajaran dan bantuan dalam memperbaiki diri. Apabila pembaca berkenan, penulis dapat dihubungi melalui \textit{email} ke \texttt{ronayumik@gmail.com}, atau \textit{Whatsapp} ke nomor +62821-6066-5568.


