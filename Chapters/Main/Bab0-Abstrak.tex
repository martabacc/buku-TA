\begin{abstrak}
	Industri e-commerce berkembang dengan pesat di Indonesia, seiring dengan meningkatnya jumlah pengguna internet dan menjamurnya bisnis online atau sering disebut \textit{online shop}. Salah satu jenisnya adalah lelang online, yaitu metode jual beli yang mengintegrasikan mekanisme lelang dengan Internet.\\
    \indent Dalam interaksi antara pelaku lelang online (penjual dan pembeli) pasti terjadi kegagalan/ketidakpuasan dalam transaksi lelang online.Berangkat sebuah paper yang membahas mengenai analisa kesalahan dan strategi lewat survey terhadap pengguna aplikasi lelang online di Taiwan, penulis membangun aplikasi lelang online yang disertai dengan tambahan fitur maupun saran dari paper tersebut. \\ 
    \indent Tidak hanya berdasarkan paper rujukan, penulis juga menganalisa aplikasi e-commerce yang umum digunakan di Indonesia baik user experience maupun alur transaksi, dan menambahkan beberapa fitur agar lebih sesuai dengan transaksi jual-beli online yang umum di Indonesia. Dengan aplikasi ini, diharapkan kegagalan dalam transaksi online dapat diperbaiki dan membuka peluang lelang online untuk meramaikan industri e-commerce di Indonesia.\\
\noindent \textbf{Kata-Kunci}: \textit{lelang online}, \textit{strategi }
\end{abstrak}