\chapter{PENGUJIAN DAN EVALUASI}

	Pada bab ini akan dibahas mengenai pengujian dan evaluasi pada aplikasi. Pengujian yang dilakukan terdiri dari dua pengujian yaitu pengujian fungsionalitas sistem dan pengujian statistik. Pengujian fungsionalitas mengacu pada daftar fungsionalitas pada bab III (Desain dan Perancangan) Sedangkan pengujian statistik dilakukan untuk membuktikan bahwa aplikasi benar telah mencapai tujuan yang dipaparkan pada Bab II poin 2.
    
	\section{Lingkungan Uji Coba}
    Lingkungan uji coba yang digunakan untuk pengujian adalah sebagai berikut :
    \begin{enumerate}
    \item Sistem Operasi Windows
    	\label{env_uji1}
    	\begin{enumerate}[label=(\alph*)]
        \item Version	10.0.14393 Build 14393
        \item Processor	Intel(R) Core(TM) i5 CPU       M 480  @ 2.67GHz, 2667 Mhz, 2 Core(s), 4 Logical Processor(s)
        \item System Model	HP Pavilion g4 Notebook PC
		\item System Type	x64-based PC
        \item RAM 6,00 GB
        \end{enumerate}
    \end{enumerate}
    
    \section{Pengujian}
    
      Pada subbab ini akan dijelaskan mengenai skenario pengujian yang dilakukan pada sistem, yaitu pengujian fungsionalitas dan pengujian statistik.
      \subsection{Pengujian Fungsionalitas}
      Pengujian ini dilakukan mengacu kepada subbab ... (mengarah ke kamus data). Pengujian fungsionalitas dilakukan secara mandiri oleh penulis, dengan sejumlah skenario sebagai tolak ukur keberhasilan pengujian.
      \subsubsection{Pengujian Fitur A}
      Pengujian fitur ini dilakukan pada lingkungan uji \ref{env_uji1}, dan untuk lebih lengkapnya dapat dilihat pada tabel \ref{uji-fung-1}
\begin{table}[]
\centering
\caption{Pengujian Fungsionalitas Fitur A}
\label{uji-fung-1}
\begin{tabular}{|r|l|}
\hline
\textbf{ID}                                                           & \multicolumn{1}{c|}{\textbf{TA-UJI.Proses}} \\ \hline
\begin{tabular}[c]{@{}r@{}}Referensi\\ Proses Penggunaan\end{tabular} &                                             \\ \hline
Nama                                                                  &                                             \\ \hline
\begin{tabular}[c]{@{}r@{}}Tujuan \\ Pengujian\end{tabular}           &                                             \\ \hline
\textbf{\begin{tabular}[c]{@{}r@{}}Skenario\\ Pengujian\end{tabular}} & \textbf{}                                   \\ \hline
Kondisi Awal                                                          &                                             \\ \hline
Data Uji                                                              &                                             \\ \hline
\begin{tabular}[c]{@{}r@{}}Langkah\\ Pengujian\end{tabular}           &                                             \\ \hline
\begin{tabular}[c]{@{}r@{}}Hasil\\ Diharapkan\end{tabular}            &                                             \\ \hline
\begin{tabular}[c]{@{}r@{}}Hasil\\ Pengujian\end{tabular}             &                                             \\ \hline
\begin{tabular}[c]{@{}r@{}}Kondisi\\ Akhir\end{tabular}               &                                             \\ \hline
\end{tabular}
\end{table}
      
      
    \subsection{Pengujian Statistik}
    	Pengujian statistik ditujukan untuk membuktikan benar tidaknya strategi yang diajukan oleh paper acuan \cite{ying-feng_kuo_online_2016}, yaitu dengan menggunakan skenario sesuai masalah-masalah yang dipaparkan dalam paper tersebut kepada pengguna aplikasi, lalu penulis menerapkan strategi yang diajukan oleh paper tersebut. Setelah proses penerapan strategi selesai, penulis lalu memberikan kuisioner kepada pengguna aplikasi terkait kepuasan dan \textit{return rate} untuk kembali menggunakan aplikasi, juga tingkat kepercayaan pengguna aplikasi terhadap proses lelang.
        \subsection{Ringkasan Pengguna Aplikasi}
        Untuk pengujian ini, penulis mencantumkan ringkasan informasi pengguna yang terlibat dalam pengujian, karena umur, riwayat transaksi online dan tingkat kepercayaan berpengaruh kepada pemberian strategi yang akan diberikan pada tabel \ref{skenario} . Tabel Ringkasan Pengguna Aplikasi ini dapat dilihat pada tabel \ref{ringkasan-user-uji}.
        
        
       \begin{table}[]
      \centering
      \caption{Tabel Ringkasan Informasi Pengguna Aplikasi yang terlibat dalam Pengujian}
      \label{ringkasan-user-uji}
      \begin{tabular}{|c|l|c|c|c|}
      \hline
      \textbf{No.} & \textbf{Inisial} & \textbf{\begin{tabular}[c]{@{}c@{}}Umur/\\ Gender\end{tabular}} & \textbf{\begin{tabular}[c]{@{}c@{}}Riwayat\\ Transaksi\\ Online\end{tabular}} & \textbf{\begin{tabular}[c]{@{}c@{}}Kepercayaan\\ thd Transaksi\\ Online\end{tabular}} \\ \hline
      1& AB               & 21/P & Sering                                                                        & Tinggi                                                                                \\ \hline
      2            & CD               & 30/L                                                            & Jarang                                                                        & Sedang                                                                                \\ \hline
      \end{tabular}
      \end{table}
        
        Tabel selanjutnya berisi \textit{mapping} antara pengguna, masalah yang dihadapi strategi yang diberikan kepada pengguna. Tabel ini dapat dilihat pada tabel \ref{skenario}        
        \begin{table}[]
        \centering
        \caption{Tabel Skenario Pengujian}
        \label{skenario}
        \begin{tabular}{|c|l|c|c|}
        \hline
        \textbf{No}             & \textbf{Inisial} & \textbf{Skenario}                                                                         & \textbf{\begin{tabular}[c]{@{}c@{}}Strategi\\ yang Diterapkan\end{tabular}}                \\ \hline
        1                       & AB               & \begin{tabular}[c]{@{}c@{}}Bad Information\\ Barang tidak sesuai dengan spek\end{tabular} & \multirow{2}{*}{\begin{tabular}[c]{@{}c@{}}Pemberian Voucher\\ Free Shipping\end{tabular}} \\ \cline{1-3}
        \multicolumn{1}{|l|}{2} & DF               & \begin{tabular}[c]{@{}c@{}}Seller Fraud\\ Barang tidak dikirim\end{tabular}               &                                                                                            \\ \hline
        \multicolumn{1}{|l|}{3} & GG               & \begin{tabular}[c]{@{}c@{}}Slow Shipping\\ Pengiriman barang sangat lambat\end{tabular}   & \multirow{2}{*}{\begin{tabular}[c]{@{}c@{}}Pemberian Voucher\\ Diskon\end{tabular}}        \\ \cline{1-3}4 & CD & Skenario D                                                                                &                                                                                            \\ \hline
        \end{tabular}
        \end{table}

	Hasil pengujian dari setiap skenario yang ada di tabel \ref{skenario} disajikan per tabel tabel, seperti yang dapat dilihat pada tabel berikut :
        
        
	\section{Evaluasi Pengujian}
    Pada subbab ini, akan diberikan hasil evaluasi terhadap pengujian yang telah dilakukan, meliputi hasil evaluasi pengujian fungsional dan pengujian statistik.
    
    \subsection{Evaluasi Pengujian Fungsionalitas}
    Rangkuman pengujian fungsionalitas dapat dilihat pada tabel \ref{eval-uji-fung}
    
    \begin{table}[]
    \centering
    \caption{Rangkuman Pengujian Fungsionalitas di Semua Lingkungan Uji}
    \label{eval-uji-fung}
    \begin{tabular}{|c|c|c|c|c|}
    \hline
    No                 & ID                       & \textbf{Nama} & \textbf{Skenario 1} & \textbf{\begin{tabular}[c]{@{}c@{}}Ling.\\ Uji\\ 1\end{tabular}} \\ \hline
    \multirow{2}{*}{1} & \multirow{2}{*}{UJI-F-1} & Fitur 1       & Skenario A1         & v                                                                \\ \cline{3-5} 
                       &                          & Fitur 2       & Skenario A2         & v                                                                \\ \hline
    \multirow{2}{*}{2} & \multirow{2}{*}{UJI F-2} & Fitur 3       & Skenario B1         & v                                                                \\ \cline{3-5} 
                       &                          & Fitur 4       & Skenario B2         & v                                                                \\ \hline
    \end{tabular}
    \end{table}
    
    \subsection{Evaluasi Pengujian Statistik}
    Rangkuman pengujian Statistik dapat dilihat pada tabel \ref{eval-uji-fung}, yang mencakup persentase perubahan pengguna terhadap aspek kepercayaan terhadap aplikasi, \textit{return rate}, dan kepercayaan terhadap transaksi online.

\begin{table}[]
\centering
\caption{Evaluasi Pengujian Statistik}
\label{eval-uji-statistik}
\resizebox{\textwidth}{!}{ \begin{tabular}{|c|c|c|l|l|l|}
\hline
\multirow{2}{*}{\textbf{\begin{tabular}[c]{@{}c@{}}\#\\ Skenario\end{tabular}}} & \multirow{2}{*}{\textbf{\begin{tabular}[c]{@{}c@{}}Tahap\\ Pengujian\end{tabular}}} & \multicolumn{3}{c|}{\textbf{Aspek yang Dinilai}} & \multicolumn{1}{c|}{\multirow{2}{*}{\textbf{Kesimpulan}}} \\ \cline{3-5}& & \textbf{\begin{tabular}[c]{@{}c@{}}Kepercayaan\\ terhadap\\ Aplikasi\end{tabular}} & \multicolumn{1}{c|}{\textbf{\begin{tabular}[c]{@{}c@{}}Return\\ Rate\end{tabular}}} & \multicolumn{1}{c|}{\textbf{\begin{tabular}[c]{@{}c@{}}Kepercayaan\\ Terhadap\\ Transaksi\\ Online\end{tabular}}}  & \multicolumn{1}{c|}{}                     \\ 
\hline
\multirow{3}{*}{1}                                                & \begin{tabular}[c]{@{}c@{}}Keadaan\\ Awal\end{tabular}                              & n \%                                                                               & a\%                                                                                 & a\%                                                                                                             & \multirow{3}{*}{Meningkat}                                \\ \cline{2-5}
                                                                                & \begin{tabular}[c]{@{}c@{}}Saat\\ Skenario\end{tabular}                             & x \%                                                                               & a\%                                                                                 & a\%                                                                                                             &                                                           \\ \cline{2-5}
                                                                                & \begin{tabular}[c]{@{}c@{}}Setelah\\ Strategi\end{tabular}                          & z\%                                                                                & a\%                                                                                 & a\%                                                                                                             &                                                           \\ \hline
\end{tabular}}
\end{table}

	{Paragraf membahas evaluasi}