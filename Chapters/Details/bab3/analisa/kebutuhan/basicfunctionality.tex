\subsection{Deskripsi Umum Aplikasi}
	\label{deskripsi-umum-app}

	\textit{Bid! Bid! Bid!} Istilah tersebut pasti tidak asing bagi para pecinta lelang online. Lelang sendiri adalah proses membeli dan menjual barang atau jasa dengan cara menawarkan kepada penawar, menawarkan tawaran harga lebih tinggi, dan kemudian menjual barang kepada penawar harga tertinggi (Wikipedia.com).  Di Indonesia, sistem lelang sudah digunakan sejak jaman Hindia Belanda dimana saat itu sistem lelang pertama kali diperkenalkan di Indonesia dan biasa digunakan lelang terhadap aset-aset pejabat atau pemerintah yang dimutasi pada saat itu (Artikel sejarah lelang, 2011). \\
	\indent Aplikasi lelang online dapat digunakan sebagai wadah bagi para pecinta lelang online untuk melakukan kegiatan lelang atau bagi para penjual atau pembeli yang ingin menjual atau membeli barang yang sesuai baik dari kualitas maupun harga. Identifikasi aktor dalam sistem lelang online dijelaskan dalam tabel \ref{identifikasi-aktor}.
	
	\LTXtable{\textwidth}{tables/03/analisa/identifikasi_pengguna}
	 
	
	\indent Dalam proses bisnisnya sendiri, lelang cukup sederhana, yaitu pengguna dapat melelang dan dapat menjual barang untuk dilelang. Yang berarti, pengguna dapat memanajemen riwayat lelang dan memanajemen barang yang ia daftarkan untuk dilelang. Hal ini akan didefinisikan lebih lanjut dalam subbab \ref{keb-fungsional}, yaitu subbab Spesifikasi Kebutuhan Fungsional.
