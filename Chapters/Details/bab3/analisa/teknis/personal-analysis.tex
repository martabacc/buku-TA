
\subsection{Analisa Pribadi Penulis}
	Dalam pemaparan selanjutnya, penulis akan 
	\subsubsection{Analisa \textit{User Experience} dari E-Commerce di Indonesia}
	\label{alasan-ux-ecommerce-indonesia alasan-app-serupa}
	Selama masa pengerjaan aplikasi, penulis sering menganalisa dan memperhatikan kebiasaan-kebiasaan yang umum di website \textit{e-commerce} di Indonesia. Salah satu yang paling sering dianalisa oleh penulis adalah adalah situs Tokopedia. Dalam pengembangannya, \textit{user interface} aplikasi akan dipengaruhi analisa ini, yang dijabarkan seperti berikut:
	\begin{enumerate}
		\item Halaman yang muncul bukanlah eagerloading, tapi \textit{lazy loading}\\
		\indent Ini adalah solusi cerdas untuk mengakali \textit{delay loading item} yang sudah pasti jumlahnya sangat banyak (maka butuh \textit{query} yang tentunya memakan waktu cukup lama), namun juga memainkan faktor psikologi / \textit{user behaviour} pengguna dengan membiarkan pengguna melihat tahap demi tahap halaman 'diisi'.
		\item \textit{User Interface} yang sederhana dan pemilihan warna yang \textit{soft}
	\end{enumerate}
	
	\subsubsection{Analisa Keamanan pada koneksi Soket}
	\label{alasan-socket.io}
	Untuk mengakomodasi fitur yang bersifat \textit{realtime}, dibutuhkan koneksi ke soket secara terus menerus. Hal ini tentu dapat menjadi sasaran empuk \textit{security} karena jika tidak diamankan, maka dapat menjadi peluang besar bagi para pihak yang tidak berkepentingan untuk merusak proses bisnis aplikasi.\\
	\indent Namun, jika dalam setiap koneksi soket harus mengirimkan \textit{credentials}, hal ini tentu menjadi tidak praktis dan malah lebih berbahaya karena membiarkan data-data sensitif seperti \textit{password} dan \textit{username} berlalu-lalang di jaringan internet. Selain itu, \textit{disadvantage}nya adalah ketidakpraktisan untuk selalu meng\textit{query} database setiap kali ada koneksi, tentu saja ini memperlambat kerja \textit{database} dan menambah waktu \textit{delay}. Maka dari itu, penulis mengidentifikasi poin-poin penting berikut :
		\begin{itemize}
			\item Hindari \textit{query} database untuk \textit{autentikasi} yang sifatnya masif
			\item Menggunakan mekanisme authentikasi yang menggunakan \textit{credentials} karena rentan dengan masalah keamanan
			\item Mencari metode yang lebih efektif, cepat untuk autentikasi selain 
		\end{itemize}
	
	\subsubsection{Analisa \textit{Best Practice} dalam Struktur Perangkat Lunak}
		\label{alasan-best-practice}
		Pada dasarnya, Laravel adalah kerangka kerja MVC. Namun, ada banyak fitur yang ada dalam aplikasi Lelang Online ini yang tidak terakomodasi dalam MVC, misal sebagai berikut :
		\begin{enumerate}
			\item Sistem Verifikasi lewat Email - yang berarti aplikasi harus berinteraksi dengan SMTP server
			\item Sistem \textit{Generate} Token JWT.io, dimana dalam proses \textit{Generate Token} sama sekali tidak ada database dilibatkan.
		\end{enumerate}
		\indent Jika fitur-fitur tersebut 'dipaksa' dimuat ke dalam MVC, maka tentu saja strukturnya menjadi ganjil, dan muncul \textit{code smell} berikut :
		\begin{enumerate}
			\item \textit{Large Class}, dimana terdapat satu buah file yang sangat panjang (biasanya merupakan entitas utama, dalam hal ini contohnya barang/\textit{item})
			\item \textit{Inappropriate Intimacy}, dimana terdapat satu kelas yang menyimpan \textit{logic} yang tidak seharusnya ia simpan
			\item \textit{Duplicated Code}
		\end{enumerate}		
		\indent Dari hasil analisa ini, penulis mengidentifikasikan strategi-strategi yang akan diterapkan dalam rancangan struktur aplikasi pada subbab \ref{software-structure}, yaitu sebagai berikut:
			\begin{enumerate}
				\item Penggunaan Repository Pattern
				\\ Memisahkan antara Data Processing Layer dan View Layer - agar lebih rapi, terstruktur, hal ini juga dapat menghindari \textit{Duplicated Code}.
				\item Penambahan Komponen : Service dan Provider \\
				Untuk memisahkan \textit{logic} aplikasi yang terkait dengan akses\textit{eksternal services}. Tujuannya, agar jika kedepannya terdapat perbaikan fitur/penambahan fitur, lebih mudah melakukan \textit{traceback} terhadap file/kelas yang bertanggungjawab terhadap fitur tersebut.
			\end{enumerate}
	
	\subsubsection{Analisa Aplikasi Serupa}
	\label{alasan-app-serupa}
		Selama pengerjaan aplikasi, penulis menganalisa aplikasi serupa. Penulis menemukan aplikasi yang kurang lebih alur bisnis/alur penggunaan aplikasinya serupa yaitu : Carousell. Penulis melihat beberapa kesamaan antara sifat transaksi aplikasi tugas akhir saya dengan aplikasi tersebut, yaitu:
		\begin{enumerate}
			\item Sama-sama tidak mengakomodasi pembayaran
			\item Sama-sama tidak adanya kepastian harga (bedanya, pada Carousell yang terjadi adalah \textit{bargaining}
		\end{enumerate} \
		\indent Sehingga dalam alur proses nya, banyak diadaptasi dari Carousell, agar pengguna dapat lebih familiar dan \textit{predictability}nya lebih tinggi jika diadaptasi dari \textit{E-commerce} lainnya yang lebih umum digunakan oleh pengguna.
		
	
	\subsubsection{Analisa Aplikasi Serupa}
	\label{alasan-app-serupa}
	Selama pengerjaan aplikasi, penulis menganalisa aplikasi serupa. Penulis menemukan aplikasi yang kurang lebih alur bisnis / alur penggunaan aplikasinya serupa yaitu : Carousell. \\
	\indent Penulis melihat ada beberapa kesamaan antara sifat transaksi aplikasi tugas akhir saya dengan aplikasi tersebut, yaitu :
	\begin{enumerate}
		\item Sama-sama tidak mengakomodasi pembayaran
		\item Sama-sama tidak adanya kepastian harga (bedanya, pada Carousell yang terjadi adalah \textit{bargaining}
	\end{enumerate}
	\indent Sehingga dalam alur proses nya, banyak diadaptasi dari Carousell, agar pengguna dapat lebih familiar dan \textit{predictability}nya lebih tinggi jika diadaptasi dari \textit{E-commerce} lainnya yang lebih umum digunakan oleh pengguna.
	
	\subsubsection{Analisa Penyimpanan Data}
	
	Untuk penyimpanan data, terdapat 2 jenis data yang sifatnya cukup berbeda, yaitu sebagai berikut:
	\begin{enumerate}
		\item \textbf{Data transaksional disimpan di DBMS SQL - \textit{Relational}}
		\newline
		\indent Data yang sifatnya \textit{transaksional}, seperti data \textit{bidding}, data pengguna, dan lain sebagainya.
		Untuk data ini, lebih baik jika menggunakan database Postgre, untuk menjaga integritas data dan \textit{integrity checking} juga  menjadi lebih baik.
		\item \textbf{Data non-transaksional disimpan di DBMS NoSQL}
		\newline
		\indent Data \textit{chatting}, data \textit{joined rooms} kurang tepat jika disimpan dalam database transaksional karena sifat pertambahan datanya yang sangat cepat, masif dan urgensi integritas data tidak terlalu diprioritaskan (dibanding dengan data transaksional pada poin sebelumnya). Oleh karena itu, baiknya data ini disimpan pada database NoSQL dengan alasan-alasan sebagai berikut.
		\begin{itemize} 
			\item Banyaknya transaksi \textit{read write};
			\item Ketidaksamaan frekuensi \textit{read and write} data semua pengguna;
			\item Sifat permintaan transaksi yang cepat; dan
			\item Kemungkinan perubahan struktur atribut pada pesan (misal: \textit{attachments}, \textit{forwarding}, \textit{replying}, dll) akan sangat menyulitkan pengembangan selanjutnya jika menggunakan database transaksional yang terpaku pada skema database yang ditetapkan di awal pengembangan aplikasi.
		\end{itemize} 	
		
		\item \textbf{Data citra/gambar menggunakan layanan Pihak Ketiga} \newline
		\indent Sekarang telah banyak penyedia jasa \textit{cloud computing} sebagai infrastruktur, seperti Amazon Web Service, Google Cloud Storage Google Alasan-alasan menggunakan AWS sebagai data storage untuk gambar adalah sebagai berikut :
		\begin{enumerate}[noitemsep,topsep=0pt]
			\item Skalabilitas aplikasi lebih terjaga. 
			\newline Dengan memisahkan penyimpanan antara gambar dan server sehingga lebih mudah me\textit{maintain} perkembangan aplikasi, dan lebih fokus terhadap pengembangan aplikasi.
			\item Menyediakan \textit{built-in} keamanan, fleksibel dan efisiensi \cite{wikipedia_amazon_2016}
		\end{enumerate}
		
		\item \textbf{Optimasi \textit{assets}} \newline
		\indent Dalam banyak kesempatan, penulis seringkali mendapati bahwa \textit{delay} untuk \textit{loading assets} lebih lama daripada \textit{loading data} dari database. Berikut penulis akan memaparkan hasil analisa berupa penyebab dan \textit{tackling} permasalahan tersebut.
			\begin{enumerate}
				\item \textit{Useless assets} yang disertakan dalam halaman : Memisahkan \textit{essentials assets} dan menyertakan \textit{script} yang hanya digunakan oleh halaman tersebut.
				\item Logika penyusunan script yang tidak efektif dan optimal (misal: ada satu script yang menyertakan file yang tidak diperlukan) : dilakukan \textit{pre-processing} berupa \textit{minifying, optimization, compiling, compression} terhadap \textit{assets}
				\item Latensi ke server yang cukup tinggi (misal: kecepatan sambungan internet yang rendah) : \textit{Caching}, \textit{upgrading server} agar dapat "lebih terjangkau" secara jaringan, penerapan PWA (\textit{Progressive Web Apps}) untuk sisi \textit{user experience}.
			\end{enumerate}
	\end{enumerate}