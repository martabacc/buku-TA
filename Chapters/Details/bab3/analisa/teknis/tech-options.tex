
\subsection{\textit{Technology Options}}
\label{tech-options}

	Pada subbab sebelumnya, penulis sudah memaparkan arsitektur dasar yang dibutuhkan dalam rancang bangun aplikasi. Terkait dengan arsitektur dasar tersebut, banyak pilihan teknologi yang dapat mengimplementasikan arsitektur tersebut. Dalam pemaparan selanjutnya, akan dijelaskan alasan penulis \textit{best practice} dalam pemilihan teknologi yang digunakan, didasarkan pada \textit{best practices} dan pengalaman-pengalaman penulis. Keterkaitan dengan aspek-aspek yang dijelaskan sebelumnya pada subbab \ref{tech-analysis}.
	
	\subsubsection{\textsc{Nginx} sebagai Web Server}
		\begin{itemize}
			\item Kelebihan
				\begin{enumerate}
					\item Konfigurasi yang lebih \textit{friendly} dan terstruktur
					\item Ketersediaan fitur yang lengkap \& krusial (\textit{reverse proxy}, memungkinkan skalabilitas \& \textit{load balancing})
					\item \textit{Learning-gap} yang kecil terhadap pengalaman penulis/sudah familiar
				\end{enumerate}
			\item Opsi lainnya
				\begin{enumerate}
					\item Apache2:  Fiturnya kurang lengkap
					\item Node.js:  \textit{Learning-gap} yang besar bagi penulis/belum familiar
					\item Python:  Belum Familiar, dan perlu eksplorasi fitur lebih dalam
				\end{enumerate}
		\end{itemize}
	
	\subsubsection{\textsc{PostgreSQL} untuk Penyimpanan Data}
	\begin{itemize}
		\item Kelebihan
		\begin{enumerate}
			\item \textit{Learning gap} yang kecil
			\item Stabil karena telah digunakan dan dikembangkan oleh banyak \textit{developer} selama bertahun-tahun
		\end{enumerate}
		\item Opsi lainnya
		\begin{enumerate}
			\item SQL Server:  Instalasi yang kompleks, penggunaan \textit{resource} yang cukup besar
		\end{enumerate}
	\end{itemize}
	
	\subsubsection{\textsc{MongoDB} untuk Penyimpanan Data Nontansaksional}
		\begin{itemize}
			\item Kelebihan
			\begin{enumerate}
				\item \textit{Learning curve} yang mudah/sintaksnya kurang lebih sama dengan sintaks \textit{database} transaksional pada umumnya
				\item Performa yang cepat karena menggunakan BSON
				\item Fitur yang lengkap untuk \textit{sustainability} aplikasi seperti (Replikasi, Sharding, dll)
				\item \textit{Handling} terhadap data yang sangat besar yang cukup bagus, cocok untuk data yang masif seperti \textit{chatting}.
			\end{enumerate}
			\item Opsi lainnya
			\begin{enumerate}
				\item Redis:  Cepat, namun penyimpanan dilakukan di RAM sehingga lebih cocok untuk penyimpanan \textit{auth session}, bukan untuk penyimpanan data yang sifatnya masif
				\item Cassandra:  \textit{Learning-gap} yang besar, namun fiturnya lengkap untuk \textit{data mining}
			\end{enumerate}
		\end{itemize}
		
		
	\subsubsection{\textsc{CDN} sebagai \textit{Assets Sources}}
		\begin{itemize}
			\item Kelebihan
			\begin{enumerate}
				\item Akses cepat karena besar kemungkinan asset tersebut telah di\textit{cache} sebelumnya dalam browser pengguna
				\item Mengurangi \textit{bandwith} server
				\item Telah dioptimasi oleh pengembang masing-masing asset.
			\end{enumerate}
			\item Opsi lainnya
			\begin{enumerate}
				\item Disimpan dalam server: Mengurangi \textit{bandwith} server (\textit{cost} meningkat)
			\end{enumerate}
		\end{itemize}
		
	\subsubsection{\textsc{AWS S3} untuk \textit{Content Growth Scalability}}
	\begin{itemize}
		\item Kelebihan
		\begin{enumerate}
			\item \textit{Benefit} yang sangat \textit{krusial}:  keamanan, skalabilitas, \textit{availability} - karena sudah di\textit{handle} langsung oleh pengembang \textit{cloud computing} yang ahli di bidangnya
			\item Perkembangan jumlah konten yang akan disimpan (gambar barang yang diupload pengguna)  tentunya bersifat sangat masif, sehingga tidak mungkin disimpan dalam server
		\end{enumerate}
		\item Opsi lainnya
		\begin{enumerate}
			\item Disimpan dalam server: Mengurangi performa server karena sifatnya yang memakan \textit{resource} cukup banyak, dan menambah \textit{cost} untuk \textit{upgrade server storage}
		\end{enumerate}
	\end{itemize}
	
	\subsubsection{\textsc{SendGrid} untuk SMTP Relay}
	\begin{itemize}
		\item Kelebihan
		\begin{enumerate}
			\item Konfigurasi yang mudah
			\item Dokumentasi yang cukup lengkap dan mudah ditemukan
			\item Fitur yang lengkap
			\item Adanya \textit{free storage} dari akun Github Student Pack penulis
		\end{enumerate}
		\item Opsi lainnya
		\begin{enumerate}
			\item MailChimp:  Dokumentasi kurang lengkap, tidak ada \textit{free storage} untuk akun penulis
		\end{enumerate}
	\end{itemize}
	
	\subsubsection{\textsc{Vue.js} untuk \textit{Workloads Sharing}}
		\begin{itemize}
			\item Kelebihan
			\begin{enumerate}
				\item \textit{Learning-gap} relatif kecil dibandingkan \textit{Javascript tools} lainnya, karena didesain khusus untuk Laravel
				\item Adanya program utilitas (webpack) yang membuat performa Vue.js jauh lebih cepat
				\item Logika aplikasi dapat di\textit{obfuscate} dengan webpack (\textit{embedded} dalam Laravel)
			\end{enumerate}
			\item Opsi lainnya
			\begin{enumerate}
				\item React:  \textit{Learning gap} dan \textit{learning curve} yang sangat besar untuk penulis
				\item jQuery:  tidak efektif karena \textit{code smells} yang ditimbulkan cukup banyak
			\end{enumerate}
		\end{itemize}
		
	\subsubsection{\textsc{Socket.io} untuk Mekanisme Asinkronus}
		\begin{itemize}
			\item Kelebihan
			\begin{enumerate}
				\item \textit{Learning-gap} relatif kecil dibandingkan \textit{Javascript tools} lainnya, karena didesain khusus untuk Laravel
				\item Adanya program utilitas (webpack) yang membuat performa Vue.js jauh lebih cepat
				\item Logika aplikasi dapat di\textit{obfuscate} dengan webpack (\textit{embedded} dalam Laravel)
			\end{enumerate}
			\item Opsi lainnya
			\begin{enumerate}
				\item React:  \textit{Learning gap} dan \textit{learning curve} yang sangat besar untuk penulis
				\item jQuery:  tidak efektif karena \textit{code smells} yang ditimbulkan cukup banyak
			\end{enumerate}
		\end{itemize}		
		
	\subsubsection{\textsc{JWT} untuk Keamanan Soket}
		\begin{itemize}
			\item Kelebihan
			\begin{enumerate}
				\item \textit{Library support} yang lengkap untuk komponen-komponen lainnya
				\item Efektif dan efisien karena tidak ada \textit{query} ke database untuk autentikasi
			\end{enumerate}
			\item Opsi lainnya
			\begin{enumerate}
				\item \textit{Query} ke \textit{database} secara konvensional:  Sifat koneksi soket yang masif akan sangat memberatkan \textit{database} jika setiap kali ada koneksi baru, harus melakukan \textit{query database} sehingga tidak efektif
				\item \textit{Session caching} dengan Redis:  \textit{Learning gap} yang besar
			\end{enumerate}
		\end{itemize}
		
	\subsubsection{\textsc{Laravel Dusk} untuk \textit{Functionality Testing Script}}
		\begin{itemize}
			\item Kelebihan
			\begin{enumerate}
				\item \textit{Learning gap} yang kecil karena didesain sefamiliar mungkin dengan Laravel
			\end{enumerate}
			\item Opsi lainnya
			\begin{enumerate}
				\item Selenium:  \textit{Learning curve} yang besar
				\item Phantom.js:  \textit{Learning curve} yang besar
			\end{enumerate}
		\end{itemize}
	
	

	
	
	