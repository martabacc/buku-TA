    
\subsection{Perancangan \textit{Contents, Assets and Data Storages}}

	\begin{enumerate}
    \item \textbf{Data Citra/Gambar disimpan di \textsc{Amazon Web Services}} \newline
    \indent Gambar-gambar barang yang didaftarkan untuk dilelang, disimpan di \textit{cloud} dengan menggunakan \textit{Amazon Web Services}. Alasan-alasan menggunakan AWS sebagai data storage untuk gambar adalah sebagai berikut :
        \begin{enumerate}[noitemsep,topsep=0pt]
        \item Skalabilitas aplikasi lebih terjaga. 
        \newline Dengan memisahkan penyimpanan antara gambar dan server sehingga lebih mudah me\textit{maintain} perkembangan aplikasi, dan lebih fokus terhadap pengembangan aplikasi.
        \item Menyediakan \textit{built-in} keamanan, fleksibel dan efisiensi \cite{wikipedia_amazon_2016}
        \item Mencoba belajar menggunakan Amazon Web Services
        \end{enumerate}
        
    \item \textbf{Data \textit{Assets} Website}
    \newline
    \indent Untuk data \textit{assets} yang dibutuhkan untuk website, terdapat beberapa kriteria penyimpanan berikut
    
      \begin{itemize}[noitemsep,topsep=0pt]
      \item Jika file tersebut sudah umum digunakan dan terdapat file CDNnya, maka akan akses CDNnya
      \newline
      Hal ini dimaksudkan agar \textit{loading} lebih cepat, sesuai dengan yang tercantum pada sumber \cite{sitepoint_7_2011}
      \item Jika file tersebut merupakan \textit{custom asset}, \textit{asset} yang dikustomisasi khusus untuk aplikasi ini, maka asset tersebut akan disimpan dalam server.
      \end{itemize}
    \end{enumerate}