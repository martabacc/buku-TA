% Melihat riwayat harga yang ditawarkan pada barang yang dilelang

\begin{table}[H]
	\centering
	\begin{tabular}{|r|p{8cm}|}
		\hline
		\textbf{Kode}                                                    & UC-03.04                                                     \\ \hline
		\textbf{Nama}                                                    & \textbf{Melihat Detail Riwayat Penawaran Harga} \\ \hline
		\textbf{Aktor}                                                   & Pengguna 
		\\ \hline
		\textbf{Deskripsi}                                               & Pengguna hendak melihat daftar semua barang yang pernah didaftarkan dalam sistem.
		\\ \hline
		\textbf{Tipe}                                                    & Fungsional 
		\\ \hline
		\textbf{\textit{Precondition}}
		& Informasi daftar barang belum ditampilkan. \\ \hline
		\textbf{\textit{Postcondition}} 
		& Informasi daftar barang sudah ditampilkan. \\ \hline
		\multicolumn{2}{|c|}
		{\textbf{Alur Kejadian Normal}}                                                                            \\ \hline
		\multicolumn{1}{|l|}{}                                           & 
		\begin{enumerate}
			\item Pengguna dalam keadaan terautentikasi, mengklik "Item Anda" -> "Manage Items" pada \textit{navbar} bagian atas halaman.
			\item Sistem menampilkan halaman yang berisi daftar barang yang didaftarkan pengguna.
			\item Pengguna mengklik barang yang ingin dilihat daftar penawaran harganya
			\item Sistem menampilkan halaman detail informasi barang
			\item Pengguna mengklik tombol "Lihat Riwayat Penawaran"
			\item Sistem menampilkan halaman berisi daftar riwayat penawaran.
		\end{enumerate}
		\\ \hline
		\multicolumn{2}{|c|}{\textbf{Alur Kejadian Alternatif}}                                                         \\ \hline
		\multicolumn{1}{|l|}{}                                           & -
		\\ \hline
	\end{tabular}
	\caption{Spesifikasi Kasus Penggunaan : Melihat Riwayat Penawaran Harga}
	\label{uc03.04}
\end{table}