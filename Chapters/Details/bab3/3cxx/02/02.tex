% Mencari Barang Lelang

\begin{table}[H]
	\centering
	\begin{tabular}{|r|p{8cm}|}
		\hline
		\textbf{Kode}                                                    
			& UC-02.02                                                     
			\\ \hline
		\textbf{Nama}                                                    
			& \textbf{Mencari Barang Lelang}                                         
			\\ \hline
		\textbf{Aktor}                                                   
			& Pengguna                                                    
			\\ \hline
		\textbf{Deskripsi}
			& Pengguna ingin mencari barang lelang dengan kriteria nama tertentu
			\\ \hline
		\textbf{Tipe}                                                    
			& Fungsional
			\\ \hline
		\textbf{\textit{Pre Condition}}
			& Pengguna menemukan barang lelang yang ia cari dengan kriteria \textit{string}.
			\\ \hline
		\textbf{\textit{Post Condition}}
			& Pengguna menemukan barang lelang yang ia cari dengan kriteria \textit{string} masukan.
			\\ \hline
			\multicolumn{2}{|c|}
			{\textbf{Alur Kejadian Normal}} 
			\\ \hline
		\multicolumn{1}{|l|}{} 
			\begin{enumerate}
				\item Pengguna memasukkan kriteria \textit{string }pencarian di \textit{field} masukan di \textit{Header Bar}
				\item Setelah selesai, pengguna mengklik tombol "Cari"
				\item Sistem mencari barang terdaftar yang sesuai dengan kriteria masukan pengguna
				\item \label{uc0202-a} Jika ketemu, sistem menampilkan halaman "Hasil Pencarian" beserta barang yang sesuai dengan kriteria pengguna.
				\item Pengguna lalu mengklik barang yang sesuai dengan keinginan
				\item Sistem menampilkan detail barang yang sesuai dengan keinginan pengguna
			\end{enumerate}
			\\ \hline
		\multicolumn{2}{|c|}
			{\textbf{Alur Kejadian Alternatif}} 
			\\ \hline
		\multicolumn{1}{|l|}{}                                           
			& \textbf{Tidak ada barang terdaftar dalam sistem yang sesuai dengan kriteria pengguna.}
			\\ \hline
		\multicolumn{1}{|l|}{} & 
			\begin{itemize}
				\item[\ref{uc0202-a}]a. Sistem tidak dapat menemukan barang yang sesuai
				\item[\ref{uc0202-a}]b. Sistem menampilkan "Hasil Pencarian" namun dengan keterangan "Hasil pencarian kosong"
			\end{itemize}
			\\ \hline
	\end{tabular}
	\caption{Spesifikasi Kasus Penggunaan : Mencari Barang Lelang}
	\label{uc02.02}
\end{table}	
