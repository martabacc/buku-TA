% Melihat Riwayat Penawaran Lelang Barang

\begin{table}[H]
	\centering
	\caption{Spesifikasi Kasus Penggunaan : Melihat Riwayat Penawaran Lelang Barang}
	\label{uc02.04}
	\begin{tabular}{|r|p{8cm}|}
		\hline
		\textbf{Kode}                                                    
		& UC-02.04                                                    
		\\ \hline
		\textbf{Nama}                                                    
		& \textbf{Melihat 3 Riwayat Penawaran Lelang Barang Teratas}                                         
		\\ \hline
		\textbf{Aktor}                                                   
			& Pengguna                                                    
			\\ \hline
		\textbf{Deskripsi}
			& Pengguna ingin melihat riwayat penawaran lelang terhadap barang yang ia daftarkan
			\\ \hline
		\textbf{Tipe}                                                    
			& Fungsional
			\\ \hline
		\textbf{\textit{Pre Condition}}
			& Pengguna menemukan barang lelang yang ia cari dengan kriteria \textit{string}.
			\\ \hline
		\textbf{\textit{Post Condition}}
			& Pengguna menemukan barang lelang yang ia cari dengan kriteria \textit{string} masukan.
			\\ \hline
		\multicolumn{2}{|c|}{\textbf{Alur Kejadian Normal}}
			\\ \hline
		\multicolumn{1}{|l|}{} & 
			\begin{enumerate}
				\item Pengguna membuka halaman "Kelola Barang"
				\item Pengguna mengklik barang yang ingin dilihat informasi riwayat penawaran lelangnya
				\item Sistem menampilkan halaman informasi barang tersebut
				\item Pengguna mengklik tombol "Lihat Penawaran Teratas"
				\item Sistem menampilkan \textit{modal} berisi 3 riwayat penawaran lelang teratas barang tersebut.
				%\item Pengguna memasukkan kriteria \textit{string }pencarian di \textit{field} masukan di \textit{Header Bar}
				%\item \label{uc0202-a} Jika ketemu, sistem menampilkan halaman "Hasil Pencarian" beserta barang yang sesuai dengan kriteria pengguna.
			\end{enumerate}
			\\ \hline
		\multicolumn{2}{|c|}
		{\textbf{Alur Kejadian Alternatif}} 
		\\ \hline
		\multicolumn{1}{|l|}{}                                           
		& -
		\\ \hline
	\end{tabular}
\end{table}	