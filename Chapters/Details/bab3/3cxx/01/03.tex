\begin{table}[H]
	\centering
	\caption{Spesifikasi Kasus Penggunaan: Konfirmasi Email}
	\label{uc01.03}
	\begin{tabular}{|r|p{8cm}|}
		\hline
		\textbf{Kode}                                                    & 	UC-01.03                                                     \\ \hline
		\textbf{Nama}                                                    & \textbf{Konfirmasi Email}                                         \\ \hline
		\textbf{Aktor}
			& Pengguna                                                    \\ \hline
		\textbf{Deskripsi}                                               & Pengguna melakukan konfirmasi email agar status akun pengguna menjadi teraktivasi \\ \hline
		\textbf{Tipe}
			& Fungsional                                                  \\ \hline
		\textbf{\textit{Precondition}}
			& Status akun pengguna masih belum terverifikasi \\ \hline
		\textbf{\textit{Postcondition}} 
			&  Status akun pengguna masih sudah	 terverifikasi \\ \hline
		\multicolumn{2}{|c|}{\textbf{Alur Kejadian Normal}}                                                                            \\ \hline
		\multicolumn{1}{|l|}{}                                           & 
			\begin{enumerate}
				\item Pengguna membuka halaman \textit{inbox} email pengguna di \textit{sistem email service} yang mereka gunakan.
				\item pengguna mencari dan membuka email konfirmasi yang dikirimkan oleh Lelangapa
				\item Sistem \textit{email service} pengguna menampilkan isi email konfirmasi, beserta sebuah tombol "Konfirmasi email"
				\item Pengguna mengklik tombol "Konfirmasi Email"
				\item Halaman \textit{browser} akan di\textit{redirect} ke URL konfirmasi email
				\item Sistem menampilkan halaman \textit{landing page} dimana status akun pengguna sudah terverifikasi.
			\end{enumerate}
		\\ \hline
		\multicolumn{2}{|c|}{\textbf{Alur Kejadian Alternatif}}                                                         \\ \hline
		\multicolumn{1}{|l|}{}                                           & -
		\\ \hline
	\end{tabular}
\end{table}