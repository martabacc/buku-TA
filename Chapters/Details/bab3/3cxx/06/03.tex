	% Memasukkan kupon
	
	\begin{table}[H]
		\centering
		\begin{tabular}{|r|p{8cm}|}
			\hline
			\textbf{Kode}
			& UC-05.01
			\\ \hline
			\textbf{Nama}
			& \textbf{Memperbarui Kupon} 
			\\ \hline
			\textbf{Aktor}    
			& Administrator 
			\\ \hline
			\textbf{Deskripsi}
			& Administrator ingin memperbarui kupon
			\\ \hline
			\textbf{Tipe}
			& Fungsional 
			\\ \hline
			\textbf{\textit{Precondition}}
			& Kupon belum diperbarui
			\\ \hline
			\textbf{\textit{Postcondition}} 
			& Kupon berhasil diperbarui
			\\ \hline
			\multicolumn{2}{|c|}
			{\textbf{Alur Kejadian Normal}}
			\\ \hline
			\multicolumn{1}{|l|}{} & 
			\begin{enumerate}
				\item Administrator membuka halaman 'Manajemen Kupon'
				\item Sistem menampilkan halaman form Edit Kupon pada sesuai dengan data kupon yang ingin diperbarui
				\item Administrator memasukkan informasi kupon baru yang ingin ditambahkan, lalu ketik tombol "Submit"
				\item Sistem mengecek informasi baru kupon kupon,lalu sistem me\textit{redirect} ke halaman manajemen kupon dengan info sukses.
				% \item \label{uc0301-show1page}Sistem menampilkan halaman yang berisi form pendaftaran barang
				% \item \label{al-0301-a} Sistem memvalidasi data yang dimasukkan pengguna
			\end{enumerate}
			\\ \hline
			\multicolumn{2}{|c|}{\textbf{Alur Kejadian Alternatif}} \\ \hline
			\multicolumn{1}{|l|}{}                   
			& -
			\\ \hline
		\end{tabular}
		\caption{Spesifikasi Kasus Penggunaan : Menambahkan Kupon}
		\label{uc06.03}
	\end{table}