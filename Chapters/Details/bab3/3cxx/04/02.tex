	% Menambahkan Review Pengguna
	
	
	\begin{table}[H]
		\centering
		\begin{tabular}{|r|p{8cm}|}
			\hline
			\textbf{Kode}                                                    
			& UC-04.02
			\\ \hline
			\textbf{Nama}                                                    
			& \textbf{ Menambahkan \textit{Review} Pengguna } 
			\\ \hline
			\textbf{Aktor}                                                   
			& Pengguna 
			\\ \hline
			\textbf{Deskripsi}                                               
			& Pengguna ingin menambahkan \textit{review} dari \textit{transaksi} yang pernah dilakukan.
			\\ \hline
			\textbf{Tipe}
			& Fungsional 
			\\ \hline
			
			\textbf{\textit{Precondition}}
			& \textit{Review} dari pengguna belum tercatat/tersimpan dalam sistem

			\\ \hline
			
			\textbf{\textit{Postcondition}} 
			& \textit{Review} dari pengguna berhasil tercatat dalam sistem
			\\ \hline
			
			\multicolumn{2}{|c|}
			{\textbf{Alur Kejadian Normal}}                                                                            
			\\ \hline
			\multicolumn{1}{|l|}{} & 
			\begin{enumerate}
				\item Pengguna mengklik halaman 'Riwayat Transaksi'
				\item Sistem menampilkan halaman Riwayat Transaksi yang pernah dilakukan pengguna
				\item Pengguna mengklik \textit{tab} jenis transaksi yang pernah dilakukan (Beli atau Lelang)
				\item \label{uc0402-show1page}Sistem menampilkan riwayat transaksi sesuai dengan jenis transaksi yang dipilih pengguna
				\item Pengguna mengklik transaksi yang ingin diberikan \textit{review}
				\item \label{al-0402-a}Sistem mengecek apakah \textit{review} sudah pernah diberikan sebelumnya
				\item Sistem menampilkan \textit{modal} berisi \textit{field input} jumlah \textit{rating}
				\item Pengguna mengisi \textit{field} tersebut sesuai jumlah rating yang ingin diberikan
				\item Setelah selesai, pengguna klik 'Next'
				\item Sistem menampilkan \textit{modal} kedua, berisikan \textit{field input} untuk deskripsi \textit{review}
				\item Pengguna mengisikan \textit{field input} sesuai dengan deskripsi yang ingin diberikan
				\item Setelah selesai, pengguna mengklik tombol 'Simpan Review'
				\item \label{al-0402-b}Sistem memvalidasi masukan dari pengguna
				\item Jika tervalidasi, sistem menampilkan modal berisi informasi sukses menyimpan review
				\item Pengguna klik tombol 'Oke'
				\item Sistem menutup \textit{modal} dan kembali ke halaman di poin \ref{uc0402-show1page}
				
				% \item \label{uc0301-show1page}Sistem menampilkan halaman yang berisi form pendaftaran barang
				% \item \label{al-0301-a} Sistem memvalidasi data yang dimasukkan pengguna
			\end{enumerate}
			\\ \hline
			\multicolumn{2}{|c|}{\textbf{Alur Kejadian Alternatif}} \\ \hline
			\multicolumn{1}{|l|}{}                                   \pagebreak        
			& \textbf{Review untuk transaksi yang dipilih sudah pernah diberikan}
			\\ \hline
			\multicolumn{1}{|l|}{}& 
			\begin{itemize}
				\item[\ref{al-0402-a}a.] Sistem mendeteksi bahwa review untuk transaksi tersebut sudah pernah dilakukan
				\item[\ref{al-0402-a}b.] Sistem menampilkan modal berisi informasi bahwa review sudah pernah diberikan, dan \textit{auto-close modal} setelah 4 detik
				\item[\ref{al-0402-a}c.] Sistem menampilkan halaman di poin \ref{uc0402-show1page}
			\end{itemize}
			\\ \hline
			
			\multicolumn{1}{|l|}{}      
			& \textbf{Data masukan pengguna tidak valid}
			\\ \hline
			\multicolumn{1}{|l|}{}& 
			\begin{itemize}
				\item[\ref{al-0402-b}a.] Sistem mendeteksi masukan pengguna tidak valid.
				\item[\ref{al-0402-b}b.] Sistem menampilkan pesan Error berisi \textit{error message}
				\item[\ref{al-0402-b}c.] Sistem menampilkan halaman di poin \ref{uc0402-show1page}
			\end{itemize}
			\\ \hline
		\end{tabular}
		\caption{Spesifikasi Kasus Penggunaan : Memberikan \textit{Review} Pengguna}
		\label{uc04.02}
	\end{table}