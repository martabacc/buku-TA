\section{Pengujian}
	Pada subbab ini, penulis akan memaparkan pengujian terhadap aplikasi. Pengujian yang dilakukan adalah pengujian fungsionalitas, dimana penulis menggunakan tool Laravel Dusk sebagai \textit{testing code} untuk menguji fungsionalitas aplikasi.\\
	\tabularnewline Dikarenakan keterbatasan waktu, dan atas saran dari pembimbing, penulis tidak menuliskan \textit{testing script}	untuk keseluruhan fungsionalitas yang sudah pasti teruji, seperti Login (sudah menggunakan \textit{facade} Laravel), transaksi CRUD dll.\\
	Pada pemaparan ini, penulis mengidentifikasikan fungsionalitas utama dalam aplikasi lelang ini adalah sebagai berikut :
	\begin{enumerate}
		\item Pengujian Fungsionalitas Lelang
			  \begin{enumerate}
			  	\item Pengujian Penawaran Lelang
			  \end{enumerate}
	   \item Pengujian Fungsionalitas Voucher
			  \begin{enumerate}
			  	\item Pengujian Penggunaan Voucher
			  \end{enumerate}
	\end{enumerate}
	
	Pada bagian ini juga, penulis menuliskan \textit{summary} pengujian fungsionalitas ini pada .
	
	\subsection{Pengujian Fungsionalitas Lelang}
		Pada pengujian ini, terdapat beberapa skenario pengujian yang dipaparkan dalam tabel berikut :
		
	
	\subsection{Pengujian Fungsionalitas Voucher}
		Pada pengujian ini, terdapat beberapa skenario pengujian yang dipaparkan dalam tabel berikut :
	
	
	\subsection{\textit{Summary} Pengujian}
		
		