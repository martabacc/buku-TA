Rekapitulasi pengujian pengguna/\textit{user test} adalah sebagai berikut:

\indent Rekapitulasi tersebut diurutkan berdasarkan besarnya/signifikansi persentase perbedaan antara kedua platform, dapat dilihat pada tabel \ref{user-test-recap-sorted}.
\LTXtable{\textwidth}{table/05/pengguna/recap_sorted}

Visualisasi perbandingan dapat dilihat lewat diagram garis pada gambar \ref{diagram-pengguna-chart}.

\begin{figure}[H]
	\centering
	\includegraphics[width=\textwidth]{images/bab5/ujipengguna/chart.png}
	\caption{Diagram perbandingan pengujian \textit{user experience} pengguna}
	\label{diagram-pengguna-chart}
\end{figure}

Dalam hal ini, dapat disimpulkan bahwa impresi \textit{user experience} sudah baik, dan skor \textit{user experience}nya sedikit diatas sistem serupa lainnya. Perbedaannya sudah cukup signifikan adalah \textit{recommendation}, kemudahan penggunaan dan desain web yang baik. Namun, yang menjadi perhatian adalah perbedaan \textit{performa} yang masih sangat kecil. Hal ini terkait dengan pengujian \textit{speed test} di subbab selanjutnya.