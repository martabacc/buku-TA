\begin{longtable}{|X|X|}
		\caption{Pengujian Fungsionalitas Fitur Manajemen Penawaran}
		\label{uji-fungsional-2-penawaran}
	\\
	
	\hline
		\textbf{ID} & \textbf{UJ-1-KP02} \\ \hline
	\endfirsthead
	
	\multicolumn{2}{r}%
	{\tablename\ \thetable{} -- lanjutan dari halaman sebelumnya} \\
	\hline 
		\textbf{ID} & \textbf{UJ-1-KP02} \\ \hline
	\endhead
	
	\hline \multicolumn{2}{|r|}{{Dilanjutkan ke halaman selanjutnya}} \\ \hline
	\endfoot
	
	\hline
	\endlastfoot
	
	\textbf{Referensi Kasus Penggunaan}
		& KP02 \\ \hline
	\textbf{Nama}
		& Pengujian fitur manajemen penawaran \\ \hline
	\textbf{Skenario 1}
		& Menguji fitur melihat barang yang sedang aktif dilelang \\ \hline
	Kondisi Awal
		& Sistem \\ \hline
	Data Uji
		& Data uji menggunakan data penulis \\ \hline
	Langkah pengujian
		& Membuka halaman website aplikasi lelang online via \textit{browser} di alamat https://lelangapa.com \\ \hline
	Hasil yang Diharapkan
		& Sistem berhasil menampilkan data barang yang sedang aktif dilelang \\ \hline	
	Hasil Pengujian
		& 100\% berhasil \\ \hline	
	Kondisi Akhir
		& \textit{Screenshot} pengujian ini dapat dilihat pada gambar \ref{ss-kp02-01} \\ \hline	

	\textbf{Skenario 2}
		& Menguji fitur mencari barang \\ \hline
	Kondisi Awal
		& Sistem menampilkan halaman dengan elemen \textit{input} search barang \\ \hline
	Data Uji
		&  \\ \hline
	Langkah pengujian
		& \begin{enumerate}
		\item Elemen \textit{input search} diisi dengan kata kunci ``Jersey''
		\item Setelah selesai mengisi, mengklik tombol ``Search''
	\end{enumerate} \\ \hline
	Hasil yang Diharapkan
		& Barang yang mengandung kata ``Jersey'' muncul dalam hasil pencarian \\ \hline
	Hasil Pengujian
		& 100\% berhasil \\ \hline	
	Kondisi Akhir
		& \textit{Screenshot} pengujian ini dapat dilihat pada gambar \ref{ss-kp02-02}  \\ \hline	
		
		
	\textbf{Skenario 3}
		& Menawar barang \\ \hline
	Kondisi Awal
		& Pengguna sedang membuka halaman lelang barang \\ \hline
	Data Uji
		& Data ujinya adalah memasukkan penawaran harga yang lebih tinggi pada penawaran harga saat itu
	Langkah pengujian \\ \hline
		& \begin{enumerate}
		\item Memasukkan harga penawaran
		\item Mengklik tombol ``Tawar''
	\end{enumerate} \\ \hline
	Hasil yang Diharapkan
		& Penawaran yang baru berhasil masuk ke dalam sistem, dan harga terbaru diupdate di halaman secara \textit{realtime} \\ \hline
	Hasil Pengujian
		& 100\% berhasil \\ \hline	
	Kondisi Akhir
		& \textit{Screenshot} pengujian ini dapat dilihat pada gambar \ref{ss-kp02-03}  \\ \hline	
		
\end{longtable}