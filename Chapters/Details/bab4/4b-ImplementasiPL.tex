
  \section{Implementasi Perangkat Lunak}
  Pada subbab ini, penulis akan memaparkan mengenai spesifikasi dan pemasangan perangkat lunak yang dibutuhkan dalam rancang bangun aplikasi lelang online ini.
  
  \subsection{Arsitektur Perangkat Lunak}
  \label{arsitektur-pl-final}
	  Sesuai dengan analisa yang penulis paparkan pada 
	  
	  \begin{figure}[H]
	  	\centering
	  	\includegraphics[height=0.6\paperheight]{images/bab3/diagram/arsitektur-colorized.png}
	  	\caption{Arsitektur Aplikasi Lelang Online Ter\textit{Update} \\
	  		\textit{External Services} artinya adalah menggunakan \textit{service} dari luar, tidak dibangun sendiri. }
	  	\label{arsitektur-app}
	  \end{figure}
  
  \subsection{Pemasangan \& Konfigurasi Perangkat Lunak}
	  Pada subbabb ini, penulis menuliskan tahap-tahap pemasangan dan konfigurasi untuk komponen-komponen yang ada di subbab \ref{arsitektur-pl-final-subbab}.
	  
	\subsubsection{Instalasi \& Konfigurasi \textbf{Nginx}}

	\begin{enumerate}[label={}]
		
		\item \textbf{Instalasi \textbf{Nginx}} 
			\\Tahap-tahap untuk instalasi adalah sebagai berikut
			\begin{enumerate}[label=\roman*]				\item \mylipsum
					\begin{figure}[H]
						\centering
						\includegraphics[width=0.4\textheight]{images/no-image.png}
						\caption{Skrinsut Tahap 1 Instalasi \textbf{Nginx}}
						\label{pdm-final}
					\end{figure}				\item \mylipsum
					\begin{figure}[H]
						\centering
						\includegraphics[width=0.4\textheight]{images/no-image.png}
						\caption{Skrinsut Tahap 2 Instalasi \textbf{Nginx}}
						\label{pdm-final}
					\end{figure}		
						
			\end{enumerate}
		\item 
		\textbf{Konfigurasi \textbf{Nginx}}
		\begin{enumerate}[label=\roman*]	
			\item \mylipsum
			\begin{figure}[H]
				\centering
				\includegraphics[width=0.4\textheight]{images/no-image.png}
				\caption{Skrinsut Tahap 1 Konfigurasi \textbf{Nginx}}
				\label{pdm-final}
			\end{figure}					
			\item \mylipsum
			\begin{figure}[H]
				\centering
				\includegraphics[width=0.4\textheight]{images/no-image.png}
				\caption{Skrinsut Tahap 1 Konfigurasi Nginx \textbf{Nginx}}
				\label{pdm-final}
			\end{figure}	
 	\end{enumerate}
	\input{Chapters/Details/bab4/4b/laravel}
	\input{Chapters/Details/bab4/4b/postgre}
	\input{Chapters/Details/bab4/4b/mongo}
	\subsubsection{\textsc{Vue.js}}
	% todo inline numbering
	
\begin{enumerate}\bfseries
\item \textbf{Package Dependencies} \\
    \indentenum Pada versi terbaru Laravel (5.4*), Laravel secara \textit{default} menyertakan \textit{package} Laravel Mix - yaitu fitur untuk \textit{compiling assets} dengan Webpack, dengan hasil akhir \textit{compiled assets} (terutama \textit{script} Javascript) yang eksekusinya jauh lebih cepat, karena menggunakan V8 -- sebuah \textit{engine} Javascript yang telah dioptimasi yang bersifat \textit{just-in-time} (JIT) yang memproduksi \textit{machine code} dari sebuah \textit{script} Javascript lalu dieksekusi.\\
  
  \textbf{\textit{Main Problem}} \\
  \indentenum Masalah muncul saat versi Laravel yang digunakan untuk membangun aplikasi adalah versi (5.3) -- dan jika Laravelnya di\textit{upgrade}, tidak ada jaminan bahwa \textit{deprecated dependencies} (keadaan dimana sebuah \textit{package} tidak di\textit{support} oleh versi terbaru) -- yang berarti harus \textit{refactoring code} yang pasti memakan waktu lama.\\			  

	\textbf{\textit{Insights}} \\
	  \indentenum Penulis menganalisa perbedaan mendasar package.json antara Laravel 5.3 dan 5.4 adalah sebagai berikut:
	\begin{enumerate}
  	  	\item Basis : Perubahan basis yang awalnya Gulp menjadi Webpack
  	  	\item \textit{Dependencies} : Webpack ternyata menggunakan beberapa plugin tambahan yang tidak diakomodasi dalam package.json di versi 5.3
  	  	\item \textit{Run Script} : Terdapat beberapa perubahan signifikan terhadap \textit{run script alias} di versi 5.4 - dibandingkan pada versi 5.3.
  	  	\item \textit{Additional Files} : Terdapat beberapa file konfigurasi tambahan agar proses kompilasi aset dapat berjalan dengan baik.
	\end{enumerate}
	\ \\
  
  \textbf{\textit{Solution}} \\
    \indentenum Penulis lalu mengoreksi dan \textit{update package.json} dengan pendekatan \textit{trial and error}, dan bisa terselesaikan dengan script berikut :
		  	
\begin{lstlisting}[language=json]
{
	"private": true,
	"scripts": 	{
		"_comment" : "Lists of running npm commands defined here"
	},
	"devDependencies": {
		"axios": "^0.15.3",
		"bootstrap-sass": "^3.3.7",
		"cross-env": "^3.2.3",
		"jquery": "^3.1.1",
		"laravel-mix": "0.*",
		"lodash": "^4.17.4",
		"vue": "^2.1.10"
	},
	"dependencies": {
		"vue-resource": "^1.3.1"
	}
}
 	\end{lstlisting}
			  	
			  
\item \textbf{\textit{Package Dependencies Optimization}}
		\\ \indent
		Langkah-langkah untuk \textit{compiling assets} adalah sebagai berikut :
		\begin{enumerate}
			\item 
		\end{enumerate}
		Masalah optimasi muncul saat beberapa \textit{script} Vue.js ternyata menggunakan \textit{dependencies} yang sama, yaitu axios, vue dan http.
		\\ \indentenum
	 \end{enumerate}
	\input{Chapters/Details/bab4/4b/node}
	\input{Chapters/Details/bab4/4b/socketio}
	\subsubsection{Instalasi \textsc{JWT} pada Laravel dan Node.js}
	\subsubsection{\textit{Whitelisting} pada \textsc{SendGrid}}
	% todo inline numbering
	\textit{Whitelisting} adalah kebalikan dari teknologi \textit{blacklist}. Jika \textit{blacklist} merupakan daftar dari sekumpulan web domain ataupun alamat email, dan URL yang terindikasi tidak “aman” sehingga secara otomatis akan diblokir oleh komputer maupun jaringan agar tidak dapat diakses, maka \textit{whitelist} kebalikan dari \textit{blacklist} yaitu daftar yang diperbolehkan untuk diakses oleh komputer atau jaringan dimana terdapat sekumpulan URL, web domain maupun alamat email yang “aman”.\\
	\indent Masalah dimulai saat \textit{email} konfirmasi akun yang dikirimkan oleh sistem aplikasi lelang online - dengan menggunakan SMTP \textit{relay} - kepada \textit{email pengguna}, masuk ke dalam kotak pesan sebagai \textit{spam}. Hal ini tentu tidak baik - karena seharusnya masuk ke \textit{inbox} sebagaimana \textit{email} pada umumnya. Setelah penulis mencoba mencari jalan keluar, terutama SendGrid tidak menyediakan \textit{whitelisting} dengan menggunakan \textit{root domain} (karena penulis membeli domain lelangapa.com, penulis tidak dapat mengirim \textit{email} konfirmasi akun dengan menggunakan any\_email\_address@lelangapa.com, dan hingga buku ini ditulis penulis tetap tidak tahu alasan pastinya apa). Masalah ini baru selesai dengan menggunakan pemecahan berikut:
	\begin{enumerate}
		\item Membuat sebuah domain noreply.lelangapa.com
		\item Me\textit{register} domain \textit{whitelisting} di pengaturan SendGrid
	\end{enumerate}
	
	\begin{figure}[H]
		\centering
		\includegraphics[width=.8\textwidth]{images/bab4/pl/whitelist-success.png}
		\caption{\textit{Whitelisting} berhasil dijalankan}
		\label{whitelist-success}
	\end{figure}
	
	\indent 
	\begin{figure}[H]
		\centering
		\includegraphics[width=.8\textwidth]{images/bab4/pl/detail-whitelist.png}
		\caption{Detail Informasi Email yang Masuk ke Kotak Masuk Pengguna}
		\label{detail-whitelist}
	\end{figure}
	
	\subsubsection{Instalasi \& Konfigurasi \textbf{Amazon Web Service S3}}

	\begin{enumerate}[label={}]
		
		\item \textbf{Instalasi \textbf{AWS}} 
			\\Tahap-tahap untuk instalasi adalah sebagai berikut
			\begin{enumerate}[label=\roman*]				\item \mylipsum
					\begin{figure}[H]
						\centering
						\includegraphics[width=0.4\textheight]{images/no-image.png}
						\caption{Skrinsut Tahap 1 Instalasi \textbf{AWS}}
						\label{pdm-final}
					\end{figure}				\item \mylipsum
					\begin{figure}[H]
						\centering
						\includegraphics[width=0.4\textheight]{images/no-image.png}
						\caption{Skrinsut Tahap 2 Instalasi \textbf{AWS}}
						\label{pdm-final}
					\end{figure}		
						
			\end{enumerate}
		\item 
		\textbf{Konfigurasi \textbf{AWS}}
		\begin{enumerate}[label=\roman*]
			\item \mylipsum
			\begin{figure}[H]
				\centering
				\includegraphics[width=0.4\textheight]{images/no-image.png}
				\caption{Skrinsut Tahap 1 Konfigurasi \textbf{AWS}}
				\label{pdm-final}
			\end{figure}					
			\item \mylipsum
			\begin{figure}[H]
				\centering
				\includegraphics[width=0.4\textheight]{images/no-image.png}
				\caption{Skrinsut Tahap 1 Konfigurasi Nginx \textbf{AWS}}
				\label{pdm-final}
			\end{figure}	
	 	\end{enumerate}
	\end{enumerate}
	
	 
  
  
  
  
  
  
