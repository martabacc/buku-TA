\subsection{Memperbarui Barang}
Halaman ini dapat diakses oleh semua pengguna, baik yang belum terdaftar maupun sudah. Halaman ini menampilkan form berisi elemen \textit{input} data barang yang sebelumnya sudah terisi data-data barang sebelumnya, dan pengguna dapat memperbarui data lalu mengklik tombol Perbarui, dan untuk kasus alternatif dapat dilihat pada tabel spesifikasi kasus penggunaan \ref{uc03.02}.\\
\indent Pada implementasinya, kasus penggunaan ini kurang lebih sama dengan kasus pengunaan Mendaftarkan Barang (subbab \ref{kasus-penggunaan-daftarbarang}) karena adanya penggunaan layanan pihak ketiga (AWS Storage Service) sebagai tempat penyimpanan gambar. Kode-kode sumber terkait dengan implementasi kasus penggunaan Mendaftarkan Barang untuk Dilelang adalah sebagai berikut:
\begin{enumerate}
	\item Logika \textit{back-end} ditulis menggunakan PHP yang dicantumkan dalam kode sumber \ref{cdbe.03-02}; 
	\item Logika \textit{view} ditulis menggunakan jQuery yang dicantumkan dalam kode sumber \ref{cdjq.03-02}; dan
	\item Logika \textit{update image}/\textit{storage} data gambar, yang ditulis menggunakan PHP dicantumkan pada kode sumber \ref{cdaws.03-02}
\end{enumerate} 

\begin{lstlisting}[label=cdbe.03-02,style=php,caption=Kode Sumber \textit{Back-end} Memperbarui Barang]
/*
    Fungsi edit() dipanggil untuk menampilkan halaman perbarui barang
    Fungsi update() dipanggil saat form di halaman update barang disubmit
    File : ItemController
*/
public function edit($id)
{
	/*	method : GET */
    $data['item'] = $this->itemRep->find($id);

    /*	Jika barang sedang aktif lelang,
    	tidak dapat diedit (redirect kembali)	*/
    if($data['item']->bid_status==1){
        return redirect()->back()->with('errorItem','Maaf barang yang sedang dilelang tidak dapat diedit!');
    }
    /*	otorisasi	*/
    if($data['item']->id_user!=Auth::user()->id){
        return redirect()->back()->with('errorItem','Maaf anda tidak terotorisasi!');
    }

    return view('pages.'.$this->pageFolder.'.update', $data);
}


/*
    Fungsi update() dipanggil oleh ajax lewat jQuery
    untuk memvalidasi dan update data ke dalam database
*/
public function store(Request $request)
{
    /*  method : POST */
    $unserialize = $this->unserializeForm($request['data']);
    /*  Validating data */
    $validate = Validator::make($unserialize, $this->itemRep->rules(), $this->itemRep->messages());

    if(false){
        /* Jika gagal, return false dan error message 
            ke view
         */
        return ['success'=>false, 'msg' => $validate->messages()];
    }
    else{
        /* 
            Jika sukses, return true 
            dan id barang yang berhasil diinsert
            untuk selanjutnya diproses oleh browser
            untuk upload gambar barang ke URL terpisah.
         */
        return ['success' => true, 'id' =>  10];
    }
}	  
\end{lstlisting}

\begin{lstlisting}[label=cdjq.03-02,style=htmlcssjs,caption=Kode Sumber \textit{View} Memperbarui Barang]
/*	Fungsi ini berfungsi untuk \textit{update} gambar barang, diteruskan kepada \textit{script} terpisah 
*/
init: function() {
    var submitButton = document.querySelector("#click");
    var myDropzone = this; // closure
	submitButton.addEventListener("click", function() {
		 swal({
		   text: 'Memperbarui data barang anda...',
		   allowOutsideClick: false,
		   showConfirmButton: false,
		   onOpen: function (){
		       var dataSubmission = $('.submititem').serialize();
		       var startime=
			       '&start_time=' + 
			       addTimebasedTZ($('#start_time').data().date);
		       var endtime = '&end_time=' + addTimebasedTZ($('#end_time').data().date);
		       dataSubmission += startime;
		       dataSubmission += endtime;
		       $.ajax({
		         type: "POST",
		         url: '{{ route('item.update') }}',
		         data: { _token: '{{csrf_token()}}', data : dataSubmission },
		         success: function( msg ) {
		             console.log (msg);
		             if(msg.success == false){
		                 $('.asdf').css('display','block');
		                 $('.asdf').empty();
		                 $.each(msg.msg, function(i, val){
		                       $('.asdf').append(
		                           '<li>'+val+'</li>'
		                       );
		                   });
		                   swal('Oops',
				                'Maaf, data anda belum valid. Silahkan cek kembali'
				                ,'error');
			
		               }
		               else{
		                 if(myDropzone.files.length >0){
		                     iditem = msg.id;
		                     myDropzone.processQueue();
		                 }
			
		                 else{
				
		                  swal({
		                      title: 'Sukses!',
		                      type:'success',
		                      allowOutsideClick : false,
		                      showConfirmButton : false,
		                      text: "Sukses menambahkan barang!",
		                      timer: 1000
		                  }).then(function () {
		                      document.location = "{{ url('item/') }}" + iditem;
		                  });
		                 }
		               }
		           }
		         });
		     }
		 });
	});

	this.on("processing", function() {
	    swal('Uploading image..');
	});

	this.on("sending", function(file, xhr, data) {
	    if(iditem != 0){
	        data.append("_token", "{{ csrf_token() }}");
	        data.append("itemid", iditem);
	    }
	});
},
/*	
	saat gagal error gambar,
	bagian fungsi error dipanggil
*/
error: function(file, response) {
    swal({
        title: 'Oops',
        type:'error',
        allowOutsideClick: false,
        showConfirmButton:false,
        text: "Maaf ada kesalahan upload gambar, silahkan upload gambar di halaman edit gambar.",
        timer: 2000
    }).then(function () {
        document.location = "{{ url('item') }}/" + iditem;
    });

},
/*	
	saat sukses update gambar,
	bagian fungsi error dipanggil
*/
success: function(file,done) {
    swal({
        title: 'Sukses!',
        allowOutsideClick: false,
        showConfirmButton:false,
        type:'success',
        text: "Sukses memperbarui barang!",
        timer: 1000
    }).then(function () {
        document.location = "{{ url('item/') }}" + iditem;
    });
}

});

\end{lstlisting}


\begin{lstlisting}[label=cdaws.03-02,style=php,caption=Kode Sumber \textit{Back-end} Upload Gambar Barang]
/*	Fungsi ini berfungsi untuk \textit{upload} gambar lewat script terpisah (untuk alasan integrasi data dengan sistem yang dibuat oleh partner tugas akhir)
*/

use Aws\S3\S3Client;

$collection = (new MongoDB\Client("mongodb://127.0.0.1:27017"))->lelangkita->itemimages;
if ($_SERVER['REQUEST_METHOD']=='POST'){
	constructFilePath();	
	$imgpath = $path."/".$filename;

	$decoded_image = base64_decode($image);

  try {
    $s3client = S3Client::factory(array(
      'credentials' => array(
        'key' => 's3\_key\_credentials',
        'secret' => 's3\_secret\_credentials'
      ),
      'profile' => 's3\_profile',
      'region' => 's3\_selected\_server\_region'
    ));

    $upload = $s3client->putObject(
      array(
        'Bucket' => 'img-s7.lelangapa.com',
        'Key' => $imgpath,
        'Body'   => $decoded_image,
        'ContentEncoding' => 'base64',
        'ContentType' => 'image/'.$ext,
        'ACL'    => 'public-read'
      )
    );

    $insertToDB = $collection->insertOne([
                'item_id' => $itemid,
                'url' => $imgpath,
                        'is_main_image' => $isMainImage
        ]);

        if ($isMainImage == true) {
         $indexImageURL = "https://src-api.lelangapa.com/apis/index/submit/image";
         $headers = array(
           'Content-Type: application/json'
         );
         $post_data = array(
                 "id_item" => $itemid,
                 "main_image_url" => $imgpath
         );
         $ch = curl_init();
         curl_setopt($ch, CURLOPT_URL, $indexImageURL);
         curl_setopt($ch, CURLOPT_HTTPHEADER, $headers);
         curl_setopt($ch, CURLOPT_RETURNTRANSFER, 1);
         curl_setopt($ch, CURLOPT_POST, true);
         curl_setopt($ch, CURLOPT_POSTFIELDS, json_encode($post_data));
         curl_setopt($ch, CURLOPT_SSL_VERIFYPEER, FALSE);
         $indexresult = curl_exec($ch);
         curl_close($ch);
        }
    }
    else {
      $collection->updateOne(
        ['_id' => new \MongoDB\BSON\ObjectID($imageid)],
        ['$set' => array('url' => $imgpath, 'is_main_image' => $isMainImage)]
      );
    }

    if ($upload) {
      $res = array('status' => 'success', 'result' => '1');
                echo "success";
    }
    else {
      $res = array('status' => 'failed', 'result' => '0');
                echo "Failed";
    }
  }
  catch(Exception $e) {
    exit($e->getMessage());
  }
}


\end{lstlisting}
