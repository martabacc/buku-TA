\subsubsection{\textsc{Vue.js}}
	% todo inline numbering
	
\begin{enumerate}\bfseries
\item \textbf{Package Dependencies} \\
    \indentenum Pada versi terbaru Laravel (5.4*), Laravel secara \textit{default} menyertakan \textit{package} Laravel Mix - yaitu fitur untuk \textit{compiling assets} dengan Webpack, dengan hasil akhir \textit{compiled assets} (terutama \textit{script} Javascript) yang eksekusinya jauh lebih cepat, karena menggunakan V8 -- sebuah \textit{engine} Javascript yang telah dioptimasi yang bersifat \textit{just-in-time} (JIT) yang memproduksi \textit{machine code} dari sebuah \textit{script} Javascript lalu dieksekusi.\\
  
  \textbf{\textit{Main Problem}} \\
  \indentenum Masalah muncul saat versi Laravel yang digunakan untuk membangun aplikasi adalah versi (5.3) -- dan jika Laravelnya di\textit{upgrade}, tidak ada jaminan bahwa \textit{deprecated dependencies} (keadaan dimana sebuah \textit{package} tidak di\textit{support} oleh versi terbaru) -- yang berarti harus \textit{refactoring code} yang pasti memakan waktu lama.\\			  

	\textbf{\textit{Insights}} \\
	  \indentenum Penulis menganalisa perbedaan mendasar package.json antara Laravel 5.3 dan 5.4 adalah sebagai berikut:
	\begin{enumerate}
  	  	\item Basis : Perubahan basis yang awalnya Gulp menjadi Webpack
  	  	\item \textit{Dependencies} : Webpack ternyata menggunakan beberapa plugin tambahan yang tidak diakomodasi dalam package.json di versi 5.3
  	  	\item \textit{Run Script} : Terdapat beberapa perubahan signifikan terhadap \textit{run script alias} di versi 5.4 - dibandingkan pada versi 5.3.
  	  	\item \textit{Additional Files} : Terdapat beberapa file konfigurasi tambahan agar proses kompilasi aset dapat berjalan dengan baik.
	\end{enumerate}
	\ \\
  
  \textbf{\textit{Solution}} \\
    \indentenum Penulis lalu mengoreksi dan \textit{update package.json} dengan pendekatan \textit{trial and error}, dan bisa terselesaikan dengan script berikut :
		  	
\begin{lstlisting}[language=json]
{
	"private": true,
	"scripts": 	{
		"_comment" : "Lists of running npm commands defined here"
	},
	"devDependencies": {
		"axios": "^0.15.3",
		"bootstrap-sass": "^3.3.7",
		"cross-env": "^3.2.3",
		"jquery": "^3.1.1",
		"laravel-mix": "0.*",
		"lodash": "^4.17.4",
		"vue": "^2.1.10"
	},
	"dependencies": {
		"vue-resource": "^1.3.1"
	}
}
 	\end{lstlisting}
			  	
			  
\item \textbf{\textit{Package Dependencies Optimization}}
		\\ \indent
		Langkah-langkah untuk \textit{compiling assets} adalah sebagai berikut :
		\begin{enumerate}
			\item 
		\end{enumerate}
		Masalah optimasi muncul saat beberapa \textit{script} Vue.js ternyata menggunakan \textit{dependencies} yang sama, yaitu axios, vue dan http.
		\\ \indentenum
	 \end{enumerate}