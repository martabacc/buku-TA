\section{Struktur Dokumen \LaTeX{}}
Dokumen \LaTeX{} terdiri dari struktur yang dibuat berdasarkan struktur dokumen sehari-hari. Sebagai penulis dokumen, Anda wajib menggunakan struktur ini sehingga \LaTeX{} dapat melakukan hal lain yang membantu Anda dalam mengorganisir dokumen seperti misalnya pembuatan Daftar Isi. Berikut adalah struktur dokumen yang ada di \LaTeX{} diurutkan berdasarkan hirarkinya.

\begin{ltabulary}{|L|L|} % L = Rata kiri untuk setiap kolom, | = garis batas vertikal.

% Kepala tabel, berulang di setiap halaman
\caption{Struktur hirarki dokumen \LaTeX{}} \label{tabelStrukturDokumen} \\
\hline
\textbf{Nama} & \textbf{Peruntukkan} \\ \hline

\endhead
\endfoot
\endlastfoot

% Isi Tabel
\textbf{\textbackslash{}part\{Judul Bagian\}} & \texttt{book} \\ \hline
\textbf{\textbackslash{}chapter\{Judul Bab\}} & \texttt{book} dan \texttt{report} \\ \hline
\textbf{\textbackslash{}section\{Judul Subbab\}} & semua kecuali \texttt{letter} \\ \hline
\textbf{\textbackslash{}subsection\{Judul Subsubbab\}} & semua kecuali \texttt{letter} \\ \hline
\textbf{\textbackslash{}subsubsection\{Judul Subsubsubbab\}} & semua kecuali \texttt{letter} \\ \hline
\textbf{\textbackslash{}paragraph\{Judul Paragraf\}} & semua\\ \hline

\end{ltabulary}

\subsection{Pengujian Performa}
      Pengujian 
      \subsubsection{Pengujian Kecepatan Fitur A}
      Pengujian fitur ini dilakukan pada lingkungan uji \ref{env_uji1}, dan untuk lebih lengkapnya dapat dilihat pada tabel \ref{uji1}
      \begin{table}[]
      \centering
      \caption{Pengujian Fitur B}
      \label{uji2}
      \begin{tabular}{llll}
      \multicolumn{1}{c}{\textbf{ID}} & \multicolumn{3}{c}{\textbf{TA-UJI.Proses}}        \\
      Referensi Proses Penggunaan     & \multicolumn{3}{l}{}                              \\
      Nama                            & \multicolumn{3}{l}{}                              \\
      Tujuan Pengujian                & \multicolumn{3}{l}{\multirow{2}{*}{}}             \\
      \textbf{Skenario Pengujian}     & \multicolumn{3}{l}{}                              \\
      Langkah Pengujian               & \multicolumn{3}{l}{}                              \\
      Kecepatan Buka Halaman          & Halaman       & \multicolumn{2}{l}{Google Chrome} \\
                                      & A             & 18 KB           & 0.987s         
      \end{tabular}
      \end{table}
      