\begin{longtable}{|X|X|X|X|}
\caption{Rekapitulasi Hasil Pengujian Pengguna Aplikasi Lelang Online} 
\label{user-test-recap} 
\\

\hline\textbf{Parameter} 
& \textbf{\begin{tabular}[c]{@{}c@{}}Rata-rata\\ Nilai\\ Aplikasi Lain\end{tabular}}
& \textbf{\begin{tabular}[c]{@{}c@{}c@{}}Rata-rata\\ Nilai\\ Aplikasi Lelangapa\end{tabular}} 
& \textbf{\begin{tabular}[c]{@{}c@{}}Persentase \\ Perbedaan \end{tabular}} \\ \hline 
\endfirsthead

\multicolumn{4}{c}%
{\tablename\ \thetable{} -- dilanjutkan dari halaman sebelumnya} \\
\hline\textbf{Parameter} 
& \textbf{\begin{tabular}[c]{@{}c@{}}Rata-rata\\ Nilai\\ Aplikasi Lain\end{tabular}}
& \textbf{\begin{tabular}[c]{@{}c@{}c@{}}Rata-rata\\ Nilai\\ Aplikasi\\	 Lelangapa\end{tabular}} 
& \textbf{\begin{tabular}[c]{@{}c@{}}Persentase \\ Perbedaan \end{tabular}} \\ \hline 
\endhead

\hline \multicolumn{4}{|r|}{{dilanjutkan ke halaman setelahnya}} \\ \hline
\endfoot

\hline
\endlastfoot

Desain \& Impresi Web	&	3,3	&	4,1	&	{\color[HTML]{009901}0,8	(meningkat)}\\ \hline
Kejelasan \& konsistensi sistem	&	3,5	&	4,2	&	{\color[HTML]{009901}0,7 (meningkat)}	\\ \hline
Kemudahan penggunaan	&	3,1	&	3,9	&	{\color[HTML]{009901}0,8 (meningkat)}	\\ \hline
Kejelasan status proses	&	3,7	&	3,9	&	{\color[HTML]{009901}0,2 (meningkat)}	\\ \hline
Error message yang jelas	&	3,3	&	4	&	{\color[HTML]{009901}0,7 (meningkat)}	\\ \hline
Performa	&	3,7	&	3,8	&	{\color[HTML]{009901} 0,1 (meningkat)}	\\ \hline
Rating keseluruhan	&	3,7	&	4,3	&	{\color[HTML]{009901}0,6 (meningkat)}	\\ \hline
Akan rekomendasi aplikasi ini pada teman?	&	3,4	&	4,3	&	{\color[HTML]{009901}0,9 (meningkat)}	\\ \hline

\end{longtable}